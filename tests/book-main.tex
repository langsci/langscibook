\documentclass[output=book
%   ,nonflat
%   ,modfonts,
  ,colorlinks
  ,undecapitalize
  ,collection
  ,showindex
  ,draftmode
  ,openreview
  ,nobabel
  ,booklanguage=french
  ,oldstylenumbers
  ]{langscibook}


% \usepackage{./langsci-basic}
\usepackage{langsci-tbls}
\usepackage{langsci-forest-setup}
\usepackage{langsci-linguex}
\usepackage[nojssambox]{./langsci-gb4e}
\usepackage{langsci-optional}
\usepackage{langsci-tobi}
\usepackage{langsci-lgr}
\usepackage{langsci-plots}


\usepackage{lipsum}
\usepackage{multicol}
\usepackage{minibox}
\usepackage{forest}
\usepackage{tikz}
\usepackage{pgfplots}

% % \usepackage{./langsci-avm}
% \avmoptions{center}
% \avmfont{\scshape}
% \avmvalfont{\normalfont}
% \avmsortfont{\normalfont\itshape}
%
\renewcommand{\lsISBNhardcover}{999-3-123456-99-9}
\renewcommand{\lsISBNsoftcover}{999-3-123456-99-9}
\renewcommand{\lsISBNdigital}{978-3-946234-65-4}
\renewcommand{\lsID}{999}
\renewcommand{\publisherstreetaddress}{LangSci\\Galactic Highway 42\\Olympus Mons}


\bibliography{localbibliography}


\BackBody{
What causes a language to be the way it is? Some features are universal, some are inherited, others are borrowed, and yet others are internally innovated. But no matter where a bit of language is from, it will only exist if it has been diffused and kept in circulation through social interaction in the history of a community. This book makes the case that a proper understanding of the ontology of language systems has to be grounded in the causal mechanisms by which linguistic items are socially transmitted, in communicative contexts. A biased transmission model provides a basis for understanding why certain things and not others are likely to develop, spread, and stick in languages. Because bits of language are always parts of systems, we also need to show how it is that items of knowledge and behavior become structured wholes. The book argues that to achieve this, we need to see how causal processes apply in multiple frames or 'time scales' simultaneously, and we need to understand and address each and all of these frames in our work on language. This forces us to confront implications that are not always comfortable: for example, that "a language" is not a real thing but a convenient fiction, that language-internal and language-external processes have a lot in common, and that tree diagrams are poor conceptual tools for understanding the history of languages. By exploring avenues for clear solutions to these problems, this book suggests a conceptual framework for ultimately explaining, in causal terms, what languages are like and why they are like that.
}

\dedication{Für Alma, Ariel, Block, Frau Brüggenolte, Chopin, Christina, Doro, Edgar, Elena, Elin, Emma, den ehemaligen FCR Duisburg, Frida, Gabriele, Hamlet, Helmut Schmidt, Henry, Ian Kilmister, Ingeborg, Ischariot, Jean-Pierre, Johan, Kurt, Lemmy, Liv, Marina, Martin, Mats, Mausi, Michelle, Nadezhda, Herrn Oelschlägel, Oma, Opa, Pavel, Philly, Sarah, Scully, Stig, Tania, Tante Klärchen, Tarek, Tatjana, Herrn Uhl, Ullis schreckhaften Hund, Vanessa und so. Wenn das schonmal klar sein würde.}


\title{Test for the langsci-* packages}
\author{Lang Uage\and Science\lastand Press}
\renewcommand{\lsYear}{1999}


\renewcommand{\lsSeries}{scl}

\lsCoverTitleSizes{40pt}{15mm}
\begin{document}
\renewcommand{\lsImpressumExtra}{Manuscript submitted in fulfillment of entering the Galactical Hall of Fame.}
\maketitle
\tableofcontents
\mainmatter

\part{Part test}
\chapter{Tests}
\section{langsci-gb4e}
\subsection{Environments and syntactic sugar}
% \begin{exe}
\ex this is a beginexe example
\end{exe}

\begin{exe}
\ex
\gll  this is a beginexe example which goes to the end of the line but not beyond\\
     This Is A Beginexe Example Which Goes To The End Of The Line But Not Beyond\\
\glt  `this is a beginexe example which goes to the end of the line but not beyond'
\end{exe}

 
\ea 
\label{ex:13-227} 
\gll anú eesó míiš ki \textbf{(nu)} \textbf{dhoóṛ} \textbf{yhóol-u} \textbf{de}\\
\textsc{3msg.prox.nom} \textsc{rem.msg.nom} man \textsc{comp} \textsc{3msg.prox.nom} yesterday come.\textsc{pfv"=msg} \textsc{pst}\\
\glt `This line must not flow into the margin.'
\z
 

\begin{exe} 
\ex
\gll anú eesó míiš ki \textbf{(nu)} \textbf{dhoóṛ} \textbf{yhóol-u} \textbf{de}\\
\textsc{3msg.prox.nom} \textsc{rem.msg.nom} man \textsc{comp} \textsc{3msg.prox.nom} yesterday come.\textsc{pfv"=msg} \textsc{pst}\\
\glt `This line must not flow into the margin either.'
\end{exe}



% % % % \exbegin
\ex an exbegin example
\exend

% \ea
  \ea second level given by small a.
  \ex  and by small b.
  \z
\z

% \begin{exe}
 \ex
\begin{xlist}
\ex an xlist example
\ex another xlist example
\end{xlist}

\end{exe}



% \ea
  \ea Dieses war der erste Streich
  \ex  und der zweite folgt sogleich
    \ea und hier der dritte eine Ebene tiefer
    \z
  \z
\z

% \begin{exe}
 \exi{$\pi$} This sentence has the identifier $\pi$
\end{exe}

\begin{exe}
 \ex \label{ex:mylabel} This example has a label \texttt{ex:mylabel}
\end{exe}

\begin{exe}
 \exr{ex:mylabel} This sentence refers to the example with the label \texttt{ex:mylabel} \ref{ex:mylabel}
\end{exe}

\begin{exe}
 \exp{ex:mylabel} This sentence has a prime ' and refers to  the example with the label \texttt{ex:mylabel} \ref{ex:mylabel}
\end{exe}

\begin{exe}
 \sn this sentence is nicely typeset but has no number
\end{exe}

% \ea
\gll ceci est un test ea\\
dem cop indef test ea\\
\glt `This is an ea Test'
\z
% \begin{exe}
\ex
\begin{xlist}
 \ex[*]{wrong sentence this is}
 \ex[?]{This sentence verry questionabl.}
 \ex[\%]{Some people might object to [this over the top heavily modified noun phrase]}
 \ex This sentence is OK.
\end{xlist}
\end{exe}

% % align table
% align parbox
% align tree
% align for subex
 
  \ea
    \parbox[t]{.8\textwidth}{
      \vspace{-.7\baselineskip}
      \begin{tabular}{ll}
	this \\
	is \\
	vertically\\
	large\\
	content
      \end{tabular}
     }
  \z 
% \ea 
\gll Dit is een zin.\\ 
    this is a sentence\\
\glt `This translation has standard vertical space from the gloss'
\z

\nogltOffset

\ea 
\gll Dit is een zin.\\ 
    this is a sentence\\
\glt `This translation has less vertical space from the gloss'
\z

\resetgltOffset

\ea 
\gll Dit is een zin.\\ 
    this is a sentence\\
\glt `This translation has standard vertical space from the gloss again'
\z

% \subsection{linguex}

\Lsciex. abc
\a.  abc
\b. def
  \a. ghi
  \b. jkl
  \b. mno
  

\Lsciex. 
\gll abc def\\
ghi jkl\\
\glt `no sense attached'

 
\Lsciex. 
\ag. fgh ijk\\
     asd fgh\\
\bg. jkl uio\\
     qwe ert\\
     
     
     
\ea this is gb4e
 \ea indented\\
 	\gll abc def\\
         ghi jkl\\
  \z
\z  
     
%
%
% \subsection{Glosses}
% \ea
 \gll dies ist ein gll Test\\
      this is a gll Test\\
 \glt `This is a test for double aligned lines (gll)'
\z



% \ea
 \glll dies ist ein glll Test\\
      this is a glll Test\\
      \textsc{det} \textsc{cop} \textsc{det} \textsc{n} \textsc{n}\\
 \glt `This is a test for triple aligned lines (glll)'
\z
% \ea
 \gllll dies ist ein gllll Test\\
      this is a gllll Test\\
      \textsc{det} \textsc{cop} \textsc{det} \textsc{n} \textsc{n}\\
     1 2 3 4 5\\
 \glt `This is a test for quadruble aligned lines (gllll)'
\z
% \ea
\gll Ein potentieller Mega-Deal in der Industrie sorgt an der B{\"o}rse f{\"u}r Euphorie.\\
        \INDF.\textsc{m}.\SG.\NOM{} potential.\textsc{m}.\SG.\NOM{} mega-deal in the industry care-3\SG.\PRS.\IND{} \LOC{} \DEF.\F.\SG.\DAT{} stock.exchange for euphoria\\
\glt `A potential Mega-Deal in industry causes euphoria at the stock exchange' (Der Spiegel, 2014-04-24) (Very long lines should break nicely)
\z

% \ea
 \gll dies ist ein gll Test mit normaler exewidth\\
      this is a gll Test\\
 \glt `This is a test for double aligned lines (gll)'
\z

\setcounter{xnumi}{123}
\setcounter{equation}{123}

\ea
 \gll dies ist ein ea Test mit langem Label exewidth der in die nächste Zeile überläuft\\
      this is a ea Test with long label e which in the next line overflows\\
 \glt `This is a ea test with a long label which flows over into the next line.'
\z

\begin{exe}
 \ex
 \gll dies ist ein exe Test mit langem Label exewidth der in die nächste Zeile überläuft\\
      this is a exe Test with long label e which in the next line overflows\\
 \glt `This is a exe test with a long label which flows over into the next line.'
\z
\end{exe}


\setcounter{equation}{16} 
%
% \subsection{Footnotes}
% \ea
\gll die {\"U}bersetzung hat eine Fu{\ss}note\\
     the translation has a footnote\\
\glt `The translation has a footnote.\footnote{This is the footnote for the translation}'
\z
% \ea
\gll  hier{\footnotemark} befindet sich eine Fu{\ss}note\\
      here is.situated refl indef footnote\\
\glt `Here is a footnote.'
\z
\footnotetext{Here is the text for the footnotemark.}
% Similar things can be found in German\footnote{%
\eafirst
    \gll Es regnet\\
    it rains\\
    \glt `It rains.'
  \zlast
} and in Spanish\footnote{%
\eafirst
    \ea
      \gll llueve\\
      rains\\
      \glt it rains
    \ex
      \gll nieve\\
      snows\\
      \glt it snows
    \z
  \z
}

% \subsection{Metadata}
% \ea
\langinfo{German}{Indo-European}{personal knowledge}\\
\gll Deutsch ist meine Mutter-sprache\\
     German cop.pres my mother-tongue\\
\glt `German is my mother tongue.'
\z

% \subsection{Boxing}
% \xbox{\textwidth}{
\ea this example spans the whole page as it is very long \z
}

\xbox{.4\textwidth}{
\ea this example accepts another example to its right \z
}
\xbox{.4\textwidth}{
\ea this example accepts another example to its left \z
}\\

% \subsection{Jambox}
% \ea
\gll Ich esse Reis\\
     I eat rice \\\jambox{Assertion}
\glt `I eat rice.'
\z

\ea
\gll Esse ich Reis?\\
     I eat rice \\\jambox{Question}
\glt `Do I eat rice?'
\z

\ea
\gll Reis esse ich.\\
    rice eat I \\\jambox{Topicalization}
\glt `Rice, I eat.'
\z
% \subsection{XPs}
% \obar{N} is N$^0$, but looks nicer\\
\ibar{N} is \={N}, but looks nicer\\
\iibar{N} is $\stackrel{=}{N}$, but looks nicer\\
\mbar{N} is N$^{max}$, but looks nicer\\
\spec{\ibar{N}} specifier of \={N}
% \input{gb4e-tests/crossrefref}
% \input{gb4e-tests/eal}
%
% \subsection{Styles for source line}
% \ea
\gll Dieser Text ist nicht kursiv\\
     this text is not italic\\
\glt `This text is roman'
\z
% \renewcommand{\exfont}{\itshape}
\ea
\gll Dieser Text ist kursiv\\
     This Text is italic\\
\glt `This text is in italics.'
\z
\renewcommand{\exfont}{\upshape}
% The underlinings in the following examples should all be of the same height. The words should also be of the same height.

\ea
\gll This sentence has \ulp{underlined}{8mm} \ule{passages} and \ule{underlined} words\\
 normal normal normal short                     normal normal veryveryveryveryverylong end\\
\z

\ea
\gll \ulp{xxx}{4mm} \ulp{fff}{4mm} \ulp{jjj}{4mm} \ulp{jxf}{4mm} \ule{xxx}  \ule{jjj}  \ule{fjx}  \ule{fff} \ulp{underlined}{3mm} \ule{fassjaes} abd\\
a b c d e  f g h i j k\\
\z
%
% % \section{AVM}
% % \begin{avm}
\@0\[\asort{eating}
actor & \@1 \\ 
theme & \@2 \]
\end{avm}
%
% \section{Maths}
% 
$\frac{n!}{k!(n-k)!} = \binom{n}{k}$

% $\lim_{x \to \infty} \exp(-x) = 0$

$\forall x \in X, \quad \exists y \leq \epsilon$
%
% \section{Trees}
% \Forest{
  [S [NP] 
    [VP [V  [\textit{eats}] ]
      [NP] ]]
}

%
% \section{Fonts}
% The following characters should display nicely

\begin{tabularx}{\textwidth}{Xl l >{\itshape}l>{\bfseries}l>{\scshape}l>{\ttfamily}l>{\sffamily}l}
\lsptoprule
 & code & rm & it & bf & sc & tt & sf \\
\midrule
Schrock\\
\midrule
 Combining Macron-acute & 1DC4 & a᷄ & a᷄ & a᷄ & a᷄ & a᷄ &  a᷄  \\
modifier letter small a & 1D43 & ᵃ & ᵃ & ᵃ & ᵃ & ᵃ & ᵃ \\
modifier letter small e & 1D49 & ᵉ & ᵉ & ᵉ & ᵉ & ᵉ & ᵉ \\
modifier letter small open e& 1D4B & ᵋ & ᵋ & ᵋ & ᵋ & ᵋ & ᵋ \\
modifier letter small capital i& 1DA6 & ᶦ & ᶦ & ᶦ & ᶦ & ᶦ & ᶦ \\
modifier letter small o & 1D52 & ᵒ & ᵒ & ᵒ & ᵒ & ᵒ & ᵒ \\
modifier letter small u & 1D58 & ᵘ & ᵘ & ᵘ & ᵘ & ᵘ & ᵘ \\
\\
Brindle\\
\midrule
modifier letter raised down arrow  & A71C & ꜜ & ꜜ & ꜜ & ꜜ & ꜜ & ꜜ \\
\\
Gabelentz\\
\midrule
latin capital letter egyptological alef & A722 & Ꜣ & Ꜣ & Ꜣ & Ꜣ & Ꜣ & Ꜣ \\
latin small letter egyptological alef & A723 & ꜣ & ꜣ & ꜣ & ꜣ & ꜣ & ꜣ \\
\\
Sasaki, TMNLP TC 3 i\\
\midrule
 & 8B1B & 講 & 講 & 講 & 講 & 講 & 講 \\
 & 8AC7 & 談 & 談 & 談 & 談 & 談 & 談 \\
 & 793E & 社 & 社 & 社 & 社 & 社 & 社 \\
\end{tabularx}



% {\downstep} 
{\↓} 
{\saltillo} 
{\Saltillo} 
{\ꞌ} 
{\Ꞌ} 

{\higha}{ᵃ}
{\highe}{ᵉ}
{\highE}{\textsf{ᵋ}}  
{\highI}
{\higho}{ᵒ}
{\highO}{\textsuperscript{ɔ}}
{\highu}{ᵘ}
{\highU}{\textsuperscript{ʊ}}

% \subsection{Tie bars}
\subsubsection{Heights}
a͡o     
e͡i     
p͡f     
k͡p     
t͡͡ʃ    
H͡L     
x͡x     
X͡X     
x͡I     
I͡x     
I͡I     


\subsubsection{Widths}
xi͡ix    
xm͡mx    
xi͡mx    
xm͡ix   
\\
XI͡IX    
XM͡MX    
XI͡MX    
XM͡IX   
\\       
xI͡ix    
xM͡mx    
xI͡mx    
xM͡ix    
\\        
xi͡Ix    
xm͡Mx    
xi͡Mx    
xm͡Ix    


\subsubsection{With command \textbackslash hitie\{x\}\{y\}}

xx\hitie{i}{i}xx   
xx\hitie{I}{I}xx   
xx\hitie{i}{I}xx   
xx\hitie{I}{i}xx  
\\
xx\hitie{W}{W}xx  
xx\hitie{w}{W}xx  
xx\hitie{W}{w}xx  
\\                 
xx\hitie{I}{m}xx  
xx\hitie{m}{I}xx  
xx\hitie{M}{i}xx  
xx\hitie{i}{M}xx  
\\                  
xx\hitie{x}{M}xx  
xx\hitie{M}{X}xx  
\\                  
xx\hitie{\i}{W}xx  
xx\hitie{W}{\i}xx 
\\                  
xx\hitie{\i}{W}xx 
xx\hitie{W}{\i}xx \\

\subsubsection{With command \textbackslash hitier\{x\}\{y\}/\textbackslash hitiel\{x\}\{y\}}
xx\hitier{\i}{W}xx           
xx\hitiel{W}{\i}xx          
\\
with optional argument to adjust horizontal placement\\
xx\hitier[.9]{\i}{W}xx      
xx\hitiel[.9]{W}{\i}xx      
% % \input{font-tests/numbers.tex} %please check in file
%
% \section{Figures}\label{sec:tables}
%
% \section{Tables}\label{sec:tables}
% Tables are discussed in \sectref{sec:tables}. \tabref{tab:example} is a table and \figref{fig:example} is a figure 

\begin{table}
 \begin{tabular}{c|c}
  a & b\\
\midrule
c & d\\
 \end{tabular}
\caption{example table}
\label{tab:example}
\end{table}

\begin{figure}
 \begin{tabular}{c|c}
  a & b\\
\midrule
c & d\\
 \end{tabular}
\caption{example figure}
\label{fig:example}
\end{figure}

%
% \section{Crossrefs}
% Tables are discussed in \sectref{sec:tables}. \tabref{tab:example} is a table and \figref{fig:example} is a figure 

\begin{table}
 \begin{tabular}{c|c}
  a & b\\
\midrule
c & d\\
 \end{tabular}
\caption{example table}
\label{tab:example}
\end{table}

\begin{figure}
 \begin{tabular}{c|c}
  a & b\\
\midrule
c & d\\
 \end{tabular}
\caption{example figure}
\label{fig:example}
\end{figure}

%
% \section{Bibliography}
% \subsection{Plain}
 
\bigskip
\verb+\citet{Chomsky1957}+

        \citet{Chomsky1957}            


\bigskip
\verb+\citet{Chomsky1957,Comrie1981}+

       \citet{Chomsky1957,Comrie1981} 


\bigskip
\verb+\citep{Chomsky1957}+

       \citep{Chomsky1957}            


\bigskip
\verb+\citep{Chomsky1957,Comrie1981}+

      \citep{Chomsky1957,Comrie1981} 


\bigskip
\verb+\citealt{Chomsky1957}+

     \citealt{Chomsky1957}          


\bigskip
\verb+\citealt{Chomsky1957,Comrie1981}+

    \citealt{Chomsky1957,Comrie1981}


\bigskip
\verb+\citeauthor{Chomsky1957}+

   \citeauthor{Chomsky1957}         

  


\bigskip
\verb+\citeyear{Chomsky1957}+

   \citeyear{Chomsky1957}          

  

\bigskip
\verb+\citegen{Chomsky1957}+

    \citegen{Chomsky1957}          

   

 \subsection{Pages}
 
\bigskip
\verb+\citet[12]{Chomsky1957}+

        \citet[12]{Chomsky1957}            

 


\bigskip
\verb+\citep[12]{Chomsky1957}+

       \citep[12]{Chomsky1957}            

 


\bigskip
\verb+\citealt[12]{Chomsky1957}+

     \citealt[12]{Chomsky1957}          

 
 
  


\bigskip
\verb+\citeyear[12]{Chomsky1957}+

   \citeyear[12]{Chomsky1957}          

  

\bigskip
\verb+\citegen[12]{Chomsky1957}+

    \citegen[12]{Chomsky1957}          

   

 \subsection{Two arguments}
  


\bigskip
\verb+\citep[see][12]{Chomsky1957}+

       \citep[see][12]{Chomsky1957}            

 


\bigskip
\verb+\citealt[see][12]{Chomsky1957}+

     \citealt[see][12]{Chomsky1957}          

 
 
   
   

 
 
\subsection{Same author}
\bigskip
\verb+\citet{Chomsky1957}+

        \citet{Chomsky1957}            


\bigskip
\verb+\citet{Chomsky1957,Chomsky1965aspects}+

       \citet{Chomsky1957,Chomsky1965aspects} 

 
 
\subsection{AAB}
\bigskip
\verb+\citet{Chomsky1957}+

        \citet{Chomsky1957}            


\bigskip
\verb+\citet{Chomsky1957,Chomsky1965aspects,Comrie1981}+

       \citet{Chomsky1957,Chomsky1965aspects,Comrie1981} 

 
 
\subsection{ABA}
\bigskip
\verb+\citet{Chomsky1957}+

        \citet{Chomsky1957}            


\bigskip
\verb+\citet{Chomsky1957,Comrie1981,Chomsky1965aspects}+

       \citet{Chomsky1957,Comrie1981,Chomsky1965aspects} 

 
 
\subsection{AAA} 

\bigskip
\verb+\citet{Chomsky1957,Chomsky1965aspects,Chomsky1965cartesian}+

       \citet{Chomsky1957,Chomsky1965aspects,Chomsky1965cartesian} 
 
\subsection{AABA} 
\bigskip
\verb+\citet{Chomsky1957,Chomsky1965aspects,Comrie1981,Chomsky1965cartesian}+

       \citet{Chomsky1957,Chomsky1965aspects,Comrie1981,Chomsky1965cartesian} 
\bigskip
\verb+\citealt{Chomsky1957,Chomsky1965aspects,Comrie1981,Chomsky1965cartesian}+

       \citealt{Chomsky1957,Chomsky1965aspects,Comrie1981,Chomsky1965cartesian} 

 
\section{Ordering of names with diacritics} 
\citet{Circov1900,MeierCircovac1900}

\section{Cite work in same volume}

\citetv{Chomsky1957}

\citepv{Chomsky1957}


\section{Different entry types}
\subsection{incollection}
\citet{Meier2000}

\section{Bib fields}

The URL should be available in book entries like \citet{Url2001}.

There should be no warnings issued for entries with a complex date like \citet{ComplexYear2000}.

\citet{
bookwithnote,
articlewithnote,
incollectionwithnote,
inproceedingsnwithnote,
miscwithnote,
websitewithnote,
electronicwithnote,
unpublishedwithnote, 
inbookwithnote} all have notes.

\citet{rfc1654} is a standard with a nice reference in the prose text. Check that this standard is listed in the list of references
% % \input{bib-tests/index}
%
% \printbibliography[notkeyword={techreport},notkeyword={website},title={References}]
%
% next thing up is list of standards, included with \verb+keywords={standard},+ in the bib file.
%
% \printbibliography[keyword={standard},title={Standards}]
%
%
%
% next thing up is list of websites, included with \verb+keywords={website},+ in the bib file.
%
% \printbibliography[keyword={website},title={Websites}]
%
%
%
% \section{Index}
% This indexes \ili{German}

This indexes \isi{Noun phrase}
\footnote{The \isi{verb phrase} is in the footnote.}

This indexes \iai{John Malkovich}

\il{Deutsch| see{German}}
\is{NP| see{Noun phrase}}
\ia{Malkovich| see{John Malkovich}}



\ilsa{German}{Germanic}
\issa{Noun phrase}{Nouns}

%
% \section{Diagrams}
% \section{Plots}

\begin{figure} 
  \barplot{Person}{\%}{P01,P02,P03}{
	      (P01,19.47733441) 
	      (P02,04.99311069) 
	      (P03,01.22486586)
  }
  \caption{Ratio of fixation time in the caption area in relation to fixation time to the whole screen}
  \label{fig:barplot}
\end{figure}


\langsciplot{data}{}

\langsciplot{data}{ybar}

\langsciplot{data}{xbar}


\begin{tikzpicture}[trim axis right,trim axis left]
    \begin{axis}[
        xlabel={TMNLP},
        ylabel={blue},
        axis lines*=left,
        width  = \textwidth,
        height = .3\textheight,
        nodes near coords,
        xtick=data,
        x tick label style={},
        ybar,
        bar width=.4cm,
        ymin=0,
        symbolic x coords={A,B,C}
        ]
        \addplot[ybar,tmnlpone!80!black,fill=tmnlpone] plot coordinates {
            (A,6)
            (B,2)
            (C,1)
            };
        \addplot[ybar,tmnlptwo!80!black,fill=tmnlptwo] plot coordinates {
            (A,3)
            (B,4)
            (C,2)
            };
        \addplot[ybar,tmnlpthree!80!black,fill=tmnlpthree] plot coordinates {
            (A,4)
            (B,3)
            (C,5)
        };
        \addplot[ybar,tmnlpfour!80!black,fill=tmnlpfour] plot coordinates {
            (A,5)
            (B,2)
            (C,1)
        };
    \end{axis}
\end{tikzpicture}


\begin{tikzpicture}[trim axis right,trim axis left]
    \begin{axis}[
        xlabel={SIDL},
        ylabel={green},
        axis lines*=left,
        width  = \textwidth,
        height = .3\textheight,
        nodes near coords,
        xtick=data,
        x tick label style={},
        ybar,
        bar width=.3cm,
        ymin=0,
        symbolic x coords={A,B,C}
        ]
        \addplot[ybar,sidlone!80!black,fill=sidlone] plot coordinates {
            (A,6)
            (B,2)
            (C,1)
            };
        \addplot[ybar,sidltwo!80!black,fill=sidltwo] plot coordinates {
            (A,3)
            (B,4)
            (C,2)
            };
        \addplot[ybar,sidlthree!80!black,fill=sidlthree] plot coordinates {
            (A,4)
            (B,3)
            (C,5)
        };
        \addplot[ybar,sidlfour!80!black,fill=sidlfour] plot coordinates {
            (A,5)
            (B,2)
            (C,1)
        };
    \end{axis}
\end{tikzpicture}


\begin{tikzpicture}[trim axis right,trim axis left]
    \begin{axis}[
        xlabel={SILP},
        ylabel={red},
        axis lines*=left,
        width  = \textwidth,
        height = .3\textheight,
        nodes near coords,
        xtick=data,
        x tick label style={},
        ybar,
        bar width=.4cm,
        ymin=0,
        symbolic x coords={A,B,C}
        ]
        \addplot[ybar,silpone!80!black,fill=silpone] plot coordinates {
            (A,6)
            (B,2)
            (C,1)
            };
        \addplot[ybar,silptwo!80!black,fill=silptwo] plot coordinates {
            (A,3)
            (B,4)
            (C,2)
            };
        \addplot[ybar,silpthree!80!black,fill=silpthree] plot coordinates {
            (A,4)
            (B,3)
            (C,5)
        };
        \addplot[ybar,silpfour!80!black,fill=silpfour] plot coordinates {
            (A,5)
            (B,2)
            (C,1)
        };
    \end{axis}
\end{tikzpicture}


\begin{tikzpicture}[trim axis right,trim axis left]
    \begin{axis}[
        xlabel={LV},
        ylabel={brown},
        axis lines*=left,
        width  = \textwidth,
        height = .3\textheight,
        nodes near coords,
        xtick=data,
        x tick label style={},
        ybar,
        bar width=.4cm,
        ymin=0,
        symbolic x coords={A,B,C}
        ]
        \addplot[ybar,lvone!80!black,fill=lvone] plot coordinates {
            (A,6)
            (B,2)
            (C,1)
            };
        \addplot[ybar,lvtwo!80!black,fill=lvtwo] plot coordinates {
            (A,3)
            (B,4)
            (C,2)
            };
        \addplot[ybar,lvthree!80!black,fill=lvthree] plot coordinates {
            (A,4)
            (B,3)
            (C,5)
        };
        \addplot[ybar,lvfour!80!black,fill=lvfour] plot coordinates {
            (A,5)
            (B,2)
            (C,1)
        };
    \end{axis}
\end{tikzpicture}

%
% \section{Floats}
%  

\lipsum[1]

\begin{table}
 \begin{tabular}{lll}
\lsptoprule
  a & b c & d\\
  1 2 & 3 & 4 \\
\lspbottomrule
 \end{tabular}
\caption{This is a short caption.}
\end{table}

\lipsum[2]

\begin{table}
 \begin{tabular}{lll}
\lsptoprule
  a & b c & d\\
  4 5 & 6 & 7 \\
\lspbottomrule
 \end{tabular}
\caption{This is a very long caption which stretches over several lines. It includes additional explanations of the table and helps the reader to interpret the content. It could be shorter, but here, it is really important that it is long.}
\end{table}


\lipsum[7]

\begin{table}
 \begin{tabular}{lllrllrlrll}
\lsptoprule
  a & b c & d& b c & d& b c & d& b c & d\\
  4 5 & 6 & 7 & 6 & 7 & 6 & 7 & 6 & 7 \\
  a & b c & d& b c & d& b c & d& b c & d\\
  4 5 & 6 & 7 & 6 & 7 & 6 & 7 & 6 & 7 \\
  a & b c & d& b c & d& b c & d& b c & d\\
  4 5 & 6 & 7 & 6 & 7 & 6 & 7 & 6 & 7 \\
  a & b c & d& b c & d& b c & d& b c & d\\
  4 5 & 6 & 7 & 6 & 7 & 6 & 7 & 6 & 7 \\
\lspbottomrule
 \end{tabular}
\caption{This is a very long caption which stretches over several lines. It includes additional explanations of the table and helps the reader to interpret the content. It could be shorter, but here, it is really important that it is long.}
\end{table}


\lipsum[8] 

\begin{table}
 \begin{tabularx}{\textwidth}{llXXXXXXXXXXXll}
\lsptoprule
  a & b c &   a & b c &   a & b c & d& b c & d& b c & d& b c & d\\
  4 5 & 6 &   4 5 & 6 &   4 5 & 6 & 7 & 6 & 7 & 6 & 7 & 6 & 7 \\
  a & b c &   a & b c &   a & b c & d& b c & d& b c & d& b c & d\\
  4 5 & 6 &   4 5 & 6 &   4 5 & 6 & 7 & 6 & 7 & 6 & 7 & 6 & 7 \\
  a & b c &   a & b c &   a & b c & d& b c & d& b c & d& b c & d\\
  4 5 & 6 &   4 5 & 6 &   4 5 & 6 & 7 & 6 & 7 & 6 & 7 & 6 & 7 \\
  a & b c &   a & b c &   a & b c & d& b c & d& b c & d& b c & d\\
  4 5 & 6 &   4 5 & 6 &   4 5 & 6 & 7 & 6 & 7 & 6 & 7 & 6 & 7 \\
  a & b c &   a & b c &   a & b c & d& b c & d& b c & d& b c & d\\
  4 5 & 6 &   4 5 & 6 &   4 5 & 6 & 7 & 6 & 7 & 6 & 7 & 6 & 7 \\
  a & b c &   a & b c &   a & b c & d& b c & d& b c & d& b c & d\\
  4 5 & 6 &   4 5 & 6 &   4 5 & 6 & 7 & 6 & 7 & 6 & 7 & 6 & 7 \\
  a & b c &   a & b c &   a & b c & d& b c & d& b c & d& b c & d\\
  4 5 & 6 &   4 5 & 6 &   4 5 & 6 & 7 & 6 & 7 & 6 & 7 & 6 & 7 \\
  a & b c &   a & b c &   a & b c & d& b c & d& b c & d& b c & d\\
  4 5 & 6 &   4 5 & 6 &   4 5 & 6 & 7 & 6 & 7 & 6 & 7 & 6 & 7 \\
\lspbottomrule
 \end{tabularx}
\caption{This is a very long caption which stretches over several lines. It includes additional explanations of the table and helps the reader to interpret the content. It could be shorter, but here, it is really important that it is long.}
\end{table}


\lipsum[3] 

\begin{figure}
\fbox{

\parbox{5cm}{abc\\

\hspace{3cm}def \\

\hspace{2cm}g~~hi

\vspace{3cm}y\\

\rule{.4\textwidth}{3pt} \\g}}

\caption{This is a short figure caption.}
\end{figure}

\lipsum[4-5]

\begin{figure}
 \fbox{\fbox{\parbox{5cm}{\Huge ~~~~def\\ \Large X \fbox{D} EF}}}
\caption{This is a very long caption dwelling on several aspects of this figure. It is rather extensive, but helps the reader to better make sense of this involved diagram.}
\end{figure}

\lipsum[5]



\begin{figure}
\fbox{

\parbox{\textwidth}{abc\\

\hspace{4cm}afasd \\

\hspace{1cm}agagdsg~~hi

\vspace{2cm}y\\

\rule{.4\textwidth}{3pt} \\g

\rule{.2\textwidth}{3pt} \\f

}
}

\caption{This is a very long caption dwelling on several aspects of this figure. It is rather extensive, but helps the reader to better make sense of this involved diagram.}
\end{figure}

\lipsum[4-5]

\begin{figure}
 \fbox{\fbox{\parbox{\textwidth}{\Huge ~~~~dref\\ \Large X \fbox{D EF}}}}
\caption{This is a short figure caption.}
\end{figure}

\lipsum[5]



%
%
% % \input{intonation-tests/tobi}
%
% \section{Langsci-lgr tests}
% \ABL
\ABS
\ACC
\ADJ
\ADV
\AGR
\ALL
\ANTIP 
\APPL  
\ART
\AUX
\BEN
\CAUS
\CLF
\COM
\COMP
\COMPL 
\COND
\COP
\CVB
\DAT
\DECL
\DEM
\DEF
\DET
\DIST
\DISTR 
\DU
\DUR
\ERG
\EXCL
\F
\FOC
\FUT
\GEN
\IMP
\INCL
\IND
\INDF
\INS
\INTR
\IPFV
\IRR
\LOC
% \M %
\N
\NEG
\NMLZ
\NOM
\OBJ
\OBL
% \P %
\PASS
\PFV
\PL
\POSS
\PRED
\PRF
\PRS
\PROG
\PROH
\PROX
\PST
\PTCP
\PURP
\Q 
\QUOT
\RECP
\REFL
\REL
\RES
% \S  %
\SBJ
\SBJV
\SG
\TOP
\TR
\VOC                    

\ea
\gll ceci n' est pas une pomme\\
     \DET{} \NEG{} \COP{} \NEG{} \INDF{} apple\\
\glt `this is not an apple'     
\z
%
% \section{Langsci-basic tests}
% % \input{package-tests/basic}
%
% \section{Langsci-optional tests}
% % \input{package-tests/optional}
%
% \section{Textbooks tests}
% \subsection{Simple boxes}

\lipsum[1]

\tblsbwboxlight{
\boxheader{This box is light gray}
\lipsum[2]
}

\lipsum[14]
 
\tblsbwboxdark{%
\boxheader{This box is dark gray}
\lipsum[3]
}
 
\subsection{Lined boxes}

\lipsum[15]

\tblsbwthinsandwich{
\boxheader{This box has thin lines around it}
\lipsum[4]
}

\lipsum[5]

\tblsbwthicksandwich{
\boxheader{This box has thick lines around it}
\lipsum[6]
}


\subsection{Floating boxes}
\lipsum[7]

\tblsbwboxlight{
\boxheader{This box is inline}
\lipsum[8]
}

\begin{figure}
\color{red} This dummy figure is used to check whether the floating box interferes with figure numbering
 \caption{Dummy figure for counting}
\label{fig:dummycountfiguretbls}
\end{figure}


\begin{figure}
\tblsbwboxlight{
\boxheader{This box floats to the top}
\lipsum[9]
}
\end{figure}

\lipsum[11-12]

\begin{figure}
\color{red}The caption of this figure is only 1 higher than \figref{fig:dummycountfiguretbls}
 \caption{Dummy figure for counting. The uncaptioned figure with the floating box is not counted.}
\label{fig:dummycountfiguretbls2}
\end{figure}


\subsection{Boxes with icons}
\lipsum[13-14]


\tblsbwboxlight[glass]{
\boxheader{This light box has a looking glass icon}
\lipsum[15]
}

\lipsum[16]
 
\tblsbwboxdark[bulb]{%
\boxheader{This dark box has a bulb icon}
\lipsum[17]
}

\subsection{Multipage boxes}


\lipsum[18]

\tblsbwthinsandwich{
\boxheader{This box has thin lines around it. The lines repeat on page breaks}
\lipsum[19-23]
}

\lipsum[24-25]

\tblsbwthicksandwich{
\boxheader{This box has thick lines around it. The lines repeat on page breaks}
\lipsum[26-30]
}

\lipsum[31]
% \lipsum[1]

\begin{tblscolboxlight}
\boxheader{This box is light gray}

\lipsum[2]
\end{tblscolboxlight}
 

\begin{tblscolboxdark}
\boxheader{This box is dark gray}

\lipsum[2]
\end{tblscolboxdark}
 
 
\lipsum[3]

\begin{tblscolframebox}
\boxheader{This box has a color frame}


\lipsum[4]
\end{tblscolframebox}
\lipsum[5]


\begin{tblscolthinsandwich}
\boxheader{This box has thin lines above and below}

\lipsum[6]
\end{tblscolthinsandwich}
\lipsum[7]

\begin{tblscolthicksandwich}
\boxheader{This box has thick lines above and below}
\lipsum[8]
\end{tblscolthicksandwich}
\lipsum[9]


% \lipsum[1]

\tblssy{book}{Literaturhinweise (mit \texttt{\textbackslash tblssy})}{\lipsum[1]}

\tblssy{bulb}{Übungsaufgaben (mit \texttt{\textbackslash tblssy})}{\lipsum[42]}

\tblssy{law}{Definition (mit \texttt{\textbackslash tblssy})}{\lipsum[2]}

\tblssy{glass}{Ideen für die eigene Forschung (mit \texttt{\textbackslash tblssy})}{\lipsum[2]}

\tblsli{.8}{Test (mit \texttt{\textbackslash tblsli})}{\lipsum[22]}

\tblsfi{black!20}{Another test (mit \texttt{\textbackslash tblsfi})}{Another text}

\tblsfr{lsYellow}{book}{Literaturhinweise (mit \texttt{\textbackslash tblsfr})}{\lipsum[5]}




% \section{Syntactic sugar}
% 
Important contributions are found in \citet{Meier2022}\missref{Meier2022}.\footnote{See also \citet{Meier2033}\missref[inline]{Meier2033}}

This is a rather superfluous sentence with a rather problematic line break \rephrase{which}{that } has a suggestion how to change it.
\backmatter
\phantomsection%this allows hyperlink in ToC to work
\printbibliography[heading=references] 
\cleardoublepage

\phantomsection 
\addcontentsline{toc}{chapter}{Index} 
\addcontentsline{toc}{section}{\lsNameIndexTitle}
\ohead{Name \lsNameIndexTitle} 
\printindex 
\cleardoublepage
  
\phantomsection 
\addcontentsline{toc}{section}{\lsLanguageIndexTitle}
\ohead{\lsLanguageIndexTitle} 
\printindex[lan] 
\cleardoublepage
  
\phantomsection 
\addcontentsline{toc}{section}{\lsSubjectIndexTitle}
\ohead{\lsSubjectIndexTitle} 
\printindex[sbj]
\ohead{} 

\end{document}
