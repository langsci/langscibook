\documentclass[output=minimal]{../../langscibook}
\title[]{Distinct featural classes of \textit{anaphor} in an enriched person system}
\author{%
 Sandhya Sundaresan  \affiliation{Universität Leipzig}
}

% \epigram{Change epigram in chapters/01.tex or remove it there }

\abstract{This paper tackles the fundamental question of what an
  anaphor actually is -- and asks whether the label ``anaphor'' even
  carves out a homogenous class of element in grammar.  While most
  theories are in agreement that an anaphor is an element that is
  referentially deficient in some way, the question of how this might
  be encoded in terms of deficiency for syntactic features remains
  largely unresolved. The conventional wisdom is that anaphors lack
  some or more φ-features. A less mainstream view proposes that
  anaphors are deficient for features that directly target
  reference. Here, I present different types of empirical evidence
  from a range of languages to argue that neither approach gets the
  full range of facts quite right. The role of \textsc{person}, in particular,
  seems to be privileged. Some anaphors wear the empirical properties
  of a \textsc{person}-defective nominal; yet others, however, are sensitive
  to \textsc{person}-restrictions in a way that indicates that they are
  inherently specified for \textsc{person}. Orthogonal to these are anaphors
  whose distribution seems to be regulated, not by φ-features at
  all, but by perspective-sensitivity. Anaphors must, then, not be
  created equal, but be distinguished along featural classes. I
  delineate what this looks like against a binary feature system for
  \textsc{person} enriched with a privative [\textsc{sentience}] feature. The
  current model is shown to make accurate empirical predictions for
  anaphors that are \emph{in}sensitive to \textsc{person}-asymmetries for the
  PCC, animacy effects for anaphoric agreement, and instances of
  non-matching for \textsc{num} and \textsc{person}.}

\begin{document}
\maketitle


\section{Introduction}
Phasellus maximus erat ligula, accumsan rutrum augue facilisis in. Proin sit amet pharetra nunc, sed maximus erat. Duis egestas mi eget purus venenatis vulputate vel quis nunc. Nullam volutpat facilisis tortor, vitae semper ligula dapibus sit amet. Suspendisse fringilla, quam sed laoreet maximus, ex ex placerat ipsum, porta ultrices mi risus et lectus. Maecenas vitae mauris condimentum justo fringilla sollicitudin. Fusce nec interdum ante. Curabitur tempus dui et orci convallis molestie \citep{Chomsky1957}. 

\kant[1-4]

\begin{table}
\caption{A table}
\end{table}


\ea An example \z

\section{Another section}

\section*{Abbreviations}
\begin{tabularx}{.45\textwidth}{lX}
\textsc{cop} & copula\\
\textsc{fv} & final vowel\\
\end{tabularx}
\begin{tabularx}{.45\textwidth}{lX}
\textsc{neg} & negation\\
\textsc{sm} & subject marker\\
\end{tabularx}

We reference Chapter 2 as Chapter \ref{sec:2}

\section*{Acknowledgements}

{\sloppy
\printbibliography[heading=subbibliography,notkeyword=this]
}
\end{document}
