%% -*- coding:utf-8 -*-
\chapter{\LaTeX}

\section{Installation of the \texttt{langsci} Class}

The \latex class for typesetting Language Science Press books was developed by Timm Lichte with
help be Berthold Crysmann and me. It can be downloded from the GitHUB repository at: \url{https://github.com/langsci/latex}

\section{Using the \texttt{langsci} Class}

Once you installed the classes in your system, you may look at the file \texttt{test.tex} to see how
a book can be typeset. The code of this book is available in the directory \texttt{Guidelines}. Once
you set up your \latex files you can compile them by calling 
\begin{verbatim}
xelatex yourfilename.tex
\end{verbatim}


\section{Workflow}

\subsection{Compiling the Document}

\subsection{Makefiles}

\subsection{Using Includes}


\section{Document Structure}


\section{Packages specific for Linguistics}

\subsection{AVMs}

  \begin{avm}
    {\it word\/} $\rightarrow$
    \[morphs & $\@{e_1}\bigcirc\cdots\bigcirc\@{e_n}$\\
    morsyn & \@0 $(\@{m_1}\uplus\cdots\uplus\@{m_n})$\\
    rules & \< \[morphs & \@{e_1}\\mud & \@{m_1}\\ morsyn & \@0\],\ldots,
    \[morphs & \@{e_n}\\mud & \@{m_n}\\ morsyn & \@0\] \>
    \]
  \end{avm}

\subsection{Trees}

\subsection{OT Tableaux}

\subsection{Font Issues and Right to Left Scripts}


  
\section{Bells and Whistles}

\subsection{\texttt{varioref}}

\subsection{\texttt{german} for Hyphenation}

\section{Software}

\begin{itemize}
\item BibDesk
\item JabRef
\end{itemize}


\section{Proper Installation}




%      <!-- Local IspellDict: en_US-w_accents -->
