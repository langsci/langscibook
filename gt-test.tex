\documentclass[ number=45
			   ,series=eotms
			   ,printondemand
			   %,blackandwhite
			   %,draft=yes
			  ]{langsci}                          

\usepackage{etex}
                               
\usepackage{layout}
\usepackage{lipsum}                                                                             


\usepackage{ngerman}\selectlanguage{german}
\usepackage[german]{varioref}
% do not stop and warn! This will be tested in the final version
%\vrefwarning


\usepackage{graphicx}




\usepackage{article-ex,makros.2e,
de-date,de-commands,tree-dvips,pstricks,pst-node,
datefooter,float,my-theorems,mycommands}


\usepackage{index}
\makeatletter
% Wie im Stylefile, aber ohne \MakeUppercase
   \renewenvironment{theindex}{%
        \edef\indexname{\the\@nameuse{idxtitle@\@indextype}}%
        \if@twocolumn
            \@restonecolfalse
        \else
            \@restonecoltrue
        \fi
        \columnseprule \z@
        \columnsep 35\p@
        \twocolumn[%
            \@makeschapterhead{\indexname}%
            \ifx\index@prologue\@empty\else
                \index@prologue
                \bigskip
            \fi
        ]%
%        \@mkboth{\MakeUppercase\indexname}%
%                {\MakeUppercase\indexname}%
        \@mkboth{\indexname}%
                {\indexname}%
        \thispagestyle{plain}%
        \parindent\z@
        \parskip\z@ \@plus .3\p@\relax
        \let\item\@idxitem
    }{%
        \if@restonecol
            \onecolumn
        \else
            \clearpage
        \fi
    }
\makeatother

\makeindex
\newindex{aut}{adx}{and}{Namensverzeichnis}
\newindex{lan}{ldx}{lnd}{Verzeichnis der Sprachen}
\renewindex{default}{idx}{ind}{Sachregister}
\usepackage{my-index-shortcuts}

\usepackage{authorindex}

\let\mc=\multicolumn

% use article style since csli does not want new numbers in every chapter
\usepackage{my-gb4e-article}
\exewidth{\exnrfont (34)}

%\def\ea{\ifnum\@xnumdepth=0\begin{exe}[(234)]\else\begin{xlist}[iv.]\fi\ex}
%\def\ea{\ifnum\@xnumdepth=0\begin{exe}[(\theequation)]\else\begin{xlist}[iv.]\fi\\ex}
%\def\eas{\begin{samepage}\ifnum\@xnumdepth=0\begin{exe}[(234)]\else\begin{xlist}[\iv.]\fi\ex}
%\def\eal{\begin{exe}[(234)]\ex\begin{xlist}[iv.]}

\nodemargin5pt%\treelinewidth2pt\arrowwidth6pt\arrowlength10pt
\psset{nodesep=5pt} %,linewidth=0.8pt,arrowscale=2}
\psset{linewidth=0.5pt}
\setcounter{secnumdepth}{4}

\newcommand{\page}{S.\,}
\let\citew=\citealp
\newcommand{\NOTE}[1]{}
%\newcommand{\NOTE}[1]{\marginpar{#1}}
\newcommand{\LATER}[1]{}


\title{Danish in \newlineCover Head-Driven \newlineCover\newlineSpine Phrase Structure \newlineCover Grammar  }                        
\author{Stefan M\"uller, \newlineCover Pollet Samvelian, \newlineCover Olivier Bonami}
%\BackTitle{title text of the back page}
%\BackBody{body text of the back page}                                            
 
\begin{document}              
       
                
      
\maketitle                

\tableofcontents


\pagestyle{scrheadings}
  
\part{Part Title}	               



\chapter{Einleitung und Grundbegriffe}
\label{Kapitel-Grundbegriffe}

In diesem Kapitel soll erklärt werden, warum man sich überhaupt mit Syntax beschäftigt
(Abschnitt~\ref{sec-wozu-syntax}) und warum es sinnvoll ist, die Erkenntnisse zu formalisieren
(Abschnitt~\ref{sec-formal}).  Einige Grundbegriffe werden in den
Abschnitten~\ref{konstituententests}--\ref{sec-topo} eingeführt: Abschnitt~\ref{konstituententests}
beschäftigt sich mit Kriterien für die Unterteilung von Äußerungen in kleinere
Einheiten. Abschnitt~\ref{Abschnitt-Wortarten} zeigt, wie man Wörter in Klassen einteilen kann,
\dash, es werden Kriterien dafür vorgestellt, wann ein Wort einer Wortart wie \zb Verb oder Adjektiv
zugeordnet werden kann.  Abschnitt~\ref{Abschnitt-Kopf} stellt den Kopf"=Begriff vor, in
Abschnitt~\ref{Abschnitt-Argument-Adjunkt} wird der Unterschied zwischen Argumenten und Adjunkten
erklärt, Abschnitt~\ref{Abschnitt-GF} definiert grammatische Funktionen und
Abschnitt~\ref{Abschnitt-Toplogie} führt die topologischen Felder zur Bezeichnung von Satzbereichen
ein.

Leider ist die Linguistik eine Wissenschaft, in der es ein unglaubliches terminologisches Chaos
gibt. Das liegt zum Teil daran, dass Begriffe für einzelne Sprachen (\zb Latein\il{Latein}, Englisch\il{Englisch})
definiert und dann einfach für die Beschreibung anderer Sprachen übernommen wurden. Da sich
Sprachen mitunter stark unterscheiden und auch ständig verändern, ist das nicht immer
angebracht. Wegen der sich daraus ergebenden Probleme werden die Begriffe dann anders verwendet,
oder es werden neue erfunden. 
%
Bei der Einführung neuer Begriffe werde ich deshalb auf verwandte Begriffe oder abweichende
Verwendung des jeweils eingeführten Begriffs hinweisen, damit der Leser die Verbindung zu anderer
Literatur herstellen kann. 

\section{Wozu Syntax?}
\label{sec-wozu-syntax}

Die\il{Deutsch|(} sprachlichen Ausdrücke, die wir verwenden, haben eine Bedeutung. Es handelt
sich um sogenannte Form"=Bedeutungs"=Paare \citep{Saussure16a}. Dem Wort \emph{Baum}
mit seiner bestimmten orthographischen Form oder einer entsprechenden Aussprache
wird die Bedeutung \relation{baum} zugeordnet. Aus kleineren sprachlichen Einheiten
können größere gebildet werden: Wörter können zu Wortgruppen verbunden werden
und diese zu Sätzen.

Die Frage, die sich nun stellt, ist folgende: Braucht man ein formales System, das
diesen Sätzen eine Struktur zuordnet? Würde es nicht ausreichen, so wie wir
für \emph{Baum} ein Form"=Bedeutungs"=Paar haben, entsprechende Form"=Bedeutungs"=Paare
für vollständig ausformulierte Sätze aufzuschreiben? Das wäre im Prinzip möglich, wenn eine Sprache eine
endliche Aufzählung von Wortfolgen wäre. Nimmt man an,
dass es eine maximale Satzlänge und eine maximale Wortlänge und somit eine endliche
Anzahl von Wörtern gibt, so ist die Anzahl der bildbaren Sätze endlich.
Allerdings ist die Zahl der bildbaren Sätze selbst bei Begrenzung der Satzlänge riesig.
Und die Frage, die man dann beantworten muss, ist: Was ist die Maximallänge für
Sätze? Zum Beispiel kann man die Sätze in (\mex{1}) verlängern:
\eal
\ex Dieser Satz geht weiter und weiter und weiter und weiter \ldots
\ex {}[Ein Satz ist ein Satz] ist ein Satz.
\ex\label{einbettung-dass-Saetze} dass Max glaubt, dass Julius weiß, dass Otto behauptet, dass Karl vermutet, dass Richard bestätigt,
dass Friederike lacht
\zl
In (\mex{0}b) wird etwas über die Wortgruppe \emph{ein Satz ist ein Satz} ausgesagt, nämlich dass sie
ein Satz ist. Genau dasselbe kann man natürlich auch vom gesamten Satz (\mex{0}b) behaupten und den Satz
entsprechend um \emph{ist ein Satz} erweitern. Der Satz in (\mex{0}c) wurde gebildet, indem
\emph{dass Friederike lacht} mit \emph{dass}, \emph{Richard} und \emph{bestätigt} kombiniert
wurde. Das Ergebnis dieser Kombination ist ein neuer Satz \emph{dass Richard bestätigt,
dass Friederike lacht}. Dieser wurde analog mit \emph{dass}, \emph{Karl} und \emph{vermutet}
verlängert. Auf diese Weise erhält man einen sehr komplexen Satz, der einen weniger komplexen
Teilsatz einbettet. Dieser Teilsatz enthält wieder einen Teilsatz usw. (\mex{0}c) gleicht einer
Matrjoschka\is{Matrjoschka}: Eine Matrjoschka enthält jeweils kleinere Matrjoschkas, die sich in der Bemalung von
der sie umgebenden Matrjoschka unterscheiden können. Genauso enthält der Satz in (\mex{0}c) Teile,
die ihm ähneln, aber kürzer sind und sich in Verben und Nomen unterscheiden. Man kann das durch
Verwendung von Klammern wie folgt deutlich machen:
\ea
dass Max glaubt, [dass Julius weiß, [dass Otto behauptet, [dass Karl vermutet, [dass Richard bestätigt,
[dass Friederike lacht]]]]]
\z

\noindent
Durch Erweiterungen wie die in (\mex{-1}) können wir enorm lange und komplexe Sätze bilden.\footnote{
 Manchmal wird behauptet, dass wir in der Lage wären, unendlich lange Sätze zu bilden (\citealp*[\page
 117]{NKN2001a}; \citealp[\page 3]{KS2008a-u}). Das ist nicht richtig, da jeder Satz irgendwann einmal enden
 muss. Auch in der Theorie der formalen Sprachen in der Chomskyschen Tradition gibt es keinen
 unendlich langen Satz, vielmehr wird von bestimmten formalen Grammatiken eine Menge mit unendlich
 vielen endlichen Sätzen beschrieben (\citealp[\page 13]{Chomsky57a}). Zu Rekursion\is{Rekursion} in
 der Grammatik und Behauptungen zur Unendlichkeit unserer Sprache siehe auch \citew{PS2010a} und
 Abschnitt~\ref{Abschnitt-Rekursion}.
}
Die Festsetzung einer Grenze, bis zu der solche Kombinationen zu unserer Sprache gehören, wäre
willkürlich (\citealp[\page 208]{Harris57a}; \citealp[\page 23]{Chomsky57a}). Auch ist die Annahme, dass
solche komplexen Sätze als Gesamtheit in unserem Gehirn gespeichert sind, unplausibel. Man kann für
hochfrequente Muster bzw.\ idiomatische Kombinationen mit psycholinguistischen Experimenten zeigen,
dass sie als ganze Einheit gespeichert sind, das ist für Sätze wie die in (\mex{-1}) jedoch nicht der
Fall. Auch sind wir in der Lage, Äußerungen zu produzieren, die wir vorher noch nie gehört haben
und die auch nie vorher gesprochen oder geschrieben wurden. Es muss also eine Strukturierung der
Äußerungen, es muss bestimmte wiederkehrende Muster geben. Wir als Menschen können solche
komplexen Strukturen aus einfacheren aufbauen und umgekehrt auch komplexe Äußerungen in ihre Teile
zerlegen. Dass wir Menschen von Regeln zur Kombination von Wörtern zu größeren Einheiten Gebrauch
machen, konnte inzwischen auch durch die Hirnforschung nachgewiesen werden \citep[\page 170]{Pulvermueller2010a}.

Dass wir sprachliches Material nach Regeln kombinieren, wird besonders augenfällig, wenn die Regeln
verletzt werden. Kinder erwerben\is{Spracherwerb} sprachliche Regeln durch Generalisierungen aus dem
Input, den sie zur Verfügung haben, und produzieren dabei mitunter Äußerungen, die sie nie gehört
haben können:\is{Verb!Partikel-} 
\ea
Ich festhalte die. (Friederike, 2;6)
\z
Friederike ist dabei, die Regel für die Verbstellung zu erwerben, hat aber das gesamte Verb
an der zweiten Stelle platziert, anstatt die Verbpartikel \emph{fest} am Satzende zu belassen.

Wenn man nicht annehmen will, dass Sprache nur eine Liste von Form"=Bedeutungs"=Paaren
ist, dann muss es ein Verfahren geben, die Bedeutung komplexer Äußerungen aus
den Bedeutungen der Bestandteile der Äußerungen zu ermitteln.
Die Syntax sagt etwas über die Art und Weise der Kombination der beteiligten
Wörter aus, etwas über die Struktur einer Äußerung.
So hilft uns zum Beispiel das Wissen über Subjekt"=Verb"=Kongruenz\is{Kongruenz} bei der Interpretation
der Sätze in (\mex{1}c,d):
\eal
\label{Beispiel-mit-Kongruenz}
\ex Die Frau schläft.
\ex Die Mädchen schlafen.
\ex Die Frau kennt  die Mädchen.
\ex Die Frau kennen die Mädchen.
\zl
Die Sätze in (\mex{0}a,b) zeigen, dass ein Subjekt im Singular bzw.\ Plural
ein entsprechend flektiertes Verb braucht. In (\mex{0}a,b) verlangt das Verb nur ein
Argument, so dass die Funktion von \emph{die Frau} bzw.\ \emph{die Mädchen} klar ist.
In (\mex{0}c,d) verlangt das Verb zwei Argumente, und \emph{die Frau} und \emph{die Mädchen}
könnten an beiden Argumentstellen auf"|treten. Die Sätze könnten also bedeuten, dass
die Frau jemanden kennt oder dass jemand die Frau kennt. Durch die Flexion des Verbs und
Kenntnis der syntaktischen Gesetzmäßigkeiten des Deutschen weiß der Hörer
aber, dass es für (\mex{0}c,d) jeweils nur eine Lesart gibt.

Regeln, Muster und Strukturen in unserer Sprache aufzudecken, zu beschreiben und zu erklären, ist
die Aufgabe der Syntax. 


\section{Warum formal?}
\label{sec-formal}

Die\is{Formalisierung|(} folgenden beiden Zitate geben eine Begründung für die Notwendigkeit
formaler Beschreibung von Sprache:  
\begin{quote}
\label{quote-Chomsky-Formalisierung}%
Precisely constructed models for linguistic structure can play an
important role, both negative and positive, in the process of discovery 
itself. By pushing a precise but inadequate formulation to
an unacceptable conclusion, we can often expose the exact source
of this inadequacy and, consequently, gain a deeper understanding
of the linguistic data. More positively, a formalized theory may 
automatically provide solutions for many problems other than those
for which it was explicitly designed. Obscure and intuition-bound
notions can neither lead to absurd conclusions nor provide new and
correct ones, and hence they fail to be useful in two important respects. 
I think that some of those linguists who have questioned
the value of precise and technical development of linguistic theory
have failed to recognize the productive potential in the method
of rigorously stating a proposed theory and applying it strictly to
linguistic material with no attempt to avoid unacceptable conclusions 
by ad hoc adjustments or loose formulation.
\citep[\page5]{Chomsky57a}
\end{quote}

\begin{quote}
As is frequently pointed out but cannot be overemphasized, an important goal
of formalization in linguistics is to enable subsequent researchers to see the defects
of an analysis as clearly as its merits; only then can progress be made efficiently.
\citep[\page322]{Dowty79a}
\end{quote}
%
Wenn wir linguistische Beschreibungen formalisieren, können wir leichter erkennen, was genau eine
Analyse bedeutet. Wir können feststellen, welche Vorhersagen sie macht, und wir können alternative
Analysen ausschließen. Ein weiterer Vorteil präzise formulierter Theorien ist, dass sie sich so
aufschreiben lassen, dass sie von Computerprogrammen verarbeitet werden können. Bei einer Umsetzung
von theoretischen Arbeiten in computerverarbeitbare Grammatikfragmente fallen Inkonsistenzen sofort
auf. Die implementierten Grammatiken kann man dann dazu verwenden, große Textsammlungen, sogenannte
Korpora\is{Korpus}, zu verarbeiten, und kann dabei feststellen, welche Sätze eine Grammatik nicht
analysieren kann bzw.\ welchen Sätzen eine falsche Struktur zugeordnet wird. Zum Nutzen von
Computerimplementationen für die Linguistik siehe \citew*[\page 163]{Bierwisch63},
\citew[Kapitel~22]{Mueller99a} und \citew{Bender2008c} und Abschnitt~\ref{Asbchnitt-GB-Formalisierung}.
\is{Formalisierung|)}






\section{Konstituenten}
\label{konstituententests}

Betrachtet man den Satz in (\ref{Beispiel-Alle-Studenten-lesen}), so hat man das Gefühl, dass bestimmte Wörter zu einer
Einheit gehören.
\ea
\label{Beispiel-Alle-Studenten-lesen}
Alle Studenten lesen während dieser Zeit Bücher.
\z
So gehören die Wörter \emph{alle} und \emph{Studenten} zu einer Einheit, die etwas darüber
aussagt, wer liest. \emph{während}, \emph{dieser} und {\emph{Zeit} bilden eine Einheit, die sich
auf einen Zeitraum bezieht, in dem das Lesen stattfindet, und \emph{Bücher} sagt etwas darüber aus,
was gelesen wird. Die erste Einheit besteht wieder selbst aus zwei Teilen, nämlich aus \emph{alle}
und \emph{Studenten}, die Einheit \emph{während dieser Zeit} kann man auch in zwei Teile
unterteilen: \emph{während} und \emph{dieser Zeit}. \emph{dieser Zeit} besteht wie \emph{alle
  Studenten} aus zwei Teilen. 

Im Zusammenhang mit (\ref{einbettung-dass-Saetze}) haben wir von Matrjoschkas\is{Matrjoschka}
gesprochen. Auch bei der Zerlegung von (\mex{0}) sind zerlegbare Einheiten Bestandteile von größeren
Einheiten. Im Unterschied zu Matrjoschkas gibt es jedoch nicht nur jeweils eine kleinere Einheit,
die in einer anderen enthalten ist, sondern es gibt mitunter mehrere Einheiten, die in einer
zusammengefasst sind. Man kann sich das ganze am besten als ein System aus Schachteln vorstellen:
Eine große Schachtel enthält den gesamten Satz. In dieser Schachtel gibt es vier Schachteln, die
jeweils \emph{alle Studenten}, \emph{lesen}, \emph{während dieser Zeit} bzw.\ \emph{Bücher}
enthalten. Abbildung~\vref{Abbildung-Schachteln} zeigt das im Überblick.

\begin{figure}[htbp]
\centerline{%
\begin{pspicture}(0,0)(11,1.8)
     \rput[bl](0,0){%
\psset{fillstyle=solid, framearc=0.25,framesep=5pt}
\psframebox{%
\psframebox{%
       \psframebox{alle}
       \psframebox{Studenten}}
\psframebox{lesen}
\psframebox{%
       \psframebox{während}
       \psframebox{%
           \psframebox{dieser}
           \psframebox{Zeit}}}
\psframebox{Bücher}}}
%\psgrid
    \end{pspicture}}
\caption{\label{Abbildung-Schachteln}Wörter und Wortgruppen in Schachteln}
\end{figure}

\noindent
In diesem Abschnitt sollen Tests vorgestellt werden, die Indizien für eine engere
Zusammengehörigkeit von Wörtern darstellen. Wenn von einer \emph{Wortfolge}\is{Wortfolge}
die Rede ist, ist eine beliebige linear zusammenhängende Folge von Wörtern gemeint, 
die nicht unbedingt syntaktisch oder semantisch zusammengehörig sein müssen, \zb
\emph{Studenten lesen während} in~(\mex{0}). Mehrere Wörter, die eine strukturelle Einheit bilden,
werden dagegen als \emph{Wortgruppe}\is{Wortgruppe}, \emph{Konstituente}\is{Konstituente}
oder \emph{Phrase}\is{Phrase} bezeichnet. Den Trivialfall stellen immer einzelne Wörter dar, die
natürlich immer eine strukturelle Einheit aus einem einzelnen Element bilden.

In der traditionellen Grammatik spricht man auch von \emph{Satzgliedern}\is{Satzglied} bzw.\
\emph{Gliedteilen}\is{Gliedteil}. Die Satzglieder sind dabei die unmittelbaren Einheiten, aus denen
der Satz besteht, im Beispiel also \emph{alle Studenten}, \emph{während dieser Zeit} und
\emph{Bücher}. Die Teile, aus denen die Satzglieder bestehen, werden \emph{Gliedteil} genannt.

\citet{Bussmann2002a} zählt auch finite Verben zu den Satzgliedern, \dash, \emph{lesen} gilt als
Satzglied. In der Duden"=Grammatik \citeyearpar[\page 783]{Duden2005} ist Satzglied
allerdings anders definiert: Ein Satzglied ist hier eine Einheit des Satzes, die allein die Position vor
dem finiten Verb besetzen kann. Nach dieser Definition kann das finite Verb kein Satzglied sein. Wie
ich im Abschnitt~\ref{sec-konst-test-probleme-voranstellung} zeigen werde, führt dieser Satzgliedbegriff zu erheblichen Problemen. Ich
verwende daher die Begriffe \emph{Satzglied} und \emph{Gliedteil} in diesem Buch nicht, sondern
benutze den allgemeineren Begriff der Konstituente. 

Nach diesen Vorbemerkungen sollen nun Tests besprochen werden, die uns helfen, festzustellen, ob
eine Wortfolge eine Konstituente ist oder nicht.

\subsection{Konstituententests}

Für den Konstituentenstatus einer Wortfolge gibt es Tests, die in den folgenden Abschnitten vorgestellt werden.
Wie im Abschnitt~\ref{sec-status-der-ktests} gezeigt werden wird, gibt es Fälle, bei denen die
blinde Anwendung der Tests zu unerwünschten Resultaten führt.

\subsubsection{Ersetzungstest}

Kann man eine Wortfolge %einer bestimmten Kategorie 
in einem Satz gegen eine andere Wortfolge so
austauschen,\is{Ersetzungstest}\is{Austauschtest}\is{Substitutionstest} dass 
wieder ein akzeptabler Satz entsteht, so ist das ein Indiz dafür, dass 
die beiden Wortfolgen Konstituenten bilden.

In (\mex{1}) kann man \emph{den Mann} durch \emph{eine Frau} ersetzen, was ein Indiz dafür
ist, dass beide Wortfolgen Konstituenten sind.
\eal
\ex Er kennt [den Mann].
\ex Er kennt [eine Frau].
\zl

\noindent
Genauso kann man in (\mex{1}a) die Wortfolge \emph{das Buch zu lesen} durch
\emph{der Frau das Buch zu geben} ersetzen.
\eal
\ex Er versucht, [das Buch zu lesen].\label{ex-das-buch-zu-lesen}
\ex Er versucht, [der Frau das Buch zu geben].
\zl
%
Der Test wird \emph{Ersetzungstest} oder auch \emph{Substitutionstest} genannt.

\subsubsection{Pronominalisierungstest}

Alles,\is{Pronominalisierungstest}
worauf man sich mit einem Pronomen beziehen kann, ist eine Konstituente. 
In (\mex{1}) kann man sich \zb mit \emph{er} auf die Wortfolge \emph{der Mann} beziehen:
\eal
\ex {}[Der Mann] schläft.
\ex Er schläft.
\zl

\noindent
Auch auf Konstituenten wie \emph{das Buch zu lesen} in \pref{ex-das-buch-zu-lesen}
kann man sich mit einem Pronomen beziehen, wie (\mex{1}) zeigt:
\eal
\ex Peter versucht, [das Buch zu lesen].
\ex Klaus versucht das auch.
\zl

\noindent
Der Pronominalisierungstest ist ein Spezialfall des Ersetzungstests.

\subsubsection{Fragetest}

Was\is{Fragetest} sich erfragen lässt, ist eine Konstituente.
        \eal
        \ex {}[Der Mann] arbeitet.
        \ex Wer arbeitet?
        \zl
Der Fragetest ist ein Spezialfall des Pronominalisierungstests: Man bezieht sich mit einer
bestimmten Art von Pronomen, nämlich mit einem Fragepronomen, auf eine Wortfolge.

Auch Konstituenten wie \emph{das Buch zu lesen} in \pref{ex-das-buch-zu-lesen} kann man erfragen,
wie (\mex{1}) zeigt:
\ea
Was versucht er?
\z

\subsubsection{Verschiebetest}

Wenn Wortfolgen\is{Permutationstest|(}\is{Verschiebetest|(}\is{Umstelltest|(} ohne Beeinträchtigung der Akzeptabilität des Satzes verschoben
bzw.\ umgestellt werden können, ist das ein Indiz dafür, dass sie eine Konstituente bilden.

In (\mex{1}) sind \emph{keiner} und \emph{diese Frau} auf verschiedene Weisen angeordnet,
was dafür spricht, \emph{diese} und \emph{Frau} als zusammengehörig zu betrachten.
\eal
\ex[]{
  dass keiner [diese Frau] kennt
  }
\ex[]{
  dass [diese Frau] keiner kennt
  }
\zl
Es ist jedoch nicht sinnvoll, \emph{keiner diese} als Konstituente von (\mex{0}a) zu analysieren,
da die Sätze in (\mex{1}) und auch andere vorstellbare Abfolgen, die durch
Umstellung von \emph{keiner diese} gebildet werden können, unakzeptabel sind:\footnote{
  Ich verwende folgende Markierungen für Sätze: `*'\is{*} wenn ein Satz ungrammatisch ist,
  `\#'\is{\#} wenn der Satz eine Lesart hat, die nicht der relevanten Lesart entspricht und
  `\S'\is{\S} wenn der Satz aus semantischen oder informationsstrukturellen Gründen abweichend ist,
  \zb weil das Subjekt belebt sein müsste, aber im Satz unbelebt ist, oder weil es einen
  Konflikt gibt zwischen der Anordnung der Wörter im Satz und der Markierung bekannter
  Information durch die Verwendung von Pronomina.%
}
\eal
\ex[*]{
dass Frau keiner diese kennt
}
\ex[*]{
dass Frau kennt keiner diese
}
\zl

\noindent
Auch Konstituenten wie \emph{das Buch zu lesen} in \pref{ex-das-buch-zu-lesen}
sind umstellbar:
\eal
\ex Er hat noch nicht [das Buch zu lesen] versucht.
\ex Er hat [das Buch zu lesen] noch nicht versucht.
\ex Er hat noch nicht versucht, [das Buch zu lesen].
\zl
\is{Permutationstest|)}\is{Verschiebetest|)}\is{Umstelltest|)}

\subsubsection{Voranstellungstest}

Eine\is{Voranstellungstest|(} besondere Form der Umstellung bildet die Voranstellung. Normalerweise steht
in Aussagesätzen genau eine Konstituente vor dem finiten Verb:
\eal
\label{bsp-v2}
\ex[]{
[Alle Studenten] lesen während der vorlesungsfreien Zeit Bücher.
}
\ex[]{
[Bücher] lesen alle Studenten während der vorlesungsfreien Zeit.
}
\ex[*]{
[Alle Studenten] [Bücher] lesen während der vorlesungsfreien Zeit.
}
\ex[*]{
[Bücher] [alle Studenten] lesen während der vorlesungsfreien Zeit.
}
\zl
Die Voranstellbarkeit einer Wortfolge ist als starkes Indiz für deren Konstituentenstatus
zu werten.\is{Voranstellungstest|)}

\subsubsection{Koordinationstest}

Lassen\is{Koordination!-stest} sich Wortfolgen koordinieren, so ist das ein Indiz dafür, dass die
koordinierten Wortfolgen jeweils Konstituenten sind.

In (\mex{1}) werden \emph{der Mann} und \emph{die Frau} koordinativ verknüpft.
Die gesamte Koordination ist dann das Subjekt von \emph{arbeiten}.
Das ist ein Indiz dafür, dass \emph{der Mann}
und \emph{die Frau} Konstituenten bilden.
\ea
{}[Der Mann] und [die Frau] arbeiten.
\z
Das Beispiel in (\mex{1}) zeigt, dass sich auch Wortgruppen mit \zui koordinieren lassen:
\ea
Er hat versucht, [das Buch zu lesen] und [es dann unauffällig verschwinden zu lassen].
\z

\subsection{Bemerkungen zum Status der Tests}
\label{sec-status-der-ktests}

Es wäre schön, wenn die vorgestellten Tests immer eindeutige Ergebnisse liefern würden,
weil dadurch die empirischen Grundlagen, auf denen Theorien aufgebaut werden, klarer
wären. Leider ist dem aber nicht so. Vielmehr gibt es bei jedem der Tests Probleme,
auf die ich im Folgenden eingehen will.\LATER{AL: \citew{GHS87a-u-gekauft,Welke2007a-u}}

\subsubsection{Expletiva}
\is{Pronomen!Expletiv-|(}

Es gibt eine besondere Klasse von Pronomina, die sogenannten Expletiva, die sich nicht
auf Dinge oder Personen beziehen, also nicht referieren\is{Referenz}. Ein Beispiel ist das \emph{es} in
(\mex{1}).
\eal
\ex[]{
Es regnet.
}
\ex[]{
Regnet es?
}
\ex[]{\label{bsp-dass-es-jetzt-regnet}
dass es jetzt regnet
}
\zl
Wie die Beispiele in (\mex{0}) zeigen, kann das \emph{es} am Satzanfang
oder nach dem Verb stehen. Es kann auch durch ein Adverb vom Verb getrennt sein.
Dies spricht dafür, \emph{es} als eigenständige Einheit zu betrachten.

Allerdings gibt es Probleme mit den Tests: Zum einen ist \emph{es} nicht
uneingeschränkt umstellbar, wie (\mex{1}a) und (\mex{2}b) zeigen.
\eal
\ex[*]{\label{bsp-dass-jetzt-es-regnet}
dass jetzt es regnet
}
\ex[]{
dass jetzt keiner klatscht
}
\zl
\eal
\ex[]{\label{bsp-er-sah-es-regnen}
Er sah es regnen.
}
\ex[*]{\label{bsp-es-sah-er-regnen}
Es sah er regnen.
}
\ex[]{
Er sah einen Mann klatschen.
}
\ex[]{
Einen Mann sah er klatschen.
}
\zl
Im Gegensatz zum Akkusativobjekt \emph{einen Mann} in (\mex{0}c,d) kann das Expletivum in (\mex{0}b) nicht
vorangestellt werden.

Zum anderen schlagen auch Substitutions- und Fragetest fehl:
\eal
\ex[*]{
Der Mann/er regnet.
}
\ex[*]{
Wer/was regnet?
}
\zl

\noindent
Genauso liefert der Koordinationstest ein negatives Ergebnis:
\ea[*]{
Es und der Mann regnet/regnen.
}
\z
Dieses Fehlschlagen der Tests lässt sich leicht erklären: Schwach betonte
Pronomina wie \emph{es} stehen bevorzugt vor anderen Argumenten, direkt nach
der Konjunktion (\emph{dass} in (\ref{bsp-dass-es-jetzt-regnet})) bzw.\
direkt nach dem finiten Verb (\ref{bsp-er-sah-es-regnen}) (siehe \citew[\page 570]{Abraham95a-u}). Wird, wie
in (\ref{bsp-dass-jetzt-es-regnet}), ein Element vor das Expletivum gestellt,
wird der Satz ungrammatisch. Der Grund für die Ungrammatikalität von
(\ref{bsp-es-sah-er-regnen}) liegt in einer generellen Abneigung des 
Akkusativ"=\emph{es} dagegen, die erste Stelle im Satz einzunehmen. Es gibt zwar Belege für
solche Muster, aber in diesen ist das \emph{es} immer referentiell (\citealt[\page162]{Lenerz94a};
\citealp[\page4]{GS97a}).

%        \citet[\page162]{Lenerz94a} schreibt (\ref{bsp-geld-weg}) Peter Gallmann\aimention{Peter Gallmann}
%        zu. (\ref{bsp-experten}) stammt von Santorini und ist nach \citew[\page4]{GS97a} zitiert.%
%
Dass auch der Substitutionstest und der Fragetest fehlschlagen, ist ebenfalls
nicht weiter verwunderlich, denn das \emph{es} ist nicht referentiell.
Man kann es höchstens durch ein anderes Expletivum wie \emph{das} ersetzen.
Wenn wir das Expletivum durch etwas Referentielles ersetzen, bekommen wir semantische Abweichungen.
Natürlich ist es auch nicht sinnvoll, nach etwas semantisch Leerem zu fragen oder
sich darauf mit einem Pronomen zu beziehen.
\is{Pronomen!Expletiv-|)}


Daraus folgt: Nicht alle Tests müssen positiv ausfallen, damit eine Wortfolge als Konstituente gelten kann,
\dash, die Tests stellen keine notwendige Bedingung dar.

%% \subsubsection{Pronominalisierung}

%% Der Pronominalisierungstest sagt, dass alles, worauf man sich mit einem Pronomen beziehen kann,
%% eine Konstituente ist. Betrachtet man den folgenden kleinen Text, so sieht man, dass
%% der Pronominalisierungstest nicht unproblematisch ist.
%% \ea
%% Über Mozart hat er schon viele Bücher gelesen. Sie waren alle interessant.
%% \z
%% Das Personalpronomen \emph{sie} bezieht sich auf \emph{viele Bücher über Mozart},
%% in (\mex{0}) stehen die Bestandteile von \emph{viele Bücher über Mozart} aber nicht zusammen.
%% Man kann über (\mex{1}) sagen, dass \emph{viele Bücher über Mozart} eine Konstituente
%% bilden, aber in (\mex{0}) ist ein Bestandteil dieser Wortgruppe vorangestellt worden.
%% \ea
%% Hat er schon viele Bücher über Mozart gelesen?
%% \z
%%
%% diskontinuierliche Konstituente


\subsubsection{Der Verschiebetest}

Der Verschiebetest\is{Verschiebetest} ist in Sprachen mit relativ freier Konstituentenstellung problematisch, da sich
nicht immer ohne weiteres sagen lässt, was verschoben wurde. Zum Beispiel stehen die Wörter
\emph{gestern dem Mann} in (\mex{1}) an jeweils unterschiedlichen Positionen:
\eal
\ex weil keiner gestern dem Mann geholfen hat
\ex weil gestern dem Mann keiner geholfen hat
\zl
Man könnte also annehmen, dass \emph{gestern} gemeinsam mit \emph{dem Mann} umgestellt wurde. Eine
alternative Erklärung für die Abfolgevarianten in (\mex{0}) liegt aber darin anzunehmen, dass
Adverbien an beliebiger Stelle im Satz stehen können und dass in (\mex{0}b) nur \emph{dem Mann} vor
\emph{keiner} gestellt wurde. Man sieht auf jeden Fall, dass \emph{gestern} und \emph{dem Mann} nicht
in einer semantischen Beziehung stehen und dass man sich auch nicht mit einem Pronomen auf die
gesamte Wortfolge beziehen kann. Obwohl es so aussieht, als sei das Material zusammen umgestellt
worden, ist es also nicht sinnvoll anzunehmen, dass es sich bei \emph{gestern dem Mann} um eine
Konstituente handelt.

% Engel94a:148  Ich mag es. -> keine Umstellung möglich

% zu kompliziert
%% \eal
%% \ex Der Hund bellt oft.
%% \ex (dass) oft der Hund bellt
%% \zl



\subsubsection{Der Voranstellungstest}
\label{sec-konst-test-probleme-voranstellung} 

Wie\is{Vorfeld!-besetzung|(}\is{Voranstellungstest|(}
 bei der Diskussion von (\ref{bsp-v2}) erwähnt, steht im Deutschen normalerweise
eine Konstituente vor dem Finitum. Voranstellbarkeit vor das finite Verb wird mitunter
sogar als ausschlaggebendes Kriterium für Konstituentenstatus genannt bzw.\ in der Definition des
Begriffs Satzglied verwendet \citep[\page 783]{Duden2005}. Als Beispiel sei
hier die Definition aus \citew{Bussmann83a} aufgeführt, die in \citew{Bussmann90a}
nicht mehr enthalten ist:
\begin{quote}
{\bf Satzgliedtest}\is{Satzglied} [Auch: Konstituententest]. Auf der $\to$ Topikalisierung
beruhendes Verfahren zur Analyse komplexer Konstituenten. Da bei Topikalisierung
jeweils nur eine Konstituente bzw.\ ein $\to$ Satzglied an den Anfang gerückt werden kann,
lassen sich komplexe Abfolgen von Konstituenten (\zb Adverbialphrasen) als
ein oder mehrere Satzglieder ausweisen; in \textit{Ein Taxi quält sich im Schrittempo
durch den Verkehr} sind \textit{im Schrittempo} und \textit{durch den Verkehr}
zwei Satzglieder, da sie beide unabhängig voneinander in Anfangsposition gerückt werden
können. \citep[\page446]{Bussmann83a}
\end{quote}

\noindent
Aus dem Zitat ergeben sich die beiden folgenden Implikationen:
\begin{itemize}
\item Teile des Materials können einzeln vorangestellt werden. $\to$\\
      Das Material bildet keine Konstituente.
\item Material kann zusammen vorangestellt werden. $\to$\\
      Das Material bildet eine Konstituente.
\end{itemize}
Wie ich zeigen werde, sind beide problematisch.

Die erste ist wegen Beispielen wie (\mex{1}) problematisch:
\eal
\ex Keine Einigung erreichten Schröder und Chirac über den Abbau der Agrarsubventionen.\footnote{tagesschau, 15.10.2002, 20:00.}
\ex Über den Abbau der Agrarsubventionen erreichten Schröder und Chirac keine Einigung.
\zl
Obwohl Teile der Nominalphrase \emph{keine Einigung über den Abbau der Agrarsubventionen}
einzeln vorangestellt werden können, wollen wir die Wortfolge als eine Nominalphrase (NP) analysieren,
wenn sie wie in (\mex{1}) nicht vorangestellt ist.
\ea
Schröder und Chirac erreichten [keine Einigung über den Abbau der Agrarsubventionen].
\z
Die Präpositionalphrase \emph{über den Abbau der Agrarsubventionen} hängt semantisch von \emph{Einigung}
ab (\emph{sie einigen sich über die Agrarsubventionen}).

Diese Wortgruppe kann auch gemeinsam vorangestellt werden:
\ea
Keine Einigung über den Abbau der Agrarsubventionen erreichten Schröder und Chirac.
\z
In theoretischen Erklärungsversuchen geht man davon aus, dass \emph{keine Einigung über den Abbau
  der Agrarsubventionen} eine Konstituente ist, die unter gewissen Umständen
aufgespalten\is{NP"=Aufspaltung} werden kann. In solchen Fällen können die einzelnen
% Fanselow88 sagt, dass es zwei Phrasen sind, eine mit pro
Teilkonstituenten wie in (\mex{-2}) unabhängig voneinander umgestellt werden \citep{deKuthy2002a}. 



Die zweite Implikation ist ebenfalls problematisch, da es Sätze
wie die in (\mex{1}) gibt:
\eal
\label{bsp-mehr-vf}
\ex\label{bsp-trocken-durch-die-stadt}
{}[Trocken] [durch die Stadt] kommt man am Wochenende auch mit der BVG.\footnote{
        taz berlin, 10.07.1998, S.\,22.
      }
\ex {}[Wenig] [mit Sprachgeschichte] hat der dritte Beitrag in dieser Rubrik zu tun, [\ldots]\footnote{
  Zeitschrift für Dialektologie und Linguistik, LXIX, 3/2002, S.\,339.
}
\zl
In (\mex{0}) befinden sich mehrere Konstituenten vor dem finiten Verb, die nicht in einer syntaktischen
oder semantischen
%\NOTE{JB: intuitiv doch} 
Beziehung zueinander stehen. Was es genau heißt,
in einer syntaktischen bzw.\ semantischen Beziehung zueinander zu stehen, wird in den folgenden
Kapiteln noch genauer erklärt. Beispielhaft sei hier nur
für (\mex{0}a) gesagt, dass \emph{trocken} ein Adjektiv ist, das in (\mex{0}a) \emph{man}
als Subjekt hat und außerdem etwas über den Vorgang des Durch"=die"=Stadt"=Kommens aussagt,
sich also auf das Verb bezieht. Wie (\mex{1}b) zeigt, kann \emph{durch die Stadt} nicht
mit dem Adjektiv \emph{trocken} kombiniert werden:
\eal
\ex[]{
Man ist/bleibt trocken.
}
\ex[*]{
Man ist/bleibt trocken durch die Stadt.
}
\zl
Genauso ist \emph{durch die Stadt} eine Richtungsangabe, die syntaktisch vollständig ist
und nicht mit einem Adjektiv kombiniert werden kann:
\eal
\ex[]{
der Weg durch die Stadt
}
\ex[*]{
der Weg trocken durch die Stadt
}
\zl
Das Adjektiv \emph{trocken} hat also weder syntaktisch noch semantisch etwas mit 
der Präpositionalphrase \emph{durch die Stadt} zu tun. Beide Phrasen haben jedoch gemeinsam,
dass sie sich auf das Verb beziehen bzw.\ von diesem abhängen.

Man mag dazu neigen, die Beispiele in (\ref{bsp-mehr-vf}) als Ausnahmen
abzutun. Das ist jedoch nicht gerechtfertigt, wie ich in einer breit angelegten empirischen
Studie gezeigt habe \citep{Mueller2003b}.

Würde man \emph{trocken durch die Stadt} aufgrund des Testergebnisses als Konstituente bezeichnen
und annehmen, dass \emph{trocken durch die Stadt} wegen der Existenz von (\ref{bsp-trocken-durch-die-stadt})
auch in (\mex{1}) als Konstituente zu behandeln ist, 
wäre der Konstituentenbegriff entwertet, da man mit den Konstituententests ja gerade semantisch
bzw.\ syntaktisch zusammengehörige Wortfolgen ermitteln will.%\NOTE{Johannes Bubenzer: Unverständlich! Wieso?}
\footnote{
  Die Daten kann man mit einem leeren verbalen Kopf\is{Spur!Verb-}\is{leeres Element} vor dem finiten Verb analysieren,
  so dass letztendlich wieder genau eine Konstituente vor dem Finitum steht \citep{Mueller2005d}.
  Trotzdem sind die Daten für Konstituententests problematisch, da die Konstituententests
  ja entwickelt wurden, um zu bestimmen, ob \zb \emph{trocken} und \emph{durch die Stadt}
  bzw.\ \emph{wenig} und \emph{mit Sprachgeschichte} in (\mex{1}) Konstituenten bilden.%
}

\eal
\ex Man kommt am Wochenende auch mit der BVG trocken durch die Stadt.
\ex Der dritte Beitrag in dieser Rubrik hat wenig mit Sprachgeschichte zu tun.
\zl
Voranstellbarkeit ist also nicht hinreichend für Konstituentenstatus.

Wir haben auch gesehen, dass es sinnvoll ist, Expletiva als Konstituenten zu behandeln,
obwohl diese im Akkusativ nicht voranstellbar sind (siehe auch (\ref{bsp-er-sah-es-regnen})):
\eal
\ex[]{
Er bringt es bis zum Professor.
}
\ex[\#]{
Es bringt er bis zum Professor.
} 
\zl
Es gibt weitere Elemente, die ebenfalls nicht vorangestellt werden können. Als Beispiel
seien noch die mit inhärent reflexiven Verben\is{Verb!inhärent reflexives} verwendeten Reflexivpronomina\is{Pronomen!Reflexiv-} genannt:
\eal
\ex[]{
Karl hat sich nicht erholt.
}
\ex[*]{
Sich hat Karl nicht erholt.
}
\zl
Daraus folgt, dass Voranstellbarkeit kein notwendiges Kriterium für den Konstituentenstatus
ist. Somit ist Voranstellbarkeit weder hinreichend noch notwendig.
\is{Vorfeld!-besetzung|)}\is{Voranstellungstest|)}


\subsubsection{Koordination}
\label{Abschnitt-K-Tests-Koordination}

Koordinationsstrukturen\is{Koordination|(}\is{Koordination!-stest} wie die in (\mex{1}) sind ebenfalls problematisch:
\ea
\label{ex-gapping}
%Peter gab ihm einen Apfel und ihr eine Tomate.
Deshalb kaufte der Mann einen Esel und die Frau ein Pferd.
\z
Auf den ersten Blick scheinen in (\mex{0}) \emph{der Mann einen Esel} und \emph{die Frau ein Pferd}
koordiniert worden zu sein. Bilden \emph{der Mann einen Esel} und \emph{die Frau ein Pferd} nun jeweils Konstituenten?

Wie andere Konstituententests zeigen, ist die Annahme von Konstituentenstatus für diese Wortfolgen
nicht sinnvoll. Die Wörter kann man nicht gemeinsam umstellen:\footnote{
  Der Bereich vor dem finiten Verb wird auch \emph{Vorfeld}\is{Feld!Vor-} genannt (siehe Abschnitt~\ref{Abschnitt-Toplogie}).
  Scheinbar mehrfache Vorfeldbesetzung ist im Deutschen unter bestimmten
  Bedingungen möglich. Siehe dazu auch den vorigen Abschnitt, insbesondere
  die Diskussion der Beispiele in (\ref{bsp-mehr-vf}) auf Seite~\pageref{bsp-mehr-vf}.
  Das Beispiel in (\mex{1}) ist jedoch bewusst so konstruiert worden, 
  dass sich ein Subjekt mit im Vorfeld befindet, was aus Gründen, die 
  mit den informationsstrukturellen Eigenschaften solcher Vorfeldbesetzungen
  zusammenhängen, bei Verben wie \emph{kaufen} nicht möglich ist. Siehe auch
  \citew{dKM2003a}
%,Cook-sub} 
zu Subjekten in vorangestellten Verbalphrasen.%
}
\ea[*]{
Der Mann einen Esel kaufte deshalb.
}
\z

\noindent
Eine Ersetzung durch Pronomina ist nicht ohne Ellipse möglich:
\eal
\ex[\#]{
Deshalb kaufte er.
}
\ex[*]{
Deshalb kaufte ihn.
}
\zl
Die Pronomina stehen nicht für die zwei logischen Argumente von \emph{kaufen}, die
in (\ref{ex-gapping})  \zb durch \emph{der Mann} und \emph{einen Esel} realisiert sind,
sondern nur für jeweils eins. In der Tat wurden auch Analysen für Sätze wie (\ref{ex-gapping}) vorgeschlagen,
in denen zwei Verben \emph{kauft} vorkommen, von denen jedoch nur eins sichtbar
ist \citep{Crysmann2003c}. (\ref{ex-gapping}) entspricht also (\mex{1}):
\ea
Deshalb kaufte der Mann einen Esel und kaufte die Frau ein Pferd.
\z
Das heißt, obwohl es so aussieht als seien \emph{der Mann einen Esel} und \emph{die Frau ein Pferd}
koordiniert, sind \emph{kauft der Mann einen Esel} und \emph{(kauft) die Frau ein Pferd} koordiniert.
\is{Koordination|)}

Daraus folgt: Auch wenn einige Tests erfüllt sind bzw.\ auf den ersten Blick erfüllt zu sein
scheinen, muss es noch lange nicht sinnvoll sein, eine Wortfolge als Konstituente einzustufen, \dash,
die Tests stellen keine hinreichende Bedingung dar.



Zusammenfassend kann man sagen, dass die Konstituententests, wenn man sie ohne Wenn und Aber
anwendet, nur Indizien liefern. Ist man sich der erwähnten problematischen Fälle bewusst,
kann man mit den Tests aber doch einigermaßen klare Vorstellungen davon bekommen, welche
Wörter als Einheit analysiert werden sollten.



\section{Wortarten}
\label{Abschnitt-Wortarten}

Die Wörter in (\mex{1}) unterscheiden sich in ihrer Bedeutung, aber auch in anderen Eigenschaften.
\ea
Der dicke Mann lacht jetzt.
\z
Jedes der Wörter unterliegt eigenen Gesetzmäßigkeiten beim Zusammenbau von Sätzen. Man bildet
Klassen von Wörtern, die wesentliche Eigenschaften teilen. So gehört \emph{der} zu den Artikeln\is{Artikel},
\emph{Mann} zu den Nomina\is{Nomen}, \emph{lacht} zu den Verben\is{Verb} und \emph{jetzt} zu den
Adverben\is{Adverb}. Wie (\mex{1}) zeigt, kann man die Wörter in (\mex{0}) durch Wörter der gleichen
Wortart ersetzen.
\ea
Die dünne Frau lächelt immer.
\z
Das ist nicht immer gegeben, \zb lässt sich \emph{erholt} oder \emph{lächelst} nicht in (\mex{0})
einsetzen. Die Kategorisierung von Wörtern nach der Wortart ist also eine grobe, und wir müssen
noch viel mehr über die Eigenschaften von Wörtern sagen. In diesem Abschnitt sollen die
verschiedenen Wortarten vorgestellt werden, die anderen Eigenschaften werden dann in anderen
Abschnitten dieses Einleitungskapitels besprochen.

Die wichtigsten Wortarten sind \emph{Verb}, \emph{Nomen}\is{Nomen}, \emph{Adjektiv}\is{Adjektiv}, \emph{Präposition}\is{Pr"aposition} und
\emph{Adverb}\is{Adverb}. Statt Nomen wird auch der Begriff \emph{Substantiv}\is{Substantiv}
verwendet. In den vergangen Jahrhunderten hat man auch von Dingwörtern, Tätigkeitswörtern und Eigenschaftswörtern
gesprochen, aber diese Bezeichnungen sind problematisch, wie die folgenden Beispiele verdeutlichen
sollen:
\eal
\ex die \emph{Idee}
\ex die \emph{Stunde}
\ex das laute \emph{Sprechen}
\ex Die \emph{Erörterung} der Lage dauerte mehrere Stunden.
\zl
Bei (\mex{0}a) handelt es sich nicht um ein konkretes Ding, (\mex{0}b) bezeichnet ein Zeitintervall,
in (\mex{0}c) und (\mex{0}d) geht es um Handlungen. Man sieht, dass \emph{Idee}, \emph{Stunde},
\emph{Sprechen} und \emph{Erörterung} von ihrer Bedeutung her sehr unterschiedlich sind. Diese
Wörter verhalten sich aber dennoch in vielerlei Hinsicht wie \emph{Mann} und \emph{Frau} und werden
deshalb zu den Nomina gezählt.

Auch\is{Verb|(} der Begriff Tätigkeitswort wird inzwischen in der wissenschaftlichen Grammatik nicht mehr
benutzt, da Verben nicht unbedingt Tätigkeiten beschreiben müssen:
\eal
\ex Ihm gefällt das Buch.
\ex Das Eis schmilzt.
\ex Es regnet.
\zl
Auch müsste man \emph{Erörterung} wohl als Tätigkeitswort einordnen.\is{Verb|)}

Adjektive\is{Adjektiv|(} geben nicht immer Eigenschaften von Objekten an. Im folgenden Beispiel ist sogar das
Gegenteil der Fall: Die Eigenschaft, Mörder zu sein, wird durch das Adjektiv als wahrscheinlich
oder möglich dargestellt.
\eal
\ex der mutmaßliche Mörder
\ex Soldaten sind potentielle Mörder.
\zl
Die Adjektive selbst steuern in (\mex{1}) keine Information über Eigenschaften bei. Auch möchte man \emph{lachende} in (\mex{1})
wohl als Adjektiv einordnen.
\ea
der lachende Mann
\z
Klassifiziert man jedoch nach Eigenschaft und Tätigkeit, müsste \emph{lachend} ein Tätigkeitswort sein.\is{Adjektiv|)}

Statt\is{Flexion|(} semantischer Kriterien verwendet man heutzutage für die Bestimmung der meisten
Wortarten formale Kriterien: Man betrachtet einfach die Formen, in denen ein Wort vorkommen kann. So
gibt es \zb bei \emph{lacht} die Formen in (\mex{1}). 
\eal
\ex Ich lache.
\ex Du lachst.
\ex Er lacht.
\ex Wir lachen.
\ex Ihr lacht.
\ex Sie lachen.
\zl
Zusätzlich gibt es Formen für das Präteritum, den Imperativ, Konjunktiv I, Konjunktiv II und
infinite Formen (Partizipien und Infinitiv mit und ohne \emph{zu}). All diese Formen bilden das
Flexionsparadigma\is{Flexion!-sparadigma} eines Verbs. Im Flexionsparadigma spielen
Tempus\is{Tempus}  (Präsens\is{Pr"asens}, Präteritum\is{Pr"ateritum}, Futur\is{Futur}),
Modus\is{Modus} (Indikativ\is{Indikativ}, Konjunktiv\is{Konjunktiv}, Imperativ\is{Imperativ}),
Person\is{Person} (1., 2., 3.) und Numerus (Singular\is{Singular}, Plural\is{Plural}) eine Rolle. 
Wie (\mex{0}) zeigt, können in einem Paradigma Formen zusammenfallen.

Genauso wie Verben haben Nomina\is{Nomen} ein Flexionsparadigma:
\eal
\ex der Mann
\ex des Mannes
\ex dem Mann
\ex den Mann
\ex die Männer
\ex der Männer
\ex den Männern
\ex die Männer
\zl
Nomina kann man nach Genus\is{Genus} (feminin, maskulin, neutrum) unterscheiden. Diese Unterscheidungen sind
formaler Natur und haben nur bedingt etwas mit dem Geschlecht von Personen oder der Tatsache, dass ein
Gegenstand bezeichnet wird, zu tun:
\eal
\ex die Tüte
\ex der Krampf
\ex das Kind
\zl
Die Begriffe \emph{männlich}, \emph{weiblich} und \emph{sächlich} sollte man deshalb vermeiden.
\emph{Genus} steht für \emph{Art}. In Bantu"=Sprachen gibt es 7--10 Genera \citep{Corbett2008a}.

Im nominalen Paradigma sind neben dem Genus auch Kasus\is{Kasus} (Nominativ\is{Kasus!Nominativ},
Genitiv\is{Kasus!Genitiv}, Dativ\is{Kasus!Dativ}, Akkusativ\is{Kasus!Akkusativ}) und Numerus\is{Numerus} wichtig. 

Adjektive\is{Adjektiv} flektieren wie Nomina nach Genus, Kasus, Numerus. Sie unterscheiden sich jedoch darin von
Nomina, dass das Genus bei Adjektiven nicht fest ist. Adjektive können in allen drei Genera
gebraucht werden:
\eal
\ex eine kluge Frau
\ex ein kluger Mann
\ex ein kluges Kind
\zl
Zusätzlich zu Genus, Kasus und Numerus unterscheidet man noch verschiedene
Flexionsklassen. Traditionell unterscheidet man zwischen starker, gemischter und schwacher Flexion
des Adjektivs. Welche Flexionsklasse\is{Flexionsklasse}\label{page-Flexionsklasse-Wunderlich}
gewählt werden muss, hängt von der Form bzw.\ dem Vorhandensein eines Artikels ab:\footnote{
Dieter Wunderlich\aimention{Dieter Wunderlich} hat in einem unveröffentlichten Aufsatz gezeigt, dass
man mit den Flexionsklassen stark und schwach auskommen kann. Zu den Details siehe
\citew[Abschnitt~2.2.5]{ps2} oder \citew[Abschnitt~13.2]{MuellerLehrbuch1}. 
}
\eal
\ex ein alter Wein
\ex der alte Wein
\ex alter Wein
\zl

\noindent
Außerdem sind Adjektive für gewöhnlich steigerbar:
\eal
\ex klug
\ex klüger
\ex am klügsten
\zl
Das ist nicht immer gegeben. Insbesondere bei Adjektiven, die auf einen Endpunkt Bezug nehmen, sind
Steigerungsformen nicht sinnvoll: Wenn eine Lösung optimal ist, gibt es keine bessere, also kann man
auch nicht von einer optimaleren Lösung sprechen. Genauso kann man nicht töter als tot sein. 
%
% Anke Lüdeling:16.05.2010
% hm. das steht zwar überall so, ist aber höchstens semantisch problematisch, morphologisch
% überhaupt nicht. morphologisch und semantisch problematisch hingegen sind komposita wie eiskalt,
% mausetot etc. die einen vergleich enthalten. hier würden viele argumentieren, dass die
% steigerungsformen ungrammatisch sind.  generell ist gradation wahrscheinlich eher wortbildung als
% flexion und daher eh für die wortartzuweisung problematisch

Es gibt einige Sonderfälle wie \zb die Adjektive \emph{lila} und \emph{rosa}. Diese können flektiert
werden, es gibt neben der flektierten Form in (\mex{1}a) aber auch die unflektierte:
\eal
\ex eine lilane Blume
\ex eine lila Blume
\zl
\emph{lila} wird in beiden Fällen zu den Adjektiven gezählt. Man begründet diese Einordnung damit,
dass das Wort an denselben Stellen vorkommt wie Adjektive, die eindeutig an ihrer Flexion als
Adjektive zu erkennen sind.\is{Flexion|)}


Die bisher besprochenen Wortarten konnten alle aufgrund ihrer Flexionseigenschaften unterschieden
werden. Bei den nicht flektierbaren Wörtern muss man andere Kriterien heranziehen. Hier teilt man
die Wörter -- wie auch die schon erwähnten nicht flektierten Adjektive -- nach ihrem syntaktischen
Kontext in Klassen ein. Man unterscheidet Präpositionen\is{Pr"aposition}, Adverbien\is{Adverb},
Konjunktionen\is{Konjunktion}, Interjektionen\is{Interjektion} und mitunter auch Partikeln\is{Partikel}.
Präpositionen sind Wörter, die zusammen mit einer Nominalgruppe vorkommen, deren Kasus sie
bestimmen:
\eal
\ex in diesen Raum
\ex in diesem Raum
\zl
\emph{wegen} wird auch oft der Klasse der Präpositionen zugerechnet, obwohl es auch nach dem Nomen
stehen kann und dann Postposition\is{Postposition} genannt werden müsste:
\ea
des Geldes wegen
\z
Man spricht gelegentlich auch von Adpositionen\is{Adposition}, wenn man sich nicht auf die Stellung des Wortes
festlegen will.

Adverbien\is{Adverb} verlangen im Gegensatz zu Präpositionen keine Nominalgruppe:
\eal
\ex Er schläft in diesem Raum.
\ex Er schläft dort.
\zl
Mitunter werden die Adverbien einfach den Präpositionen zugeordnet (siehe S.~\pageref{Seite-Adverbien-PP}). Die Begründung dafür ist, dass
Präpositionalgruppen wie \emph{in diesem Raum} sich genauso verhalten wie entsprechende
Adverbien. \emph{in} unterscheidet sich von \emph{dort} nur dadurch, dass es noch eine Nominalgruppe
braucht. Aber solche Unterschiede gibt es auch innerhalb der verschiedenen Klassen der flektierbaren
Wörter. So verlangt \emph{schlafen} nur eine Nominalgruppe, \emph{erkennen} dagegen zwei.
\eal
\ex Er schläft.
\ex Peter erkennt ihn.
\zl

\noindent
Die Konjunktionen\is{Konjunktion} unterteilt man in neben- bzw.\ beiordnende und unterordnende. Zu den nebenordnenden Konjunktionen
zählen \emph{und} und \emph{oder}. In koordinativen Verknüpfungen werden meist zwei Wortgruppen mit
gleichen syntaktischen Eigenschaften verbunden. Sie stehen nebeneinander. \emph{dass} und
\emph{weil} sind unterordnende Konjunktionen, da mit diesen Konjunktionen eingeleitete Sätze Teile
eines größeren Satzes sind.
\eal
\ex Klaus glaubt, dass er lügt.
\ex Klaus glaubt ihm nicht, weil er lügt.
\zl
Die unterordnenden Konjunktionen werden auch \emph{Subjunktion}\is{Subjunktion} genannt.

Interjektionen\is{Interjektion} sind satzwertige Ausdrücke, wie \emph{Ja!}, \emph{Bitte!}, \emph{Hallo!},
\emph{Hurra!}, \emph{Bravo!}, \emph{Pst!}, \emph{Plumps!}.

Wenn Adverbien und Präpositionen nicht in eine Klasse eingeordnet werden, dann werden die Adverbien
normalerweise als Restkategorie verwendet, \dash, dass alle Nichtflektierbaren, die keine
Präpositionen, Konjunktionen oder Interjektionen sind, Adverbien sind.  Mitunter wird die Restklasse
auch noch weiter unterteilt: Nur die Wörter werden Adverb genannt, die -- wenn sie als Satzglied
verwendet werden -- vor das finite Verb gestellt werden können. Die Wörter, die nicht voranstellbar
sind, werden dagegen \emph{Partikel}\is{Partikel} genannt. Die Partikeln
werden dann nach ihrer Funktion in verschiedene Klassen wie Gradpartikel\is{Partikel!Grad-} und
Abtönungspartikel\is{Partikel!Abt"onungs-} eingeteilt. Da in diese nach der Funktion bestimmten
Klassen aber auch Wörter fallen, die zu den Adjektiven zählen, mache ich diese Unterscheidung nicht
und spreche einfach von Adverbien.


Wir haben bereits einen wesentlichen Teil der flektierbaren Wörter nach Wortarten klassifiziert. Wenn
man vor der Aufgabe steht, ein bestimmtes Wort einzuordnen, kann man den Entscheidungsbaum in
Abbildung~\vref{Abbildung-Wortarten} verwenden, den ich der Duden"=Grammatik \citeyearpar[\page 133]{Duden2005}
% in der neuen Ausgabe ist die Grafik auf der Umschlagseite
entnommen habe.
\begin{figure}[htbp]
\centerline{%
\begin{tabular}{@{}ccccc@{}}
\multicolumn{5}{@{}c@{}}{\node{wortart}{Wortart}}\\[5ex]
\multicolumn{2}{@{}c@{}}{\node{flektierbar}{flektierbar}} &&& \node{nicht}{nicht}\\
                               & & & & \node{nflektierbar}{flektierbar}\\
%
\node{Tempus}{nach Tempus} & \multicolumn{2}{@{}c@{}}{\node{Kasus}{nach Kasus}}\\[3ex]
            & \node{fGenus}{festes Genus} & \multicolumn{2}{@{}c@{}}{\node{vGenus}{veränderbares Genus}}\\[3ex]
%
            &              & \node{nk}{nicht komparierbar} & \node{k}{komparierbar}\\[3ex]
%
\node{Verb}{Verb}        & \node{Nomen}{Nomen}        & \node{Artikel}{Artikelwort}        & \node{Adjektiv}{Adjektiv} & \node{nichtflektierbare}{Nichtflektierbare}\\
            &              & \node{Pronomen}{Pronomen}\\
\end{tabular}}
\nodeconnect{wortart}{flektierbar}\nodeconnect{wortart}{nicht}%
\nodeconnect{flektierbar}{Tempus}\nodeconnect{flektierbar}{Kasus}%
\nodeconnect{Kasus}{fGenus}\nodeconnect{Kasus}{vGenus}%
\nodeconnect{vGenus}{nk}\nodeconnect{vGenus}{k}%
\nodeconnect{nflektierbar}{nichtflektierbare}%
\nodeconnect{Tempus}{Verb}\nodeconnect{fGenus}{Nomen}\nodeconnect{nk}{Artikel}\nodeconnect{k}{Adjektiv}%
%\nodeconnect{nichtflektierbare}{Praeposition}%
%\nodeconnect{Pronomen}{Personalp}\nodeconnect{Adjektiv}{attributiv}%
\caption{\label{Abbildung-Wortarten}Entscheidungsbaum zur Bestimmung von Wortarten nach \citew[\page 133]{Duden2005}}
\end{figure}
Wenn ein Wort mit Tempus\is{Tempus} flektiert, ist es ein Verb\is{Verb}, wenn es verschiedene
Kasusformen\is{Kasus} hat, muss man überprüfen, ob es ein festes Genus\is{Genus} hat. Ist das der
Fall, handelt es sich um ein Nomen\is{Nomen}. Bei Wörtern mit veränderbarem Genus wird überprüft, ob sie
komparierbar\is{Komparation} sind. Wenn ja, handelt es sich um Adjektive\is{Adjektiv}. Alle anderen Wörter kommen in eine Restkategorie, die der Duden Pronomina\is{Pronomen}/Artikelwörter\is{Artikelwort}
nennt. In dieser Restkategorie werden genauso wie bei den nicht flektierbaren Wörtern in
Abhängigkeit vom syntaktischen Verhalten Klassen gebildet. Der Duden unterscheidet Pronomina und
Artikelwörter. Die Pronomina sind nach dieser Klassifikation Wörter, die für eine gesamte Nominalgruppe wie
\emph{der Mann} stehen, die Artikelwörter werden dagegen normalerweise mit einem Nomen kombiniert.
In der lateinischen Grammatik schließt der Pronomenbegriff Pronomina im obigen Sinn und
Artikelwörter ein, da die Formen mit und ohne Nomen identisch sind. Im Laufe der Jahrhunderte haben
sich aber die Formen auseinander entwickelt, so dass man in den heutigen romanischen
Sprachen unterscheiden kann zwischen solchen, die für eine ganze Nominalgruppe
stehen können, und solchen, die zusammen mit einem Nomen auf"|treten müssen. Elemente der
letztgenannten Klasse werden auch \emph{Determinator}\is{Determinator} genannt. 
%
% Auch im Deutschen gibt es bei einigen Formen einen Unterschied
% zwischen dem Demonstrativpronomen und dem definiten Artikel:
% \eal
% \ex Wir gedenken der Opfer.
% \ex Wir gedenken derer.
% \ex Wir helfen den Männern.
% \ex Wir helfen denen.
% \zl


Folgt\is{Pronomen|(} man diesem Entscheidungsbaum, landen \zb das Personalpronomen mit seinen Formen
\emph{ich}, \emph{du}, \emph{er}, \emph{sie}, \emph{es}, \emph{wir}, \emph{ihr}, \emph{sie} und das
Possessivpronomen mit den Formen \emph{mein}, \emph{dein}, \emph{sein}, \emph{unser}, \emph{euer},
\emph{ihr} und entsprechend flektierten Varianten in der Kategorie Artikelwort/""Pronomen.  Das
Reflexivpronomen mit den Formen \emph{mich}, \emph{dich}, \emph{sich}, \emph{uns}, \emph{euch} und
das Reziprokpronomen \emph{einander} müssen dagegen als Sonderfall betrachtet werden, denn es gibt
für \emph{sich} und \emph{einander} keine ausdifferenzierten Formen mit verschiedenen Genera. Kasus
äußert sich beim Reziprokpronomen nicht morphologisch, man kann nur durch das Einsetzen von
\emph{einander} in Sätze, die einen Genitiv, Dativ bzw.\ Akkusativ verlangen, feststellen, dass es eine
Genitiv, Dativ- und eine Akkusativvariante geben muss, die aber formengleich sind:

\eal
\ex Sie gedenken seiner/einander.
\ex Sie helfen ihm/einander.
\ex Sie lieben ihn/einander.
\zl
%
Auch sogenannte Pronominaladverbien\is{Adverb!Pronaminal-} wie \emph{darauf}, \emph{darin}, \emph{worauf}, \emph{worin}
sind problematisch. Diese Formen sind aus einer Präposition und den Elementen \emph{da} und
\emph{wo} zusammengesetzt. Der Begriff \emph{Pronominaladverb} legt nun nahe, dass es in diesen
Wörtern etwas Pronominales gibt. Das kann nur \emph{da} bzw.\ \emph{wo} sein. \emph{da} und
\emph{wo} sind aber nicht flektierbar, werden also nach dem Entscheidungsbaum nicht bei den
Pronomina eingeordnet. 

Dasselbe gilt für Relativwörter wie \emph{wo} in (\mex{1}):
\eal
\ex Ich komme eben aus der Stadt, \emph{wo} ich Zeuge eines Unglücks gewesen bin.\footnote{
 	\citew*[\page 672]{Duden84}.
 	}\label{bsp-wo-ich-zeuge}
\ex Studien haben gezeigt, daß mehr Unfälle in Städten passieren, \emph{wo}\iw{wo!Relativpronomen}
      die Zebrastreifen abgebaut werden, weil die Autofahrer unaufmerksam werden.\footnote{
        taz berlin, 03.11.1997, S.\,23.
        }
\ex Zufällig war ich in dem Augenblick zugegen, \emph{wo} der Steppenwolf 
      zum erstenmal unser Haus betrat und bei meiner Tante sich einmietete.\footnote{
                Herman Hesse, \emph{Der Steppenwolf}. Berlin und Weimar: Auf"|bau-Verlag. 1986, S.\,6.
	}
\zl
Sie sind nicht flektierbar, können also nach dem Entscheidungsbaum nicht in die Klasse der Pronomina
eingeordnet werden. \citet[\page 233]{Eisenberg2004a} stellt fest, dass \emph{wo} eine Art nicht flektierbares
Relativpronomen\is{Pronomen!Relativ-} ist (in Anführungszeichen) und merkt an, dass diese Bezeichnung der ausschließlichen
Verwendung des Begriffes für nominale, also flektierende Elemente zuwiderläuft. Er benutzt deshalb
die Bezeichnung \emph{Relativadverb}\is{Adverb!Relativ-} (siehe auch \citew[\S 856, \S 857]{Duden2005}).

% Der Duden sagt dazu nichts Genaues. Gallmann 2009, C.3.2 führt dessen, deren, derer als Langform von der
% die das auf. Auch Block I S. 3 
% Die Genitivformen dessen und deren können possessiven
% Artikelwörtern nahe kommen:
% (7) Der Hofnarr bewunderte den Kaiser und [ [dessen] neue Kleider].
%
Auch gibt es Verwendungen des Relativwortes \emph{dessen} und des Fragewortes \emph{wessen} in
Kombination mit einem Nomen:
\eal
\ex der Mann, dessen Schwester ich kenne
\ex Ich möchte wissen, wessen Schwester du kennst.
\zl
Nach der Dudenklassifikation müsste man diese Elemente Relativartikelwort und
Interrogativartikelwort nennen.
Sie werden jedoch meist zu den Relativpronomina und Fragepronomina
gezählt (siehe \zb \citew[\page 229]{Eisenberg2004a}).
In Eisenbergs Terminologie ist das ganz problemlos, denn er unterscheidet nicht zwischen
Artikelwörtern, Pronomina und Nomina sondern teilt alle in die Klasse der Nomina ein. Aber auch
Autoren, die zwischen Artikeln und Pronomina unterscheiden, sprechen mitunter von Fragepronomina,
wenn sie Wörter meinen, die in Artikelfunktion oder statt einer kompletten Nominalgruppe vorkommen.

Man sollte insgesamt darauf gefasst sein, dass der Begriff \emph{Pronomen} einfach für Wörter
verwendet wird, die auf andere Einheiten verweisen, und zwar nicht in der Art, wie das Nomina wie
\emph{Buch} oder Eigennamen wie \emph{Klaus} tun, sondern kontextabhängig. \Zb kann man mit dem Personalpronomen \emph{er}
auf einen Tisch oder einen Mann verweisen. Diese Verwendung des Begriffs \emph{Pronomen} liegt quer
zum Entscheidungsbaum in Abbildung~\ref{Abbildung-Wortarten} und schließt Nichtflektierbare wie
\emph{da} und \emph{wo} ein. 
%
% Anke Lüdeling: 16.05.2010
% ganz wichtig ist bei pronomina, dass sie nicht auf ein Nomen, sondern auf einen Diskursreferenten verweisen. 

Expletivpronomina\is{Pronomen!Expletiv-} wie \emph{es} und \emph{das} und das \emph{sich} inhärent
reflexiver Verben\is{Verb!inhärent reflexives} beziehen sich
natürlich nicht auf andere Objekte. Sie werden aufgrund der Formengleichheit mit zu den Pronomina
gezählt. So muss man übrigens auch bei Annahme des engen Pronomenbegriffs verfahren, denn bei Expletivpronomina
gibt es keine Formunterschiede für verschiedene Kasus und auch keine Genus- oder
Numerusvarianten. Ginge man nach Schema F vor, würden die Expletiva in der Klasse der
Nichtflektierbaren landen. Nimmt man an, dass \emph{es} wie das Personalpronomen eine Nominativ- und
eine Akkusativvariante mit gleicher Form hat, landet man im nominalen Bereich, man muss dann aber
eingestehen, dass die Annahme von Genus für \emph{es} nicht sinnvoll ist bzw.\ muss \emph{es} zu den
Nomina zählen, wenn man in Analogie zum Personalpronomen das Genus Neutrum annehmen will.%
\is{Pronomen|)}

Wir haben noch nicht besprochen, wie mit den kursiv gesetzten Wörtern in (\mex{1}) verfahren wird:
\eal
\ex das \emph{geliebte} Spielzeug
\ex das \emph{schlafende} Kind
\ex die Frage des \emph{Sprechens} und \emph{Schreibens} über Gefühle
\ex Auf dem Europa-Parteitag fordern die \emph{Grünen} einen ökosozialen Politikwechsel.
\ex\label{Wortart-adverbiales-Adjektiv} Max lacht \emph{laut}.
\ex\label{Wortart-Satzadverb-Adjektiv} Max würde \emph{wahrscheinlich} lachen.
\zl
\emph{geliebte} und \emph{schlafende} sind Partizipformen von \emph{lieben} bzw.\
\emph{schlafen}. Diese Formen werden traditionell mit zum verbalen Paradigma gezählt. In diesem
Sinne sind \emph{geliebte} und \emph{schlafende} Verben. Man spricht hier von der lexikalischen
Wortart. In diesem Zusammenhang ist auch der Begriff \emph{Lexem}\is{Lexem} relevant. Zu einem Lexem gehören
alle Wortformen eines Flexionsparadigmas.\is{Flexion}\is{Paradigma} Im klassischen Verständnis
dieses Begriffs gehören auch 
alle regelmäßig abgeleiteten Formen dazu, \dash, dass bei Verben die Partizipien und auch
nominalisierte Infinitive zu verbalen Lexemen gehören. Diese Auf"|fassung wird nicht von allen
Sprachwissenschaftlern geteilt. Insbesondere ist problematisch, dass man hier verbale mit nominalen
und adjektivischen Flexionsparadigmen mischt, denn \emph{Sprechens} steht im Genitiv und auch die
adjektivischen Partizipien flektieren nach Kasus, Numerus und Genus. Auch bleibt unklar, warum
\emph{schlafende} zum verbalen Lexem gezählt wird, \emph{Störung} dagegen ein eigenes nominales
nicht zum Lexem \emph{stören} gehörendes Lexem bilden soll. Ich folge der neueren Grammatikforschung
und nehme an, dass bei Prozessen, in denen sich die Wortart ändert, ein neues Lexem
entsteht. In diesem Sinne gehört \emph{schlafende} nicht zum Lexem \emph{schlafen}, sondern ist eine
Form des Lexems \emph{schlafend}. Dieses Lexem hat die Wortart Adjektiv und flektiert auch entsprechend. 

Wo genau die Grenze zwischen Flexion und Derivation\is{Derivation} (der Bildung neuer Lexeme) zu ziehen ist, ist,
wie gesagt, umstritten. So zählen \citet*[\page263--264]{SWB2003a} die Bildung des Present Participle (\emph{standing})
und des Past Participle (\emph{eaten}) im Englischen\il{Englisch} zur Derivation, weil diese Formen im
Französischen\il{Französisch} noch für Genus und Numerus flektiert werden müssen. 

Adjektive wie \emph{Grünen} in (\mex{0}d) werden nominalisierte Adjektive genannt und auch
wie Nomina groß geschrieben, wenn es kein Nomen gibt, das aus dem unmittelbaren Kontext ergänzt
werden kann:
\ea
A: Willst du den roten Ball haben?\\
B: Nein, gib mir bitte den grünen.
\z
In der Antwort in (\mex{0}) ist das Nomen \emph{Ball} ausgelassen worden. Eine solche Auslassung
liegt in (\mex{-1}d}) nicht vor. Man könnte hier nun genauso annehmen, dass ein einfacher
Wortartenwechsel stattgefunden hat. Wortartenwechsel ohne ein sichtbares Affix nennt man
\emph{Konversion}\is{Konversion}. Die Konversion wird von einigen Wissenschaftlern als Unterart der
Derivation\is{Derivation} behandelt. 
Das Problem ist jedoch, dass \emph{Grüne} genau wie ein Adjektiv flektiert ist und auch das
Genus in Abhängigkeit vom Bezugsobjekt variiert:
\eal
\ex ein Grüner hat vorgeschlagen, \ldots
\ex eine Grüne hat vorgeschlagen, \ldots
\zl
Man hat also hier eine Situation, in der ein Wort zweierlei Eigenschaften hat. Man hilft sich, indem
man von einem nominalisierten Adjektiv spricht: Die lexikalische Wortart\is{Wortart!lexikalische} ist Adjektiv, die
syntaktische Wortart\is{Wortart!syntaktische} ist Nomen.

Das Wort in (\ref{Wortart-adverbiales-Adjektiv}) ist wie ein Adjektiv flektierbar, sollte also nach
unseren Tests auch als Adjektiv eingeordnet werden. Mitunter werden solche Adjektive aber dennoch zu
den Adverbien gezählt. Der Grund hierfür ist, dass diese unflektierten Adjektive sich so ähnlich
wie Adverbien verhalten:
\ea
Max lacht immer/oft/laut.
\z
Man sagt dann, dass die lexikalische Wortart Adjektiv und die syntaktische Wortart Adverb ist. Die
Einordnung von Adjektiven wie \emph{laut} in (\mex{0}) in die Klasse der Adverbien wird nicht von
allen Autoren angenommen. Stattdessen spricht man von einem adverbial verwendeten Adjektiv, \dash,
dass man annimmt, dass auch die syntaktische Wortart Adjektiv ist, dass es jedoch eine
Verwendungsweise gibt, die der der Adverbien entspricht (siehe \zb
\citew[Abschnitt~7.3]{Eisenberg2004a}). Das ist parallel zu den Präpositionen, die auch in
verschiedenen syntaktischen Kontexten auf"|treten können: 
\eal
\ex Peter schläft im Büro.
\ex der Tisch im Büro
\zl
In beiden Beispielen in (\mex{0}) liegen Präpositionalgruppen vor, aber in (\mex{0}a) modifiziert
\emph{im Büro} wie ein Adverb das Verb \emph{schläft} und in (\mex{0}b) bezieht sich \emph{im Büro}
auf das Nomen \emph{Tisch}. Genauso kann sich \emph{laut} wie in (\mex{1}) auf ein Nomen oder wie in
(\mex{-1}) auf ein Verb beziehen.
\ea
die laute Musik
\z 

\noindent
Als letzten kniffligen Fall möchte ich (\ref{Wortart-Satzadverb-Adjektiv}) besprechen. Wörter wie
\emph{wahrscheinlich}, \emph{hoffentlich} und \emph{glücklicherweise} werden Satzadverbien\is{Satzadverb}
genannt. Sie beziehen sich auf die gesamte Aussage und geben die Sprecherhaltung wieder. Zu dieser
semantisch begründeten Klasse gehören auch Flektierbare wie \emph{vermutlich} und eben
\emph{wahrscheinlich}. Wenn man all diese Wörter Adverb nennen will, dann muss man davon
ausgehen, dass bei Wörtern wie \emph{wahrscheinlich} eine Konversion stattgefunden hat, \dash, dass
\emph{wahrscheinlich} die lexikalische Wortart Adjektiv und die syntaktische Wortart Adverb hat.



\section{Köpfe}
\label{Abschnitt-Kopf}

% Der Kopf"=Begriff spielt in der Kopfgesteuerten Phrasenstrukturgrammatik
% eine wichtige Rolle, wie man unschwer am Namen der Theorie erkennen kann.
%
Der\is{Kopf|(} Kopf einer Wortgruppe/""Konstituente/""Phrase ist dasjenige Element,
das die wichtigsten Eigenschaften der Wortgruppe/""Konstituente/""Phrase bestimmt.
Gleichzeitig steuert der Kopf den Aufbau der Phrase, \dash, der Kopf verlangt
die Anwesenheit bestimmter anderer Elemente in seiner Phrase. Die Köpfe sind
in den folgenden Beispielen kursiv gesetzt:
\eal
\ex \emph{Träumt} dieser Mann?
\ex \emph{Erwartet} er diesen Mann?
\ex \emph{Hilft} er diesem Mann?
\ex \emph{in} diesem Haus
\ex ein \emph{Mann}
\zl
Die Verben bestimmen den Kasus ihrer jeweiligen Argumente (der Subjekte und Objekte).
In (\mex{0}d) bestimmt die Präposition den Kasus der Nominalphrase \emph{diesem Haus} und
leistet auch den semantischen Hauptbeitrag: Ein Ort wird beschrieben. (\mex{0}e)
ist umstritten: Es gibt sowohl Wissenschaftler, die annehmen, dass der Determinator
der Kopf ist (\zb\LATER{\cite{Brame81a,Brame82a} \citealp[\page 90]{Hudson84a};
}\citealp{Hellan86a,Abney87a,Netter94,Netter98a}), als auch solche, die annehmen, dass das Nomen der
Kopf ist (\zb \citew{vanLangendonck94a}; \citealp[\page 49]{ps2}; \citealp{Demske2001a};
\citealp[Abschnitt~6.6.1]{MuellerLehrbuch1}; \citew{Hudson2004a}). 


Die Kombination eines Kopfes mit einer anderen Konstituente wird \emph{Projektion
des Kopfes}\is{Projektion} genannt. Eine Projektion, die alle notwendigen Bestandteile zur Bildung
einer vollständigen Phrase enthält, wird \emph{Maximalprojektion}\is{Projektion!Maximal-}
genannt. Ein Satz\is{Satz} ist die Maximalprojektion eines finiten Verbs.

Abbildung~\vref{Abbildung-beschriftete-Schachteln} zeigt die Struktur von (\mex{1}) im
Schachtelmodell.
\ea
Der Mann liest einen Aufsatz.
\z
Im Gegensatz zu Abbildung~\ref{Abbildung-Schachteln} sind die Schachteln beschriftet worden.
\begin{figure}[htbp]
\centerline{%
\begin{pspicture}(0,0)(7.8,3.4)
     \rput[bl](0,0){%
\psset{fillstyle=solid, framearc=0.25,framesep=5pt}
\psframebox{%
\begin{tabular}{@{}l@{}}
VP\\
\psframebox{%
\begin{tabular}{@{}l@{}}
NP\\[2mm]
       \psframebox{\begin{tabular}{@{}l@{}}
                   Det\\der
                   \end{tabular}}
       \psframebox{\begin{tabular}{@{}l@{}}
                   N\\Mann
                   \end{tabular}}
\end{tabular}}
\psframebox{\begin{tabular}{@{}l@{}}
                   V\\liest
                   \end{tabular}}
\psframebox{%
\begin{tabular}{@{}l@{}}
NP\\[2mm]
           \psframebox{\begin{tabular}{@{}l@{}}
                   Det\\einen
                   \end{tabular}}
           \psframebox{\begin{tabular}{@{}l@{}}
                   N\\Aufsatz
                   \end{tabular}}
\end{tabular}}
\end{tabular}}}
%\psgrid
    \end{pspicture}}
\caption{\label{Abbildung-beschriftete-Schachteln}Wörter und Wortgruppen in beschrifteten Schachteln}
\end{figure}
Die Beschriftung enthält die Wortart des wichtigsten Elements in der Schachtel. VP steht für
Verbalphrase und NP für Nominalphrase. VP und NP sind die Maximalprojektionen der jeweiligen Köpfe.

Jeder, der schon einmal verzweifelt im Andenkenschrank die Photos von der Hochzeit seiner Schwester gesucht hat, wird
nachvollziehen können, dass es sinnvoll ist, die Kisten in solchen Schränken und auch die darin
enthaltenen Photoalben zu beschriften.

Interessant ist nun, dass der genaue Inhalt der Schachteln mit sprachlichem Material für das
Einsetzen in größere Schachteln nicht unbedingt wichtig ist. So kann man zum Beispiel die
Nominalphrase \emph{der Mann} durch \emph{er} oder auch durch etwas viel Komplexeres wie \emph{der
  Mann aus Stuttgart, der das Seminar zur Entwicklung der Zebrafinken besucht} ersetzen. Es ist
jedoch nicht möglich, an dieser Stelle \emph{die Männer} oder \emph{des Mannes} einzusetzen: 
\eal 
\ex[*]{ 
Die Männer liest einen Aufsatz.  
} 
\ex[*]{ Des Mannes liest einen Aufsatz.  
} 
\zl 
Das liegt daran, dass \emph{die Männer} Plural ist, das Verb
\emph{liest} aber im Singular steht. \emph{des Mannes} ist Genitiv, an dieser Stelle ist aber nur
ein Nominativ zulässig. Man beschriftet deshalb die Schachteln immer mit aller Information, die für
das Einsetzen in größere Schachteln wichtig
ist. Abbildung~\vref{Abbildung-ausfuehrlich-beschriftete-Schachteln} zeigt unser Beispiel mit 
ausführlichen Beschriftungen.
\begin{figure}[htbp]
\oneline{%
\begin{pspicture}(0,0)(14.2,3.4)
     \rput[bl](0,0){%
\psset{fillstyle=solid, framearc=0.25,framesep=5pt}
\psframebox{%
\begin{tabular}{@{}l@{}}
VP, fin\\[2mm]
\psframebox{%
\begin{tabular}{@{}l@{}}
NP, nom, 3, sg\\[2mm]
       \psframebox{\begin{tabular}{@{}l@{}}
                   Det, nom, mas, sg\\der
                   \end{tabular}}
       \psframebox{\begin{tabular}{@{}l@{}}
                   N, nom, mas, sg\\Mann
                   \end{tabular}}
\end{tabular}}
\psframebox{\begin{tabular}{@{}l@{}}
                   V, fin\\liest
                   \end{tabular}}
\psframebox{%
\begin{tabular}{@{}l@{}}
NP, akk, 3, sg\\[2mm]
           \psframebox{\begin{tabular}{@{}l@{}}
                   Det, akk, mas, sg\\einen
                   \end{tabular}}
           \psframebox{\begin{tabular}{@{}l@{}}
                   N, akk, mas, sg\\Aufsatz
                   \end{tabular}}
\end{tabular}}
\end{tabular}}}
%\psgrid
    \end{pspicture}}
\caption{\label{Abbildung-ausfuehrlich-beschriftete-Schachteln}Wörter und Wortgruppen in ausführlich beschrifteten Schachteln}
\end{figure}
Die Merkmale des Kopfes, die für die Bestimmung der Kontexte, in denen eine Phrase verwendet werden
kann, relevant sind, werden auch \emph{Kopfmerkmale}\is{Kopfmerkmal} genannt. Man sagt, dass diese Merkmale vom Kopf
projiziert werden.
\is{Kopf|)}



                                                    
        
\lipsum 
\lipsum[3-10]  

\newpage

\layout


\backmatter
\bibliographystyle{natbib.myfullname}
\bibliography{bib-abbr,biblio}



%\cleardoublepage
%\printglossary

\cleardoublepage
\small
%\addcontentsline{toc}{chapter}{Index} 

%\pdfbookmark[0]{Index}{Index}

%\printindex{autind}{Index der Namen}
%
%\printindex{wordind}{Index der Beispielwörter}
%
%\printindex{rev-wordind}{Reverse Index of Expressions}
%
%\printindex{langind}{Index of Languages and Dialects}
%
%\printindex{subind}{Index der Termini}

%\addcontentsline{toc}{chapter}{Namensregister}
\pdfbookmark[0]{Namensregister}{Namensregister}
\printindex[aut]
%\addcontentsline{toc}{chapter}{Verzeichnis der Sprachen}
\pdfbookmark[0]{Verzeichnis der Sprachen}{Verzeichnis-der-Sprachen}
\printindex[lan]
%\addcontentsline{toc}{chapter}{Sachregister}
\pdfbookmark[0]{Sachregister}{Sachregister}
\printindex

\end{document}

      
