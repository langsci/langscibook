\chapter{Conclusion}\label{conclusion}

%\setcounter{exx}{0}

\section{Summary of the findings}

In this study, I have analyzed the micro-alignment of a number of marked"=S languages.
Marked"=S languages exhibit a peculiar pattern of encoding the basic (in-) transitive roles S, A and P in that they overtly mark the S relation of intransitive verbs while using a non-overtly coded form of a noun for one of the arguments of transitive verbs (for more details cf. the definition and examples of marked"=S languages in Section~\ref{definition}). 
In addition to the S, A and P roles, this study has investigated the coding of a number of additional S-like roles.
The additional roles have been selected from the contexts defined in Chapter~\ref{method}.
While some of these roles behaved like regular overtly-marked intransitive S arguments in most languages, others were almost exclusively encoded by the zero-coded case-form. 
Figure~\ref{SumRole} summarizes  the results of the previous chapter, in which the different contexts have been compared with one another.
The roles have been ordered with respect to their likeliness to be encoded in the same way as intransitive S arguments in the languages of the sample. 
The further to the top a role is located, the more often it is encoded with the S-case.
Roles that are represented next to each other in Figure~\ref{SumRole} exhibit almost identical behavior in this respect.
These preferences are quite stable between different calculations, both when including all individual languages on which enough data was available (without employing any mechanisms of genealogical and/or areal control) and when the data were normalized to include only one data point per genus.

\begin{figure}[ht] \centering \fbox{
\begin{picture}(285,280)

\put(100,255){\makebox(70,20){intransitive S}}
\put(100,230){\makebox(70,20){locational S}}
%\put(155,230){\makebox(70,20){S of VIC}}
\put(45,205){\makebox(70,20){S of positive existential}}
\put(155,205){\makebox(70,20){S of VDC}}
\put(100,180){\makebox(70,20){S of adverbial clause}}
\put(100,155){\makebox(70,20){S of nominal predication}}
\put(100,130){\makebox(70,20){S of negative existential}}
\put(155,105){\makebox(70,20){S of relative clause}}
\put(45,105){\makebox(70,20){emphatic S}}
\put(100,80){\makebox(70,20){S of complement clause}}
\put(100,55){\makebox(70,20){predicate nominal}}
\put(45,30){\makebox(70,20){attributive possessor}}
\put(155,30){\makebox(70,20){term of address}}
\put(100,5){\makebox(70,20){citation form}}

\put(260,20){\line(0,1){250}}
\put(270,20){\line(0,1){250}}
\put(260,270){\line(1,0){10}}

\put(255,20){\line(1,0){5}}
\put(270,20){\line(1,0){5}}

\put(265,10){\line(1,1){10}}
\put(265,10){\line(-1,1){10}}

\end{picture}}

\caption{Coding of S-like roles in marked"=S languages (ordered from most S-like to least S-like)}\label{SumRole}
\end{figure}


%In addition the individual languages and genera have been compared with respect to their frequency of employing a certain case-form for a given role. In all cases two different encoding-patterns have been used for the data. The data has either been analyzed as coding via the zero-case, the S-case or an other overt case-form, or a two-way distinction between overtly coded forms and zero-coded forms has been made. The data have been rather stable between these two types of encoding as well, differences in pattering between the two encodings have only been found for a minority of roles and languages.
Two different methods of analysis, first through a ranking by percentage and second through the more sophisticated NeighborNet algorithm, have revealed a similar pattern for the different roles, namely, a gradual shift from coding via S-case to coding via zero-case. 
The subject-like roles, especially subjects of locational clauses, being the one extreme and extra-syntactic roles, especially the citation form of a noun, being the other.
%On language level the two methods have as well led to similar results. 

Apart from distinguishing which roles behave most or least like intransitive S arguments in their encoding, the similarities and differences between the individual languages have also been evaluated. 
There is a clear subtype of marked"=S languages to which most marked"=S languages of North America belong (excluding only Maidu\il{Maidu}). 
Further, there is an African type of marked"=S comprising the Afro-Asiatic languages of the Omotic and East Cushitic genera as well as Surmic. 
The Nilotic languages do not behave in a consistent pattern that would allow to classify them as following one or the other pattern. 
Languages of the Pacific exhibit some similarities, but the data does not justify proposing a distinct subtype of marked"=S for them.% probably due to the small number of languages in the overall sample. 

After this brief summary of the results, I will now discuss the implications of these findings for the understanding of marked"=S languages.
The central motivation of this study has been to test whether the unexpectedness of the marked"=S coding-pattern based on the purely formal aspects of the system can be adequately explained in terms of functional motivations, as it has been proposed in \citet{Koenig:2006}. 
This major question will be addressed in Section~\ref{discussion}. While an important factor in understanding marked"=S languages, the range of functions a case-form has cannot be the sole explanation for their existence. 
Furthermore, to allow for meaningful generalizations over the functions of marked and unmarked case-forms, one needs to apply a consistent definition of markedness that is independent of functional considerations.
Further, I will review the micro-alignment approach that I have used for this investigation and comment on its usefulness and limitations in Section~\ref{methodcomments}.
I also pointed out in the introductory chapter (Section~\ref{theoretical}) that marked"=S languages are a serious challenge for some of the more formalistic approaches to alignment and case-assignment in particular. 
I will take up this discussion in Section~\ref{consequencesformal} and comment on the possibilities of integrating the finding on marked"=S languages into these formal approaches. 
Finally, I will address some questions that remain open or have been raised by the findings of this study, and which should be targeted in future research (Section~\ref{furtherresearch}).  

\section{Generalizations about the functional motivations for marked"=S languages}\label{discussion}\is{marked-S languages!functional definition|(}

Marked"=S languages have caught the attention of linguists based on a strictly formal criterion, namely the overt marking of the S argument found with these languages. 
The unexpectedness of the marked"=S system is for example expressed as Universal 38 in \citet[75]{Greenberg:1963}.
The two points of view from which the existence of this unusual case-systems have been considered are  the historical development of these systems and their functionally-based motivations. Of course, these two points of view do not have to be mutually exclusive. Historical changes in the grammar of a language can certainly have a functional motivation; some linguists will even argue that they must have one. 
In contrast to the formal definition, a functional definition of marked"=S languages has also been
proposed \citep{Koenig:2006}.%
\enlargethispage{\baselineskip}
In this definition, the functional range of individual case-forms is the central criterion for the `markedness' or `unmarkedness' ascribed to each case-form.
The functional aspect, i.e. the number of functions covered by the case-forms, is an important aspect in the study of marked"=S languages.  
 However, the number of roles covered by either zero-case or S-case does vary considerably between the languages studied here.\footnote{As defined in Section~\ref{label} the term S-case is employed for the case-form that covers the function of encoding transitive S arguments (i.e. the nominative in nominative"=accusative languages and the absolutive in ergative"=absolutive languages). The term zero-case refers to the case-form that is used for the non-S-case-marked argument of transitive verbs, and that is zero-coded in marked"=S languages (i.e. the accusative in nominative"=accusative languages and the ergative\is{case!individual forms!ergative} in ergative"=absolutive languages).}

For some languages the existence of marked"=S coding cannot be plausibly argued for based on functional motivations of this type.
From the point of view of the formal encoding of case-forms, the Californian language Maidu\il{Maidu} is a regular marked"=S language with an overt Nominative case-marker and a zero-coded Accusative. 
Yet, the range of functions that the zero-coded Accusative covers does not extend far beyond the encoding of transitive P arguments. 
Maidu\il{Maidu} is, however, not the only problematic case for the functional account of marked"=S languages. 
The languages that are identified as being of the marked"=S type by a functional rather than purely formal definition, i.e. the languages referred to as Type 2 marked"=nominative languages by \citet[658]{Koenig:2006}, do not use the `zero-coded' form for as many of the roles as the languages meeting the strict form-based definition do.  
For the Type 2 marked"=nominative languages, the function of the accusative case as citation form (and in other extra-syntactic contexts) is taken as the main argument to consider this form as being the more basic form. 
Based on the data collected in this study, the use as a citation form is also the main function of the zero-coded form after its use as the case-form of transitive P arguments, while the Type 2 languages do not extend its use to more subject-like roles.

Taking a radically economical approach to case-marking, one could propose the following explanation. 
If two case-forms of a noun do differ in the number of segments they consist of, the form that has the smaller number of segments will be preferred because of its lower production effort. 
If the two case-forms consist of the same amount of segments, no such pressure exists to choose one form over the other. 
Linguistic explanations that propose such radically economy-based argumentations can be criticized on various grounds. 
One argument against this approach would be that actual ease of articulation rather than the bare number of segments is a stronger factor. 
Extra segments added to a form, such as final vowels, can lead  to a less complex syllable structure and thus increase the ease of articulation. 
Consequently this entire discussion returns to the initial question of how one defines the concept of linguistic markedness, which I discussed in Section~\ref{markedness}. 
Since different definitions of markedness can result in different identification of marked versus unmarked forms in individual languages, there is always the possibility of choosing the definition that best fits one's analysis of any language (e.g. defining the `marked' form as the one with the more marked syllable structure even though this might be the morphologically zero-coded form). 
While this approach improves the consistency of an analysis on a per-language level, comparability between languages and consequently cross-linguistic generalizations over marked"=S coding are rendered meaningless, since this leads to a circular definition of the marked"=S system. 
If one chooses the definition of marked versus unmarked case-form which best fits the prediction that the unmarked form is used in more contexts, then it necessarily follows that the unmarked form is made a wider use of in marked"=S languages.\is{marked-S languages!functional definition|)} 

\section{Concluding remarks on the micro-alignment approach}\label{methodcomments}\is{micro-alignment approach|(}

As Chapter~\ref{typology} has shown, languages belonging to the marked"=S coding type behave quite differently in terms of micro-alignment structures. 
While the pattern of marking the S, A and P functions of prototypical verbs employs the same pattern of case-marking in these languages, the marking of other types of clauses differs strongly between the languages. 
Differences in encoding between different clause-types or based on other factors, like the ones discussed in Section~\ref{grammar-based} are also known from languages with other coding-systems. 
Still, coarse classifications of language as being of the nominative"=accusative or ergative"=absolutive type are made frequent use of in linguistic studies.
Since much of the debate on marked"=S languages focuses on overt coding properties (and the unexpectedness of this pattern), an in-depth investigation of the coding-patterns of more than just the most basic clauses is necessary to fully understand this unusual pattern.  

The contexts and roles chosen in this study have been defined based on the variation that the languages studied here exhibited. 
Data for all languages was basically gathered in a parallel fashion and not one language after the other, since the study of one language potentially revealed a new pattern of variation that could profitably be included in the study. 
On the other hand, based on an initial list of possibly interesting domains of grammar, data have been collected on roles that did not show any interesting patterns in any or almost any languages of the sample and have thus not been presented in the final study. 

A small drawback of this approach is the frequent omission of parts of the grammar in the description of languages that do not exhibit any variation in the respective domain.\footnote{The same is, however, true for most comparative work that is carried out through available descriptions of languages.} 
Negative evidence, especially when dealing with a very limited set of examples as data base, cannot be taken as evidence of the absence of a certain pattern in a given languages. 
This has led to a considerable number of missing data points. Consequently, the respective percentage of languages that deviate from the pattern conceived as the norm, i.e. S-case-marking on subject-like roles, might be a little too high in the figures presented in Chapter~\ref{typology}. 
This is based on the assumption that if a grammar does not discuss a given context, there will more likely be no variation from the standard pattern in this domain.
In addition, the larger the number of languages studied, the larger the number of contexts of interest will become with this approach. 
Consequently, when relying largely on secondary data, the larger the number of missing data points will become. 

Given these limitations, the micro-alignment approach -- however, this is true for any approach that aims at including very fine-grained distinctions on any domain of grammar -- is best employed in more detailed studies operating samples of a smaller size. 
Preferably, primary data on the languages studied should be available, which is, however, difficult and tedious to obtain for the majority of the world's languages. 
The approach is less applicable in large scale typological studies aiming at a large number of languages included\is{micro-alignment approach|)}. 

\section{Consequences for formal theories}\label{consequencesformal}

For the languages of the marked"=S type one can identify a case-form that can be analyzed as a default\is{case!default} case, a notion that many formal theories employ.
However, this case-form is not necessarily linked to the form that is used to encode the subject function in a clause. 
For most marked"=S languages, the case-form that should be considered the default\is{case!default} case by factors such as which form is the most basic one in terms of morphological structure (derived forms versus underived forms).
The form which is used in extra-syntactic contexts does coincide with the form used to encode the non-subject argument in basic transitive clauses. 

In Chapter~\ref{introduction}\is{Lexical Decomposition Grammar|(}, I briefly introduced the feature system of Lexical Decomposition Grammar (LDG, \citealp{Wunderlich:1997,Stiebels:2002}). 
In this approach, the default\is{case!default} status, which is ascribed to the nominative or absolutive\is{case!individual forms!absolutive} case, is mirrored through the feature representation of the default\is{case!default} case, which is an empty set. Other cases have non-empty sets of features, and thus are more restricted in their use. 

As argued above for marked"=S languages, one has to assume that the accusative case (or respectively the ergative\is{case!individual forms!ergative} case) functions as the default\is{case!default} case.
If one wants to keep the generalization that the default\is{case!default} case-form should have a feature representation consisting of an empty set of features, and thus being potentially employable in all contexts, one would have to assume that the cases used in marked"=S languages have a different set of features than the  standard feature representations proposed in LDG (cf. Section~\ref{theoretical}).
The following feature representations could be employed:

\begin{itemize}
\item marked"=nominative: [-hr]
\item default accusative: [\quad]
\item marked"=absolutive: [-lr]
\item default Ergative: [\quad]
\end{itemize}

(\ref{theta2}) and (\ref{LDGalternate}) demonstrate the linking of a basic transitive verb using these feature specifications. 
As in in the standard LDG approach, feature specifications for the arguments of a verb are derived from the semantic form and the theta-structure of a verb.
The cases that are available from the lexicon of the language are then matched to the argument positions based on their feature specification, choosing the most concrete case available for each position (\ref{LDGalternate}). 
In marked"=nominative languages, the overtly coded nominative and default\is{case!default} accusative are available. 
Both argument positions could be filled with the default\is{case!default} accusative.
However, the nominative is a better match for the x argument since it is the more concrete case (i.e. it has more features specified) and its feature specification as [-hr] (`there is no higher role') is compatible with the feature specification of this argument position. 
Conversely, in marked"=absolutive languages the two available case-forms, the overtly coded absolutive\is{case!individual forms!absolutive} and default\is{case!default} ergative\is{case!individual forms!ergative}, are matched to the x and y argument position by the same mechanism.  


\begin{exe}\ex\label{theta2}
$ \underbrace{ \lambda \text{x} \quad \lambda \text{y} \quad \lambda \text{s}}_{\text{\normalsize\rm
      theta-structure}}$ \qquad $ \underbrace{\{ \text{see (x,y)} \}
    \text{(s)}}_{\text{\normalsize\rm semantic form}} $ 
\end{exe}

%\pagebreak
\begin{exe}
\ex\label{LDGalternate}
\begin{tabbing}
nominative"=accusative \quad \= \nom{} \quad \= \kill
{}\>$\lambda \text{y}$ \> $\lambda \text{x}$\\
{}\>{+hr} \> {-hr}\\
{}\>{-lr} \> {+lr}\\
marked"=nominative\>\acc{}	\> \nom{}\\
marked"=absolutive\>\abs{} \> \erg{}
\end{tabbing}
\end{exe}

The standard case-representations of LDG only make use of features that have a positive specification. 
In contrast, for the specifications I proposed for the cases of marked"=S languages, negative feature specifications are used. 
This procedure goes against most considerations relevant to the setup of feature systems, in which negative feature specifications are often equated with underspecification with respect to the given feature.
Without doubt, the introduction of the additional cases and their proposed feature specifications would deprive the LDG approach of some of its elegance.
Yet one could argue that this dispreferred feature specification employed to model case-assignment in marked"=S languages is reflected through their cross-linguistic rarity.   
Another possibility, which would not make it necessary to include negative feature specification for the representation of cases would be to introduce a new set of features for languages of the marked"=S type. 
Since this section is not meant as a proposal to reformulate LDG, but rather a sketch of how marked"=S languages could be integrated into that theory, I have restricted myself to employing the features that are already provided by the theory. 

However, there is another issue that makes the inclusion of marked"=S languages into the LDG theory problematic.
While the proposed feature values lead to the right case-assignment for prototypical transitive and intransitive clauses (\ref{LDGalternate}), some minor clause-types can not easily be analyzed by the modified feature system.
The previous chapters illustrated that marked"=S languages make common use of the zero-case in subject like roles, e.g. the subject of existential clauses (cf. Chapter~\ref{existpred}). 
While there is, in principle, no conflict in assigning the default\is{case!default} accusative to existential subjects, the Elsewhere Principle would predict that nominative case is assigned to these arguments. 
Lexical case-assignment is possible within the LDG framework, but it is counter-intuitive to the whole notion of a default\is{case!default} case if the default\is{case!default} case would have to be lexically assigned. 

At the present moment, marked"=S languages pose a challenge to LDG and other formal theories that employ similar mechanisms for case-assignment.
The issues raised here should be resolved by the proponents of such theories if they want to make general claims about the nature of case-assignment in human language. 
At present it appears that one has at least to abandon one central assumption in order to include marked"=S languages.
If one keeps the standard LDG case features for marked"=S languages, that will assign the nominative (or absolutive\is{case!individual forms!absolutive}) case to all subjects automatically, while clause-types that take zero-coded accusative (or ergative\is{case!individual forms!ergative}) subject could be handled through lexical case-assignment.
In this case, the notion of default\is{case!default} case becomes somewhat arbitrary, since many properties typically associated with default\is{case!default} case-forms (e.g. use in citation) are not fulfilled by the case-form that has the default\is{case!default} feature representation. 
If one accepts the default\is{case!default} accusative and default\is{case!default} ergative\is{case!individual forms!ergative} as legitimate cases in the theory, one has to resolve the problem of lexical assignment of default\is{case!default} case (or possibly find other mechanisms to block the assignment of the marked"=nominative/ marked"=absolutive in some contexts).\is{Lexical Decomposition Grammar|)}

While the existence of marked"=S languages results in abandoning at least one of the major generalizations for the LDG approach, other formal approaches to case-marking have no such principled difficulties in integrating languages of this type. 
Yet these other approaches would still benefit from considering marked"=S languages.
De Hoop \& Malchukov \citeyearpar{deHoop:2008}\is{Optimality Theory|(}, \citet{Malchukov:2008} and \citet{Malchukov:2011} provide an optimality-theoretic approach that can account for a number of splits in alignment systems found in different languages of the world.\footnote{Optimality Theory (mostly abbreviated as OT) is a formal mechanism that describes languages and more particularly linguistic variation though a set of supposedly universal and violable constraints. 
The ranking of these constraints, which differs between languages, leads to different outputs in the surface grammar of individual languages. 
The more highly a constraint is ranked in a language, the more important it is in that language and the more likely the effects of that constraint will be visible in the surface structure of that language. 
For a more detailed discussion of Optimality Theory, the reader is referred to the literature \citep{PrinceSmolensky:2004,Kager:1999,Legendre:2001}.} 
These analyses draw on the two prominent functions of case-marking, the discriminating function and the identifying function \citep[91--939]{Mallinson:1981}. 
 Constraints motivated by the two functions and their respective rankings are employed to account for splits based on factors such as the animacy and definiteness of the nouns involved. 
The approach has also been extended to alignment splits that are conditioned by the tense or aspect of the clause \citep{Malchukov:2011,Malchukov.tam}. 
All languages modeled in these papers are of the standard, i.e. non-marked, types of nominative"=accusative or ergative"=absolutive alignment. 
A modeling of languages of the marked"=S type in this approach would definitely be useful in order to expand the explanatory power of the approach.
I will not attempt to give a fully-fledged optimality-theoretic analysis of marked"=S coding at this point, but rather limit myself to a few general reflections on the integration of marked"=S languages into an optimality-theoretic approach. 
In order to model the general pattern of marked"=S languages in this approach, constraints that penalize overt morphology cannot be ranked very highly, since overt marking of intransitive S arguments would not be possible when these constraints were undominated. 
However, these markedness constraints do apparently have some effect in these languages, since the case-form with less or no overt coding is preferred for a number of different roles.
Furthermore, the approach of \citet{deHoop:2008} and \citet{Malchukov:2011} does not included data with the same level of granularity as I have discussed in this study, but rather have focused on prototypical transitive clauses, somewhat neglecting more specialized clause-types such as nominal predication, existential predication, and the like. 
More fine-grained information on the alignment system of a language could very probably be included in this approach. 
However, they might increase the complexity of the analysis considerably. 
Also, most optimality-theoretic analyses do not aim at depicting the entire complexity of a single language but highlight more fundamental differences between a number of languages which can be accounted for by the rearrangement of a small number of selected constraints. 
However, in order to plausibly model the grammar of an individual language (or even all possible grammars of the world's languages), optimality-theoretic approaches should eventually be able to account for these variations between different types of constructions.\is{Optimality Theory|)}
 
\section{Future research}\label{furtherresearch}

This study has demonstrated that the usage of the zero-case and S-case differ greatly between individual languages. 
As pointed out already in Section~\ref{usage-based}, another interesting factor to investigate would be actual usage-fre\-quen\-cies of the two forms. 
Especially for the languages that do not use the zero-case to encode a large number of roles, it would be a worthwhile research question to gather data on the usage frequency of the two case-forms. 
Factors such as the frequent omission of overt subject NPs could lead to the situation that the form used to encode the non-subject argument of transitive clauses is indeed used more often in discourse.   

Another point that could not be addressed in sufficient detail here is the intriguing marked"=S pattern found in a number of languages spoken in the Pacific area. 
These languages exhibit the marked"=S coding properties only in certain discourse contexts, mostly associated with constituent focus. 
To reach a better understanding of this type of marked"=S structure, original fieldwork on a number of these languages would doubtless be necessary. 

For all areas which I have studied, some kind of contact scenario that can explain the existence of the marked"=S pattern appears to be plausible. 
In East-Africa, the common assumption appears to be that the pattern originated within the languages of the Afro-Asiatic family and spread to surrounding languages such a the Surmic languages of the Nilo-Saharan family and the Nilotic language Turkana\il{Turkana}, which pattern along with the Afro-Asiatic marked"=S languages.
Also the similarity of the coding-pattern of the Yuman languages and the unrelated language Wappo\il{Wappo} could hypothetically be the traces of a prior, and supposedly larger, areal marked"=S pattern in North America, including intervening languages that abandoned the marked"=S system or became extinct before they could be documented.  
As I have pointed out, in order to study the marked"=S languages of the Pacific region and its geographical distribution and possible contact scenarios, first the majority pattern of this region, i.e. discourse-based overt S-marking, has to be studied in more depth.

In all three cases, a historical study of the contact-situation between the relevant languages would contribute much to the understanding of the phenomenon of marked"=S. 
Historical data might also give a better understanding of the origin of the marked"=S pattern altogether. 
Different explanations for the origin of this coding-system have been discussed in Section~\ref{explain}. 
While for some areas, an origin within the discourse structure of a language appears to be plausible, this source appears to be especially likely for the languages of the Pacific.
In other areas, namely North America, discourse structure does not seem to have any impact on the marked"=S systems of the languages.
This observation hints at the possibility that the phenomenon of marked"=S coding has a number of different pathways that lead to this pattern.
Ultimately, the different types of marked"=S languages my study identified might well be a residue of these distinct pathways leading to the marked"=S structure.
Thus the functions covered by the overtly coded S-case (and respectively, the functions not covered by it) will likely prove to be explainable by the diachrony of the case-marker.

