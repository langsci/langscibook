\chapter{Nominal predication}\label{nompred}   
%\setcounter{exx}{0}


%%%%%%%%%%%%%%%%%%%%%%%%%
%%%%% Section 3.1 %%%%%%%
%%%%%%%%%%%%%%%%%%%%%%%%%

\section{Introduction}

In\is{nominal predication|(} nominal predications, a predication over a noun (henceforward called the \textsc{subject of the nominal predication}\is{nominal predication!subject of}) is expressed by means of another nominal element (henceforward called the \textsc{predicate nominal}\is{nominal predication!predicate nominal}) rather than by a verb. 
Since this construction consists of two nominals%\footnote{The nominal status of the predicate nominal being defined by its lexical category being noun rather than verb, which again is defined by their overwhelming usage in nominal rather than verbal functions.} 
, which can both potentially be case-marked, both functions -- the subject of the nominal predication and the predicate nominal -- are of interest for this study. %more on reference?

The Wappo\il{Wappo} example in (\ref{NomPredEx}) demonstrates the general pattern of nominal predications. 
The subject of the nominal predication is the noun phrase \emph{ce k'ew} `that man', while the second noun phrase \emph{i ek'a} `my son' is the predicate nominal. 
Note that, unlike in other transitive or intransitive clauses in Wappo\il{Wappo}, the subject does not receive Nominative case-marking, and neither does the predicate nominal receive any overt marking.


\begin{exe}\ex \label{NomPredEx}\langinfobreak{Wappo}{Wappo-Yukian; California}{\citealp[12]{Thompsonetal:2006}}
\gll ce k'ew ce{\textglotstop}e{\textglotstop} i ek'a\\
     \dem{} man \cop{} 1\sg{} son\\
\glt	`That man is my son.'
\end{exe} 

Many languages employ additional grammatical means in nominal predications such as copulas\is{copula}. 
This is, for instance, the case in the Wappo\il{Wappo} example above. 
However, no matter whether a language employs a copula in this context or not, the predicate nominal\is{nominal predication!predicate nominal} functions as the predicator and not the copula. 
\citet[28--29]{Hengeveld:1992} demonstrates that for all non-verbal predications (of which nominal predications are a subgroup), selectional restrictions on the arguments of a predicate are due to the meaning of the predicate and independent of any copula element.\is{nominal predication|)} 

\citet[62--100]{Stassen:1997}\is{copula!absence versus presence|(} makes a number of observations about the distribution of zero and overt copulas in the languages of the world, or in his terms `zero strategies' and `full strategies'. 
The usage of zero-copulas in nominal predications has by far the widest distribution among the types of non-verbal intransitive predication and in fact is a prerequisite for the zero strategy to be used with other non-verbal predication types. 
Furthermore, \citet[65]{Stassen:1997} notes that the zero strategy is most commonly found with third persons -- a subset of which are full noun phrases, on which this study centers. 
For some languages of my sample the question whether there is a copula in nominal predications or not is crucial since nominal case-marking is different in the two constructions. 
I will address this issue in greater detail in Section~\ref{NomPredData}, in which the research questions on nominal predications for this study are outlined. 
However, the absence or presence of a copula in a given language or context is not the only noteworthy property.
Copulas have quite different properties cross-linguistically, ranging from more verb-like (taking regular verbal inflections etc.) to less verb-like (mere particles, which do not behave like other verbs of the relevant languages). 
However, these differences will not be taken into consideration in this study \citep[for a detailed study of the category copula across languages, see][]{Pustet:2003}.\is{copula!absence versus presence|)} 
 
In the discussion of nominal predications a distinction is often made between `identity'\is{nominal predication!identity} and `class-membership\is{nominal predication!class-membership}' predications \citep[100]{Stassen:1997}. 
Since in almost all languages of my sample, the formal encoding does not differ in the two types of nominal predication, both types will be discussed in parallel in this chapter. 
The distinction between these two types of nominal predication will be explicitly discussed in Section~\ref{Identity}.
In that section, the data from Tennet\il{Tennet} (Nilotic) -- the only example I am aware of of a marked"=S language with different constructions for encoding identity and class-membership -- will be presented in greater detail.  

As I noted before, both the subject and the predicate nominal are of interest for this study due to their nominal nature and the resulting potential for case-marking. 
For the predicate nominal\is{nominal predication!predicate nominal}, however, there might be some uncertainty with regard to the part of speech it functions as in this construction. 
It is possible for the predicate nominal\is{nominal predication!predicate nominal} to have verb-like encoding -- \citet{Stassen:1997} calls such cases `verbal takeover' of class-membership\is{nominal predication!class-membership} predicates. 
If the predicate nominal shows morphological marking used exclusively on verbs in that language otherwise, I will consider it to function as a verb rather than a noun in this construction. 
Thus the absence of case-marking on a lexical noun clearly showing exclusively verbal marking in nominal predications will not be considered an instance of zero-coding but as `not applicable'. 
In contrast, zero-coded predicate nominals in a language that does not require any inflection on the verb could just as well be treated as verbs as as nouns. 
In cases in which there is no evidence for or against a nominal status of predicate `nominals' I will consider predicate nominals\is{nominal predication!predicate nominal} as  belonging to the nominal rather than the verbal category. 

In the following section (\ref{CaseNomPred}), I will review the (rather sparse) literature on case-marking in nominal predications.
Afterwards, the distinction between the two semantic types of nominal predication -- class-membership and identity predication -- is discussed (Section~\ref{Identity}). 
In Section~\ref{NomPredData}, I will identify four patterns of case-marking found with nominal predication, as well as outline further research questions of the present study. 
The subsequent sections demonstrate the presence/absence of these four patterns in the marked"=S languages of North America (Section~\ref{NomPredNA}), the Afro-Asiatic (Section~\ref{NomPredAfro}), and Nilo-Saharan (Section~\ref{NomPredNilo}) phyla and the languages of the Pacific region (Section~\ref{NomPredPac}). 
Finally, a summary of the data discussed in Sections~\ref{NomPredNA}--\ref{NomPredPac} will be given in Section~\ref{NomPredOverview}. 

%%%%%%%%%%%%%%%%%%%%%%%%%
%%%%% Section 3.2 %%%%%%%
%%%%%%%%%%%%%%%%%%%%%%%%%

\section{Case-marking in nominal predication}\label{CaseNomPred}

Case-marking\is{nominal predication!predicate nominal|(} is not a prominent topic in the literature on nominal predication. 
\citet[111]{Payne:1997}, for example, in his chapter on predicate nominals, discusses various strategies of encoding with respect to the presence/absence or type of the copula, but does not mention case-marking at all. 
The literature that discusses case-marking in nominal predications is largely concerned with the case of the predicate nominal.\is{nominal predication!predicate nominal|)} 
On the subject of nominal predications\is{nominal predication!subject of}, most authors seem to assume that the same mechanisms apply as to subjects elsewhere. 
One exception to this general tendency is \citet[162,~165--168]{Dixon:2010-2}, who treats subjects of nominal predication -- his `copula subjects' (CS) and `verbless clause subjects' (VCS) -- as a distinct category (more accurately two distinct categories) from transitive and intransitive subjects. 
He notes that in individual languages CS and VCS can have different syntactic properties than the other types of subjects, among these properties being case-marking.
The data in (\ref{Yuman})\il{Diegue\~no (Mesa Grande)} exemplify a language which uses different case-marking for subjects in nominal predication than in basic (in)transitive clauses. 


\begin{exe}\ex\label{Yuman}\langinfobreak{Mesa Grande Diegue{\~n}o}{Yuman; California}{\citealp[15]{Gorbet:1976}}
\gll ixpa-pu a:sa:\textbf{-c} yis\\
     eagle-\dem{} bird-\nom{} is\_indeed\\
\glt `The eagle is a bird'
\end{exe}

\citet{Comrie:1997}\is{nominal predication!predicate nominal|(} proposes two possible accounts for case-assignment to nominal predicates (under which he also subsumes predicative adjectives): case-assign\-ment through government by the verb and case-assignment through agreement with the subject of the nominal predication.  
He argues that both possibilities are attested in the languages of the world. 
Hence, the mechanism of case-assignment to the predicate nominal -- either through government or agreement -- is a typological variable that languages vary with respect to. 
For languages in which the subject and predicate nominal do not match in case-marking only the government hypothesis is plausible. 
If, however,  both nominals have the same case, both analyses could potentially account for the observed behavior. 
To test which analysis is correct, one needs detailed data on nominal predications in that language. 
Also, the language must allow subjects to have a non-uniform case-marking in the first place, otherwise there would not be any observable difference between the two hypotheses. 
Most marked"=S languages of my sample use different case-forms for the subject and predicate nominal, hence the agreement hypothesis would not work for them. 
Of the remaining languages, there are not enough data on nominal predications to decide which account works best to explain the case-assignment to predicate nominals. 

A more formal approach dealing with case-assignment to predicate nominals is provided by \citet[243--246]{Yip:1987}.
In their approach, case is represented on a tier separate from phrase structure; case-assignment to individual NPs happens through association of the two tiers (unless case is lexically assigned through the verb). 
\citeauthor{Yip:1987} give two possible accounts for languages in which the predicate case agrees with the subject-case (their example language being Icelandic). 
In one account, the case assigned to the subject spreads to the nominal predicate; \citeauthor{Yip:1987} compare this process to the phonological principle of `Geminate Integrity', and state that this is implemented in the lexicon through a joint linking of the two nominals.  
In the second account, the nominal predicate receives its case from copying the case of the subject, with which it is co-indexed. 
In this approach the predicate nominal is assigned a special case -- called `predicative' by \citet{Yip:1987} --  through the lexical entry of the verb `to be' (i.e. the copula). 
This predicative case has the property of copying the case of the co-indexed argument.  
Since \citet{Yip:1987} only model the data from Icelandic, in which the subject and predicate of nominal predications agree in case, no implementation is proposed for languages that use different cases for the two roles. 
The second approach appears to be more promising for implementing such languages, since one would simply have to change the lexical case-assignment to the nominal predicate from `predicative' to the respective case found on predicate nominals in a language.

Finally, \citet[84]{Fillmore:1969} -- in his seminal paper on the semantic roles in language (referred to as `case roles' by him) -- makes some reference to nominal predications. 
He states that ``they represent a distinct type from those involving any of the case relations discussed above, though more than one case relation may be provided in these sentences.'' 
He ponders introducing the terms `essive' and `translative' for the type of case relations introduced in sentences of this type. 
Still, he views the requirement of number agreement between subject and predicate nominal as an issue that lacks implementation in an approach that simply introduces a new case-label for the nominal predicate.\is{nominal predication!predicate nominal|)} 
%However, he also states that the predicate nominal should probably be treated as a verbal element (V in his terminology)

%%%%%%%%%%%%%%%%%%%%%%%%%
%%%%% Section 3.3 %%%%%%%
%%%%%%%%%%%%%%%%%%%%%%%%%
\section{Identity predication}\label{Identity}\is{nominal predication!identity|(} 

So far, I have discussed nominal predication defined as a clause containing two nominal elements, one serving as the subject and the other as the predicate of the construction. 
The distinction between identity predication and class"=membership predication\is{nominal predication!class"=membership} has been glossed over.\footnote{The terminology of  `class-membership' vs. `identity' is taken from \citet{Stassen:1997}. 
Other terms used for the same distinction are `predicational' vs. `equative copula clauses' \citep{Adger:2003} or `predicational' and `referring predicate nominals' \citep{Doron:1988}.}
The two types of nominal predication differ with respect to the semantic type of their predicate nominals\is{nominal predication!predicate nominal}. 
If the predicate nominal uniquely identifies an individual, then the predication is of the identity type. 
This type of nominal predication is illustrated in (\ref{Ident}). 
Otherwise the predication is one of class-membership. 
In that case, the predicate nominal identifies a certain class of which the subject is a member as in (\ref{Class}).\il{English}

\begin{exe}\ex\label{Ident}
\begin{xlist}
\ex \textit{That man is her husband.}
\ex \textit{The morning star is the evening star.}
\end{xlist}
\end{exe}

\begin{exe}\ex\label{Class} 
\begin{xlist}\ex \textit{She is a teacher.}
\ex \textit{Whales are mammals. }
\end{xlist}
\end{exe}

From a semantic perspective, this distinction is crucial, as \citet{Doron:1988} argues. 
For English\il{English} (and to some extent also for French\il{French}), she suggests that this semantic distinction also has syntactic relevance, putting forward a number of tests to distinguish between the two types of predicate nominal constructions. 
\citet{Adger:2003} claim that there is no structural distinction between the two types of clauses. 
They support their claim with data from Scottish Gaelic and argue that the two types of clauses are identical in their syntactic representation.

\citet{Stassen:1997} distinguishes between identity and class-membership, yet he claims that the strategy of encoding identity is very frequently extended to class-membership. His `principle of identity pressure' states that whenever predicate nominals are encoded by a strategy different from all other types of intransitive predication, the strategy will be taken over from the encoding of identity predication \citep[111]{Stassen:1997}. 
The overlap of the encoding strategies of identity and class-membership predication is also revealed in my sample of marked"=S languages. 
Only in one language of my sample, Tennet\il{Tennet}, the two types of predication are encoded by different constructions. 

In Tennet\il{Tennet}, the subject is in the Accusative\is{case!individual forms!accusative} form rather than the Nominative for sentences interpreted as identity predications. 
This is irrespective of whether the clause contains an overt copula (\ref{TenEqPred}a) or not (\ref{TenEqPred}b).\footnote{In some marked"=S languages of my sample, case-marking depends on whether or not a clause has an overt copula (see Section~\ref{NomPredData}).} 
In class-membership predications the subject is in the Nominative, as is illustrated in (\ref{TenNomPredRep}).   

\begin{exe}\ex\label{TenEqPred}\langinfo{Tennet}{Surmic, Nilo-Saharan; Sudan}{\citealp[234, 233]{Randal:1998}}
\begin{xlist}
\ex\gll \textbf{\textipa{\=*an\'{\=*e}t}} \textipa{c\'I} \textipa{k-\=*e\'{\=*e}n\'{\=*I}} \textipa{d\=*em\'{\=*e}z-z\'{\=*o}h-t}\\
1\sg{}.\acc{} \am{} 1-be teach-\agnm{}-\sg{}\\
\glt `I'm the teacher.'
\ex\gll \textbf{\textipa{\=*an\=*et}} \textipa{m\'{\=*o}t-t\'{\=*o}h-t}\\
1\sg{}.\acc{} be.angry-\agnm{}-\sg{}\\
\glt `I am the brave man.'
\ex\label{TenNomPredRep}\gll\textipa{k-\=*e\'{\=*e}n\'{\=*I}} \textipa{\textbf{ann\'a}} \textipa{d\=*em\'{\=*e}z-z\'{\=*o}h-t}\\
1-be 1\sg{}.\nom{} teach-\agnm{}-\sg{}\\
\glt `I am a teacher.'
\end{xlist}
\end{exe}\is{nominal predication!identity|)}

     

%%%%%%%%%%%%%%%%%%%%%%%%%
%%%%% Section 3.4 %%%%%%%
%%%%%%%%%%%%%%%%%%%%%%%%%

\section{Research questions}\label{NomPredData}

The following examples (\ref{Maidu}--\ref{Wappo}) demonstrate the variability of case-marking with subjects and predicates\is{nominal predication!predicate nominal} of nominal predication in marked"=S languages. 
Maidu\il{Maidu} (\ref{Maidu}) marks both nominals with the overt Nominative case-suffix \emph{-m}.
In Savosavo\il{Savosavo} (\ref{Savosavo}) only the subject of a nominal predication is marked with the Nominative case (\emph{=na}) while the predicate nominal is zero-coded.
Conversely, Mesa Grande Diegue{\~n}o\il{Diegue\~no (Mesa Grande)} marks the predicate nominal with the Nominative case-suffix \emph{-c} while the subject remains zero-coded. 
Example (\ref{Yuman}) from above is repeated as (\ref{MesaGrande}).
Finally, in Wappo\il{Wappo} (\ref{Wappo}) both the subject and predicate nominal are zero-coded, as was already seen in (\ref{NomPredEx}) that is repeated here.       

%\pagebreak

\begin{exe}\ex\label{Maidu}\langinfobreak{Maidu}{Maiduan, Penutian; California}{\citealp[30]{Shipley:1964}}
\gll mym kyl\'okbe\textbf{-m} ma-\'k\'ade m\'in-kot\`o\textbf{-m}\\
     this old\_woman-\nom{} be-\question{} 2-grandmother-\nom{}\\
 \glt    `Is that old woman your grandmother?'
\end{exe}


\begin{exe}\ex\label{Savosavo}\langinfobreak{Savosavo}{Solomons East Papuan; Solomon Islands}{\citealp[128]{Wegener:2008}}
\raggedright\gll zu lo {gola kiba} sisi\textbf{=na} te lo ulunga lo-va taghata\\
and \deter{}.\sg{}.\mas{} green orn\_flower=\nom{} \emphat{} \deter{}.\sg{}.\mas{} pillow 3\sg{}.\mas{}-\gen{}.\mas{} on\_top\\
\glt `and the green flower (is) on top of the pillow' (lit.: 'and the green flower (is) the pillow its top')
\end{exe}


\begin{exe}\ex\label{MesaGrande}\langinfobreak{Mesa Grande Diegue{\~n}o}{Yuman, Hokan; California}{\citealp[15]{Gorbet:1976}}
\gll ixpa-pu a:sa:\textbf{-c} yis\\
		eagle-\dem{} bird-\nom{} is\_indeed\\
	\glt	`The eagle is a bird'
\end{exe}

\begin{exe}\ex \label{Wappo}\langinfobreak{Wappo}{Wappo-Yukian; California}{\citealp[12]{Thompsonetal:2006}}
\gll ce k'ew ce{\textglotstop}e{\textglotstop} i ek'a\\
     \dem{} man \cop{} 1.\sg{} son\\
	\glt	`That man is my son.'
\end{exe}                                                                                                  

\noindent These are all four logical possibilities of case-marking that can be derived from a set of two case-forms -- S-case and zero-case -- and two roles -- subject and nominal predicate.\footnote{If one includes the possibility of additional case-forms, the number of possible patterns is multiplied. 
The vast majority  of languages of my sample does, however, restrict the case-forms used in this context to the S- and zero-case. 
Arbore\il{Arbore}, the only exception, will be discussed in Section~\ref{NomPredAfro}.}
However, these four patterns are quite unevenly distributed within the languages of my sample, as will become apparent in Sections~\ref{NomPredNA}--\ref{NomPredPac}, in which the nominal predications in the marked"=S languages of my sample will be presented in greater depth.

After introducing the four patterns of case-marking found in nominal predications, I will now turn to the other element often present in the constructions encoding this context: the copula. 
With\is{copula!absence versus presence|(} respect to the occurrence of copula elements in nominal predications, the languages of my sample also exhibit a number of distinct patterns. 
Some languages do not have a copula element, while others must have a copula present in this context, yet again other languages exhibit variation between presence and absence of the copula in nominal predications. 
For the last type of language -- those languages in which a copula can be either present or absent -- finer distinction can be made.
First, some languages seem to have free variation between the two constructions while other languages behave in a more systematic fashion. 
The systematic languages employ copula elements in certain contexts, usually in clauses that are negated or non-present tense. 
The copula in these contexts serves as a means to mark tense or negation, a pattern well known from many languages of the world, not just marked"=S languages \citep[119]{Payne:1997}. Another distinction addresses the case-marking of the nominals in the relevant construction(s).
In some of the languages that exhibit variation between presence and absence of the copula (either free or systematic), this distinction correlates with a difference in case-marking. While the overt S-case is found in the constructions with having an overt copula, this case-marking is absent in the construction lacking the copula. 
For the languages in which some of the described variation is found, this will be addressed in more detail in the following sections discussing the data.\is{copula!absence versus presence|)}  
The data are subdivided by macro-area and genealogical affiliation. 
The latter classification is only applied to the African languages since for the other areas the number of languages is rather small and most genealogical units have only one member in the sample. 
Furthermore, the data are organized by the four patterns of case-marking for nominal predications introduced above, repeated here for convenience: 

\begin{itemize}
\item overt marking of both nominals 
\item overt marking of only the predicate nominal
\item overt marking of only the subject of nominal predication
\item no overt marking on either nominals  
\end{itemize}

%%%%%%%%%%%%%%%%%%%%%%%%%
%%%%% Section 3.5 %%%%%%%
%%%%%%%%%%%%%%%%%%%%%%%%%

\section{North America}\label{NomPredNA}

The North American languages of my sample are all located near the Pacific Coast in an area reaching from Northern California to Mexico and stretching inland as far as Arizona. 
Among these languages, the remarkable pattern exemplified in (\ref{MesaGrande}) above is found, in which the nominal predicate\is{nominal predication!predicate nominal} is marked with Nominative case and the subject\is{nominal predication!subject of} is zero-coded. 
This pattern appears exclusively in the Yuman genus. 
It is predominant in the Yuman languages, but does not appear to be attested in any other language worldwide. 
However, as we will see below, some Yuman languages employ some of the other patterns under certain conditions.

As in Diegue\~no\il{Diegue\~no (Mesa Grande)} (cf.~\ref{MesaGrande}), in Mojave\il{Mojave} the Nominative suffix \emph{-\v c} is attached to the predicate nominal and -- unlike in other clauses -- not to the subject, which in turn remains zero-coded (cf.~\ref{MojNomPred}).

\begin{exe}\ex\label{MojNomPred}\langinfobreak{Mojave}{Yuman; California}{\citealp[49]{Munro:1976}}
 \gll \textipa{\textglotstop in\super yep} \textipa{\textglotstop-i\v cuy-n\super y} \textipa{k\super waT@\textglotstop ide:\textbf{-\v c}} \textipa{ido-p\v c}\\
1\sg{}.\poss{} 1-husband-\dem{} doctor-\nom{} be-\tns{}\\
\glt `My husband is a doctor.' 
\end{exe}

Comparable structures can be found in most other Yuman languages. This is illustrated by the examples from Maricopa\il{Maricopa} (\ref{MariNomPred}), Yavapai\il{Yavapai} (\ref{YavNomPred}) and Walapai\il{Walapai} (\ref{WalNomPred})\footnote{Since the two Walapai examples come from different sources, the orthographies and levels of phonetic detail represented differ between the examples.} below.

\begin{exe}
\ex\label{MariNomPred}\langinfo{Maricopa}{Yuman; Arizona}{\citealp[38]{Gordon:1986}}
\begin{xlist}\ex\gll mmdii-ny-a chyer\textbf{-sh} duu-m\\
owl-\dem{}-\augv{} bird-\nom{} be-\rls{}\\
\glt `Owls are birds/ The owl is a bird.' 
\ex\gll 'iipaa-ny-a kwsede\textbf{-sh} (duu-m)\\
man-\dem{}-\augv{} doctor-\nom{} be-\rls{}\\
\glt `The man is a doctor.'
\end{xlist}
\end{exe}

\begin{exe}\ex\label{YavNomPred}\langinfobreak{Yavapai}{Yuman; Arizona}{\citealp[66]{Kendall:1976}}
\gll \textipa{can} \textipa{\textglotstop-\~n-pa: \textglotstop ichwa:-v\textbf{-c}} \textipa{yu-m}\\
John 1-\poss{}-enemy-\dem{}-\nom{} be-\faff{}\\
\glt `John is my enemy.' \end{exe}

\begin{exe}\ex\label{WalNomPred}\langinfo{Walapai}{Yuman; Arizona}{\citealp[160]{Redden:1966}; \citealp[480]{Watahomigie:2001}}
\begin{xlist}\ex 
\gll \textipa{\textltailn \'a} \textipa{ap\`a-v\textbf{-\v c}} \textipa{y\'u}\\
1\sg{}.\acc{} human-\refl{}-\nom{} be\\
\glt `I am a human being.'
\ex \label{Hualapai.NomPred}
\gll nya bo\'s-v-\textbf{\v c} yu\\
1\sg{}.\acc{} cat-\refl{}-\nom{}  be\\
\glt `I am a cat.'
\end{xlist}
\end{exe}

In the closely-related language Havasupai\il{Havasupai}, this pattern is also found for encoding nominal predications. 
As is demonstrated in (\ref{HavaNomPredCop}), in this construction the noun phrase referring to the predicate nominal is marked with the Nominative suffix \emph{-c}, while the subject remains zero-coded. 
However, this is not the only possibility for encoding nominal predications in that language.  
In (\ref{HavaPredAlt}), another possible construction is illustrated. 
If\is{copula!absence versus presence|(} a sentence expressing nominal predication does not contain an overt copula, a zero-coded predicate nominal -- as well as a zero-coded subject -- is found according to \citet{Kozlowski:1972}. 
However, there is no general correlation in the Yuman language family of zero-coding of predicate nominals with copula-less sentences. 
Other Yuman languages do mark the predicate nominal with overt Nominative case, even in sentences that lack a copula.  
% Maybe some better phrasing or a footnote concerning the referntiallity of predicate nominals. 


\begin{exe}\ex\label{HavNomPred}\langinfo{Havasupai}{Yuman; Arizona}{\citealp[35, 33]{Kozlowski:1972}}
\begin{xlist}
\ex\label{HavaNomPredCop}\gll jan \~na-\~nuwa ha\textbf{-c} yu\\
John 1-friend \dem{}-\nom{} be\\
\glt `John is my friend.' 
\ex\label{HavaPredAlt}\gll jan \~na-\~nuwa-ha\\
John 1-friend-\dem{}\\
\glt `John is my friend.'
\end{xlist}
\end{exe}\is{copula!absence versus presence|)}

% Reference lost: As the previous example shows, the copula can be omitted without changing the case-marking of the nominals. 

While in Havasupai\il{Havasupai} the example in (\ref{HavaPredAlt}) exemplifies an alternative construction, in Jamul\il{Jamul Tiipay} Tiipay it is the only possibility for expressing  nominal predications. 
As can be seen in (\ref{JamNomPred}), both the subject of nominal predications and the predicate nominal\is{nominal predication!predicate nominal} are zero-coded. 
 However, there is another construction in the language consisting of two nominals in which the subject is in the Nominative case (\ref{JamCopCon}). 
 \citet[184--185]{Miller:2001} explicitly distinguishes this construction from nominal predication. 
She calls it the `copula construction'\is{copula!absence versus presence}, since, unlike the regular nominal predication in Jamul\il{Jamul Tiipay} Tiipay, it contains the verb `to be'. 
In this copula construction, the subject is in the Nominative case and the other noun is zero-coded. 
At least in some cases, there seems to be a difference in meaning between the copula construction, as in (\ref{JamCopCon}), and the regular nominal predication, as in (\ref{JamNomPred}). 
While the (a) example clearly makes a statement about class-membership, the (b) example does not.   
 %Explain
 
%\pagebreak

\begin{exe}\ex\langinfo{Jamul Tiipay}{Yuman; California}{\citealp[181, 185]{Miller:2001}}
\begin{xlist}
\ex\label{JamNomPred}\gll nyech'ak-pu metiipay\\
woman-\dem{} indian\\
\glt `That woman is an Indian.'
\ex\label{JamCopCon}\gll nyech'ak-pe\textbf{-ch} metiipay we-yu\\
woman-\dem{}-\nom{} indian 3-be\\
\glt `That woman is playing Indian/pretending to be an Indian.'
\end{xlist}
\end{exe}

Another North American language with zero-coded subject and predicate\is{nominal predication!predicate nominal} in nominal predications is Wappo\il{Wappo}, as was already noted in the previous section.
This pattern has already been illustrated in example (\ref{Wappo}) above with two full noun phrases.
(\ref{WapPredNom}) exemplifies an instance of class-membership predication with a pronominal subject.

%\pagebreak

\begin{exe}\ex\label{WapPredNom}\langinfobreak{Wappo}{}{\citealp[43]{Thompsonetal:2006}}
\gll i ce{\textglotstop}e{\textglotstop} yomto\textglotstop\\
  1\sg{}.\acc{} \cop{} doctor\\
\glt `I am a doctor.'
\end{exe}

Only one of the North American marked"=S languages marks both the subject of nominal predication and the predicate nominal\is{nominal predication!predicate nominal} with the overt Nominative case-marker. 
This language is Maidu\il{Maidu}. Although this pattern is the one that is most familiar from nominative"=accusative languages of the standard type, for marked"= S languages it seems to be an exceptional pattern. 
As demonstrated in (\ref{MaiNomPred}), both subject and predicate nominal are marked with the Nominative suffix \emph{-m} in Maidu\il{Maidu} nominal predications.
In general, Maidu\il{Maidu} employs the overtly coded Nominative more than one would expect from a standard nominative"=accusative language, that is, in a wide variety of contexts. 
It thus behaves counter to the expectation of \citet{Koenig:2008} that in a marked"=nominative language the zero-coded accusative\is{case!individual forms!accusative} will have a wider range of functions than the overtly coded nominative. 


\begin{exe}\ex\label{MaiNomPred}\langinfobreak{Maidu}{Maiduan; California}{\citealp[30]{Shipley:1964}}
 \gll my-m kyle\textbf{-m} ka-k'an nik-po\textbf{-m}\\
\dem{}-\nom{} woman-\nom{} be-3 1\sg{}.\poss{}-daughter-\nom{}\\
\glt `That woman is my daughter.'
\end{exe}

In the Yuman languages, a reinterpretation of the nominal predication construction appears to be ongoing. 
In some instances, the subject of a nominal predication receives Nominative case-marking as well, as \citet[39--40]{Gordon:1986} demonstrates for Maricopa\il{Maricopa}. 
According to \citet[469--471]{Munro:1977}, this tendency can be observed in other Yuman languages as well. 

\begin{exe}\ex\langinfobreak{Maricopa}{Yuman; Arizona}{\citealp[40]{Gordon:1986}}
\gll '-ny-kwr'ak\textbf{-sh} pakyer\textbf{-sh} duu-m\\
1-\poss{}-old\_man-\nom{} cowboy-\nom{} be-\rls{}\\
\glt `My husband is a cowboy.'
\end{exe}

The data provided in this section are summarized in Table~\ref{OverviewNomPredNA}. 
The table provides an overview of the case-marking in nominal predications in the marked"=S languages of North America. 
Maidu\il{Maidu} is exceptional -- not only for this region -- in marking both nominals with overt Nominative case. 
Wappo\il{Wappo}, on the other hand, has both nominals zero-coded, a pattern that is also found as one possible pattern in the Yuman languages Havasupai\il{Havasupai} and Tiipay, where it is the most common pattern. 
The other Yuman languages have the remarkable pattern of using nominative case on the predicate nominal and zero-coding the subject in this context. 
This pattern is also found for Havasupai\il{Havasupai} in clauses with an overt copula). 
\begin{table}[ht]
\centering
\begin{tabular}{lcc}
\hline \hline
\bfseries language&\bfseries subject&\bfseries predicate nominal\\
\hline
Diegue\~no\il{Diegue\~no (Mesa Grande)} (Mesa Grande) &\acc{}&\textbf{\nom{}}\\
%\hdashline
Havasupai\il{Havasupai}&\acc{}&\textbf{\nom{}}/\acc{}\\
%\hdashline
Jamul\il{Jamul Tiipay} Tiipay&\acc{}/\textbf{\nom{}}&\acc{}\\
%\hdashline
Maricopa\il{Maricopa}&\acc{}&\textbf{\nom{}}\\
%\hdashline
Mojave\il{Mojave}&\acc{}&\textbf{\nom{}}\\
%\hdashline
Walapai\il{Walapai}&\acc{}&\textbf{\nom{}}\\
%\hdashline
Yavapai\il{Yavapai}&\acc{}&\textbf{\nom{}}\\
%\hdashline
Maidu\il{Maidu}&\textbf{\nom{}}&\textbf{\nom{}}\\
%\hdashline
Wappo\il{Wappo}&\acc{}&\acc{}\\
\hline \hline
\end{tabular}
\caption{Marking of nominal predication in the marked"=S languages of North America}\label{OverviewNomPredNA}%\\
\end{table}

%%%%%%%%%%%%%%%%%%%%%%%%%
%%%%% Section 3.6 %%%%%%%
%%%%%%%%%%%%%%%%%%%%%%%%%

\section{Afro-Asiatic}\label{NomPredAfro}

The predominant pattern in the Afro-Asiatic marked"=S languages is to have overt nominative case-marking on the subject of the nominal predication and zero-coding on the predicate nominal\is{nominal predication!predicate nominal}. 
This pattern can be found in numerous languages of the Eastern Cushitic and Omotic genera.
Yet in some cases the predicate nominal does receive overt case-marking, which does not necessarily have to be the nominative case. 

In Boraana\il{Oromo (Boraana)} Oromo, the Nominative suffix \emph{-ii} marks the subject of the nominal predication in (\ref{BorNomPred}), while the predicate nominal\is{nominal predication!predicate nominal} -- \emph{obboleesa kiya} -- remains zero-coded. 
In the closely-related Harar\il{Oromo (Harar)} variety of Oromo, a parallel structure is used, as
shown in (\ref{HarNomPred}). 

%\pagebreak

\begin{exe}\ex\label{BorNomPred}\langinfobreak{Oromo (Boraana)}{Eastern Cushitic; Kenya}{\citealp[34]{Stroomer:1995}}
\gll mamic\textbf{-ii} kuninii obboleesa kiya\\
man-\nom{} \dem{}.\nom{} brother 1\sg{}.\poss{}\\
\glt `This man is my brother.' \end{exe}

\begin{exe}\ex\label{HarNomPred}\langinfo{Oromo (Harar)}{Eastern Cushitic; Ethiopia}{\citealp[100]{Owens:1985}}
\begin{xlist}\ex\gll mak'\'aa\textbf{-n} axaax\'uu xiyy\'a \'alii\\
name-\nom{} grandfather my Ali\\
\glt `My grandfather's name is Ali.'
\ex \gll \textbf{inn\'\i i} angafa xiyy\'aa-mihi\\
he.\nom{} {elder\_brother} my-\Neg{}\\
\glt `He is not my elder brother.'
\end{xlist}
\end{exe}

This typical Afro-Asiatic pattern of marking nominal predication with the subject in Nominative case  and the predicate nominal\is{nominal predication!predicate nominal} in the accusative\is{case!individual forms!accusative} is also found in Gamo\il{Gamo} (\ref{GamNomPred}), K'abeena\il{K'abeena} (\ref{KabNomPred}), and Zayse\il{Zayse} (\ref{ZayNomPred})\footnote{It might appear a bit puzzling at first glance that the `zero-coded' predicate nominal has extra material following the noun stem. 
\citet[280--281]{Hayward:1990} gives the following description of the copular\is{copula!absence versus presence} element popping up in this construction: ``[t]he copula attaches to a phrase (NP or PP) which is focused''. 
He also demonstrates the use of the copula in cleft-like constructions with focused subjects, objects, temporal nouns, and prepositional phrases.}

%\enlargethispage{\baselineskip}

\begin{exe}\ex\label{GamNomPred}\langinfobreak{Gamo}{Omotic; Ethiopia}{\citealp[370]{Hompo:1990}}
\gll \textbf{\v{C}'aboi} lo\textglotstop o asi d-$\emptyset$-$\emptyset$-enna\\
Chabo.\nom{} good man.\acc{}2 be-\persm-\tns{}-\Neg{}\\
\glt `Chabo is not a good man.'
\end{exe}  

\begin{exe}\ex\label{KabNomPred}\langinfobreak{K'abeena}{Eastern Cushitic; Ethiopia}{\citealp[264]{Crass:2005}}
\gll ku \textbf{manc\textsuperscript{u}} moggaancoh\textsuperscript{a}\\
\dist{}.\mas{} man.\nom{} thief.\acc{}.\cop{}.\mas{}\\
\glt `This man is a thief.'%\\ original translation:
\end{exe}

\begin{exe}\ex\label{ZayNomPred}\langinfobreak{Zayse}{Omotic; Ethiopia}{\citealp[280]{Hayward:1990}}
\gll \textglotstop e-\textbf{\textglotstop a\`s{\'\i}} w\'oota\`s'-\'u-tte\\
\deter{}.\mas{}-man.\nom{} farmer-\epen{}-\cop{}\\
\glt `The man is a farmer,'
\end{exe} 

Wolaytta\il{Wolaytta} has a construction parallel to the Afro-Asiatic languages discussed so far (\ref{WolNomPred}). 
However, it is also possible to mark the predicate nominal\is{nominal predication!predicate nominal} with Nominative case, as there is no difference in meaning between examples (\ref{WolNomPredAlt1}) and (\ref{WolNomPredAlt2}). 
According to \citet{Lamberti:1997}, this alternation is especially common with feminine nouns.

\begin{exe}\ex\langinfo{Wolaytta}{Omotic; Ethiopia}{\citealp[225]{Lamberti:1997}}
\begin{xlist}
\ex\label{WolNomPred}
\gll he bitann\textbf{-ey} laagge\\
that man-\nom{} friend.\acc{}\\
\glt `That man is a friend.'
\ex\label{WolNomPredAlt1}\gll ha-nna gelawi\textbf{-ya}\\
this-\fem{} girl-\nom{}\\
\glt `This is a girl.'
\ex\label{WolNomPredAlt2}\gll ha-nna gelawi-y\textsuperscript{u}\\
this-\fem{} girl-\acc{}\\
\glt `This is a girl.'
\end{xlist}
\end{exe}

Finally\is{case!individual forms!predicative|(}, Arbore\il{Arbore} has a dedicated case-form for encoding predicate nominals\is{nominal predication!predicate nominal}, the so-called `Predicative' case (\ref{ArbNomPred}). 
The subject of nominal predications, as in the other Afro-Asiatic languages, is in the Nominative case.

\begin{exe}\ex\label{ArbNomPred}\langinfobreak{Arbore}{Eastern Cushitic; Ethiopia}{\citealp[136]{Hayward:1984}}
\gll \textbf{\textipa{m\'o}} \textipa{bal} \textipa{\textglotstop iyya-H-aw\textbf{-a}}\\
man.\nom{} was father-\mas{}-1\sg{}.\poss{}-\pred{}\\
\glt `The man was my father.'
\end{exe}\is{case!individual forms!predicative|)}

All data from the Afro-Asiatic marked"=nominative languages are summarized in Table~\vref{OverviewNomPredAfro}.
\begin{table}[ht]
\centering
\begin{tabular}{lcc}
\hline \hline
\bfseries language&\bfseries subject&\bfseries predicate nominal\\
\hline
Arbore\il{Arbore}&\textbf{\nom{}}&\textbf{\pred}\\
%\hdashline
Gamo\il{Gamo}&\textbf{\nom{}}&\acc{}\\
%\hdashline
K'abeena\il{K'abeena}&\textbf{\nom{}}&\acc{}\\
%\hdashline
Oromo (Boraana\il{Oromo (Boraana)})&\textbf{\nom{}}&\acc{}\\
%\hdashline
Oromo (Harar\il{Oromo (Harar)})&\textbf{\nom{}}&\acc{}\\
%\hdashline
Wolaytta\il{Wolaytta}&\textbf{\nom{}}&\acc{}/\textbf{\nom{}}\\
%\hdashline
Zayse\il{Zayse}&\textbf{\nom{}}&\acc{}\\
\hline \hline
\end{tabular}
\caption{Marking of nominal predication in the Afro-Asiatic marked"=S languages}\label{OverviewNomPredAfro}%\\
\end{table}
Uniformly the subject of nominal predications is marked with the Nominative case in the Afro-Asiatic marked"=S languages.
The predicate nominal exhibits some minor variation with respect to the overt encoding. 
While most languages use the zero-coded Accusative\is{case!individual forms!accusative} form to encode this function, Wolaytta\il{Wolaytta} exhibits an alternative variant of encoding it with the Nominative case (at least for some nouns) and Arbore\il{Arbore} has a special dedicated case-form for this role.  


%%%%%%%%%%%%%%%%%%%%%%%%%
%%%%% Section 3.7 %%%%%%%
%%%%%%%%%%%%%%%%%%%%%%%%%

\section{Nilo-Saharan}\label{NomPredNilo}

Like Afro-Asiatic, the Nilo-Saharan languages prefer nominative case-marking on the subject and zero-coding of the predicate nominal\is{nominal predication!predicate nominal}. 
Similar to the situation described above for Yuman Havasupai\il{Havasupai} (\ref{HavNomPred}), there is also an interaction between presence/absence of a copula\is{copula!absence versus presence} in the nominal predication and the presence/absence of overt case-marking in one Nilo-Saharan language, namely Turkana\il{Turkana}.  

%\subsection*{S-\nom{} Pred-$\emptyset$:}
The most widespread pattern of marking nominal predications in Nilo-Saharan is to mark the subject of the construction with nominative case, while the predicate nominal remains in the zero-coded accusative\is{case!individual forms!accusative} form. 
This pattern is found in the Surmic languages Murle\il{Murle} (\ref{MurNomPred}) and Tennet\il{Tennet} (\ref{TenNomPred}). 
A parallel structure is also found in the Nilotic languages, such as Maa\il{Maa} (\ref{MaaNomPred})\footnote{Maa\il{Maa} case is marked through a variation in the tonal\is{case-marking!via tone} pattern of the noun. 
The tone pattern of the Accusative case is assigned lexically, while the tonal shape of the
Nominative is derived from the lexical tone in a regular pattern.} 
%\enlargethispage{\baselineskip}
or  Nandi\il{Nandi} (\ref{NanNomPred}), and to some extent in Turkana\il{Turkana} (\ref{TurNomPred}). 

\begin{exe}\ex\label{MurNomPred}\langinfobreak{Murle}{Surmic; Sudan}{\citealp[110]{Arensen:1982}}
 \gll \textipa{boNboNec\textbf{-i}} \textipa{kibaali}\\
pelican-\nom{} bird.\pl{}\\
\glt `The pelican is a bird' \end{exe}

%\pagebreak

\begin{exe}\ex\label{TenNomPred}\langinfobreak{Tennet}{Surmic; Sudan}{\citealp[233]{Randal:1998}}
\gll\textipa{k-\=*e\'{\=*e}n\'{\=*I}} \textipa{\textbf{ann\'a}} \textipa{d\=*em\'{\=*e}z-z\'{\=*o}h-t}\\
1-be 1\sg{}.\nom{} teach-\agnm{}-\sg{}\\
\glt `I am a teacher.' \end{exe}


\begin{exe}\ex \label{MaaNomPred}\langinfobreak{Maa}{Nilotic; Kenya}{\citealp[175]{Tucker:1955}}
\gll \'a-r\'a Sir\'onk\`a\\
			1\sg{}-be Sironka.\acc{}\\
\glt `I am Sironka.' 
\end{exe}

\begin{exe}\ex\label{NanNomPred}\langinfobreak{Nandi}{Nilotic; Kenya}{\citealp[121]{Creider:1989}}
\gll n\'a:nti-i:n-t\`et \textbf{k\'\i pe:t}\\
Nandi\il{Nandi}-\sg{}-\them{} Kibet.\nom{}\\
\glt `Kibet is a Nandi\il{Nandi}.'
\end{exe}

\begin{exe}\ex\label{TurNomPred}\langinfo{Turkana}{Nilotic; Kenya}{\citealp[75, 76]{Dimmendaal:1982}}
\begin{xlist}\ex\gll\textipa{m\`E\`ErE\`{}} \textipa{a-\textbf{y\`ON}} \textipa{E-kapIlan\`{\r*I}}\\
not \NC{}-1\sg{}.\nom{} \NC{}-witch.\acc{}\\
\glt `I am not a witch.'
\ex\gll \textipa{\`E-\`a-raI\`{}} \textipa{Nes\`I} \textipa{E-kapIla-n\`{\r*I}}\\
3-\pst{}-be he.\nom{} \NC{}-witch.\acc{}\\
\glt `He was a witch.'
\end{xlist}
\end{exe}

Apart from the predominant pattern just described, there is also another pattern in Nilo-Saharan. 
In this minor pattern, both the subject of nominal predications and the predicate nominal\is{nominal predication!predicate nominal} are in the zero-coded accusative\is{case!individual forms!accusative} form. 
This pattern occurs in Turkana\il{Turkana} when the clause lacks an overt copula\is{copula!absence versus presence} -- this is the case in all positive, non-tense-marked clauses (\ref{TurkNomPredZeroS}). 
In the related language Datooga\il{Datooga}, both nouns, the subject and predicate nominal, are also in the zero-coded Accusative\is{case!individual forms!accusative} case, even if an overt copula\is{copula!absence versus presence} appears in the construction (\ref{DatNomPred}). 
The Tennet\il{Tennet} equational predication already discussed in Section~\ref{Identity} is of the same type. 

\begin{exe}\ex\label{TurkNomPredZeroS}\langinfobreak{Turkana}{}{\citealp[75]{Dimmendaal:1982}}
\gll\textipa{a-yON\`{}} \textipa{E-kapIlan\`{\r*I}}\\
\NC{}-1\sg{}.\acc{} \NC{}-witch.\acc{}\\
\glt `I am a witch.'\end{exe}


\begin{exe}
\ex\label{DatNomPred}\langinfo{Datooga}{Nilotic; Tanzania}{\citealp[172]{Kiessling:2007}}
\begin{xlist}
\ex\gll \textipa{{s\`aaw\`a}} \textipa{{m\`aan\`aN\'ood\`Ig\`a}} \textipa{g\^Il}\\
3\pl{}.\acc{} wealthy\_people.\acc{} \dem{}\\
\glt `They are wealthy people.'
\ex\gll \textipa{{n\`I\textltailn}} \textipa{\`a[a]} \textipa{m\`ur\'an\'eed\`a} \textipa{g\^Il}\\
3\sg{}.\abs{} \cop{} hero.\acc{} \dem{}\\
\glt `He was a hero.'
\end{xlist}
\end{exe}

The data presented above are summarized in Table~\vref{OverviewNomPredNilo}. 
\begin{table}[ht]
\centering
\begin{tabular}{lcc}
\hline \hline
\bfseries language&\bfseries subject&\bfseries predicate nominal\\
\hline
Datooga\il{Datooga}&\acc{}&\acc{}\\
%\hdashline
Maa\il{Maa}&\textbf{\nom{}}&\acc{}\\
%\hdashline
Murle\il{Murle}&\textbf{\nom{}}&\acc{}\\
%\hdashline
Nandi\il{Nandi}&\textbf{\nom{}}&\acc{}\\
%\hdashline
Tennet\il{Tennet} (\emph{class-membership})&\textbf{\nom{}}&\acc{}\\
Tennet\il{Tennet} (\emph{identity})&\acc{}&\acc{}\\
%\hdashline
Turkana\il{Turkana}&\textbf{\nom{}}/\acc{}&\acc{}\\
\hline \hline
\end{tabular}
\caption{Marking of nominal predication in the Nilo-Saharan marked"=S languages}\label{OverviewNomPredNilo}%\\
\end{table}
As can be seen, all Nilo-Saharan marked"=S languages mark the predicate nominal in the zero-coded accusative\is{case!individual forms!accusative} case. 
The subject of nominal predications is treated like other S/A arguments in most languages. Only the Nilotic languages Datooga\il{Datooga} and Turkana\il{Turkana} deviate from this pattern. 
While in Datooga\il{Datooga} subjects of nominal predications are always zero-coded, Turkana\il{Turkana} has a split between nominal predications that have an overt copular element and those that lack an overt copula\is{copula!absence versus presence}. 
In the construction with the overt copula, the subject is in the Nominative\is{case!individual forms!nominative} case, while the Accusative\is{case!individual forms!accusative} case is used for subjects in the construction without an overt copula. 
Tennet\il{Tennet}, the only language of the sample with a distinct construction for identity\is{nominal predication!identity} predication, uses a construction with both nouns in the zero-coded case-form for encoding identity\is{nominal predication!identity}.

%\enlargethispage{2\baselineskip}


%%%%%%%%%%%%%%%%%%%%%%%%%
%%%%% Section 3.8 %%%%%%%
%%%%%%%%%%%%%%%%%%%%%%%%%

\section{Pacific}\label{NomPredPac}

The marked"=S languages of the Pacific (Savosavo\il{Savosavo}, Aji\"e\il{Aji\"e} and Nias\il{Nias}) pattern similarly to the African languages in marking the subject of nominal predications with the standard subject-case and leaving the predicate nominal\is{nominal predication!predicate nominal} zero-coded. 

This pattern is illustrated by the Savosavo\il{Savosavo} examples in (\ref{SavNomPred}a, b). 
However, \citet[212]{Wegener:2008} notes that the Nominative\is{case!individual forms!nominative}
case-marking on the subject noun is often dropped in this type of clause, as is exemplified in
(\ref{SavNomPred}c).%
\enlargethispage{\baselineskip}

\begin{exe}\ex\label{SavNomPred}\langinfo{Savosavo}{Solomons East Papuan; Solomon Islands}{\citealp[210, 222, 214]{Wegener:2008}}
\begin{xlist}
\ex\gll Ururu=gha lava ko-va zuba\textbf{=na}\\
be.fragrant=\pl{} \propr{}.\sg{}.\mas{} 3\sg{}.\fem{}-\gen{}.\mas{} child=\nom{}\\
\glt `[Talking about eggs of a megapode] Her child (i.e. egg) has a nice smell (when
cooked).',\\
lit. `Fragrance having (is) her child.'
\ex\gll ghoma lo mapa=e  ai lo biti\textbf{=na}\\
\Neg{} \deter{}.\sg{}.\mas{} person=\emphat{} this \deter{}.\sg{}.\mas{} volcano=\nom{}\\
\glt `(It was) not a conscious being, this volcano.'
%\ex Mapa batu=e te lo-va seu=na\\
%person head=\emphat{} \emphat{} 3\sg{}.\mas{}-\gen{}.\mas{} container=\nom{}\\
%`Human heads (were) his cup.' 
\ex\gll  anyi ghajia Solomone sua mapa\\
1\sg{} self Solomo\_Islands \att{}.\sg{}.\mas{} person\\
\glt `I was the only Solomon Islander.'\\\nopagebreak[4]
lit. `I myself (was) a Solomon Island person.'
\end{xlist}
\end{exe}

The marked"=absolutive language Nias\il{Nias} has a parallel pattern of zero-coding the predicate nominal\is{nominal predication!predicate nominal}, while the subject of the nominal predication receives overt marking (\ref{NiaNomPred}).\footnote{Recall that the so-called nominal mutation in Nias\il{Nias} is used for S and P arguments, while A arguments are in the basic non-mutated form \citep{Brown:2001}.}  

\begin{exe}
\ex\label{NiaNomPred}\langinfobreak{Nias}{Sundic, Western Malayo-Polynesian, Austronesian; Sumatra, Indonesia}{\citealp[443]{Brown:2001}}
\gll a-me'e-la \textbf{ganunu}-a ha'a\\
\ipfv{}-give-\NR{} \ipfv{}.\mut{}.burn-\nmlz{} \prox{}\\
\glt `This pan was a gift.'
\end{exe}

Apart from the noted tendency of Savosavo\il{Savosavo} to leave the subject zero-coded, there is another type of nominal predicate clause without Nominative case-mark\-ing on the subject (\ref{SavoNomPredZeroS}). 
In Nias\il{Nias}, a similar structure exists with subject of nominal predications in the Unmutated case (\ref{NiaNomPredZeroS}). 
For both languages, the respective context involves a high discourse prominence of the subject of the nominal predication.  
This behavior is, however, not restricted to nominal predications as such. 
There is a general tendency of marked"=S languages to use the zero-coded form of a noun if the noun is emphasized (see Chapter~\ref{emphaticS}).

\begin{exe}\ex\label{SavoNomPredZeroS}\langinfobreak{Savosavo}{}{\citealp[221]{Wegener:2008}}
\gll Ko nini=koi Polupolu\\
3\sg{}.\fem{}.\gen{} name=\emphat{} \deter{}.\sg{}.\fem{} Polupolu\\
\glt `Her name (was) Polupolu.'
\end{exe}


\begin{exe}
\ex\label{NiaNomPredZeroS}\langinfobreak{Nias}{}{\citealp[444]{Brown:2001}}
\gll a-nunu-a ha'a, a-me'e-la\\
\ipfv{}-burn-\nmlz{} \prox{} \ipfv{}-give-\nmlz{}\\
\glt `This pan, (it was) a gift.'
\end{exe}

Nominal predications are not discussed as a construction in the descriptions of Aji\"e\il{Aji\"e}. 
I found only two examples of it in the data (\ref{AjiNomPred1},~\ref{AjiNomPred2}), both of which do not have the subject expressed as an independent nominal.
The third singular form in example (\ref{AjiNomPred1}) is the pre-verbal subject-marker rather than the independent form of a third person pronoun \emph{ce}. 
In (\ref{AjiNomPred2}), one also finds the subject agreement marker for the first person rather than the independent form \emph{\textipa{gE-\textltailn a}}.
The only generalization for Aji\"e\il{Aji\"e} thus must be that predicate nominals\is{nominal predication!predicate nominal} are zero-coded (at least as one of the options of the language), while the marking of the subject remains unknown so far.  


\begin{exe}\ex\label{AjiNomPred1}\langinfo{Aji\"e}{Oceanic, Eastern Malayo-Polynesian; Austronesian; New Caledonia}{\citealt[99, 103]{Lichtenberk:1978} after \citealt[264, 210, 211]{Fontinelle:1976}}
\begin{xlist}
\ex\gll \textipa{na} \textipa{dO} \textipa{pani-\textltailn{a}}\\
3\sg{} \intens{} mother-1\sg{}\\
\glt `She is my true mother.'
 
\ex\label{AjiNomPred2}\gll {\rm (}\textipa{ki\/}{\rm)} \textipa{gOi} \textipa{OrOka\textglotstop u}\\
\hspaceThis{(}\Hyp{} 1\sg{} chief\\
\glt `I wish I were chief'
\end{xlist} 
\end{exe}


\begin{table}[t]
\centering
\begin{tabular}{lcc}
\hline \hline
\bfseries language&\bfseries subject &\bfseries predicate nominal\\
\hline
Aji\"e\il{Aji\"e}&{-}&\acc{}\\
%\hdashline
Nias\il{Nias}&\textbf{\abs{}}&\erg{}\\
%\hdashline
Savosavo\il{Savosavo}&\textbf{\nom{}}/\acc{}&\acc{}\\
\hline \hline
\end{tabular}
\caption{Marking of nominal predication in the marked"=S languages of the Pacific}\label{OverviewNomPredPac}%\\
\end{table}

The data from the marked"=S languages of the Pacific region are summarized in Table~\ref{OverviewNomPredPac}. 
All three language of that region have zero-coded predicate nominals. 
The subject of nominal predications can be coded in the S-case, which is also used also for subjects of intransitive clauses in Savosavo\il{Savosavo} and Nias\il{Nias} (and possibly also in Aji\"e\il{Aji\"e}). However, at least in Savosavo\il{Savosavo} zero-coded subjects are often found. 

%%%%%%%%%%%%%%%%%%%%%%%%%
%%%%% Section 3.9 %%%%%%%
%%%%%%%%%%%%%%%%%%%%%%%%%

\section{Summary}\label{NomPredOverview}

Table~\ref{OverviewNomPred} summarizes all data given on the marking of nominal predications in marked"=S languages in the above sections. 
For each language, the case-form used for subjects of nominal predications (shortened to `subject' in the table) as well as the predicate nominal\is{nominal predication!predicate nominal} are listed.
In addition, I list the information on whether or not an overt copular element\is{copula!absence versus presence|(} is used in the construction. 
If a language has alternative constructions for the encoding of nominal predication, e.g. one with an overt copula and one without, each construction has its own line in the table. 
Supposedly free variation of case-marking on either of the arguments that cannot be pinned down to any clear conditions, such as: \nom{} in copula clauses, \acc{} in copula-less clauses, is represented with a slash in the respective cell\is{copula!absence versus presence|)}. 

%\enlargethispage{\baselineskip}
Most genealogical units of languages behave rather uniformly with respect to case-marking in nominal predications. 
For some of the languages with a deviating pattern, this is conditioned by other structural properties of the construction such as presence or absence of a copula (Turkana\il{Turkana} and Havasupai\il{Havasupai}), or a difference in the case inventory (Arbore\il{Arbore}'s Predicative case). 
Some languages behave differently from genealogically related languages without there being a base for this in any apparent structural conditions (Daatoga and Jamul\il{Jamul Tiipay} Tiipay). 
Also, there is in general no correlation between whether a copula\is{copula!absence versus presence} is obligatory, optional, or never present in a language and the case-marking found in nominal predications -- though for individual languages, such as Turkana\il{Turkana} and Havasupai\il{Havasupai}, this may be different.  

\begin{table}[ht]
\centering
\resizebox{\textwidth}{!}{
\begin{tabular}{lccc}
\hline \hline
\bfseries language&\bfseries subject&\bfseries pred. nominal&\bfseries zero copula\\
\hline
Aji\"e\il{Aji\"e}&{-}&\acc{}&possible\\
%\hdashline
Arbore\il{Arbore}&\textbf{\nom{}}&\textbf{\pred}&possible\\
%\hdashline
Datooga\il{Datooga}&\acc{}&\acc{}&possible\\
%\hdashline
Diegue\~no\il{Diegue\~no (Mesa Grande)} (Mesa Grande) &\acc{}&\textbf{\nom{}}&possible\\
%\hdashline
Gamo\il{Gamo}&\textbf{\nom{}}&\acc{}&possible\\
%\hdashline
Havasupai\il{Havasupai} (\textit{construction 1})&\acc{}&\textbf{\nom{}}&no\\
{Havasupai\il{Havasupai} (\textit{construction 2})}&\acc{}&\acc{}&always\\
%\hdashline
Jamul\il{Jamul Tiipay} Tiipay (\textit{construction 1})&\acc{}&\acc{}&always\\
{Jamul\il{Jamul Tiipay} Tiipay (\textit{construction 2})}&\textbf{\nom{}}&\acc{}&never\\
%%\hdashline
K'abeena\il{K'abeena}&\textbf{\nom{}}&\acc{}&no\\
%\hdashline
Maa\il{Maa}&\textbf{\nom{}}&\acc{}&no\\
%\hdashline
Maidu\il{Maidu}&\textbf{\nom{}}&\textbf{\nom{}}&no\\
%\hdashline
Maricopa\il{Maricopa}&\acc{}&\textbf{\nom{}}&no?\\
%\hdashline
Mojave\il{Mojave}&\acc{}&\textbf{\nom{}}&possible\\
%\hdashline
Murle\il{Murle}&\textbf{\nom{}}&\acc{}&yes/always\\
%\hdashline
Nandi\il{Nandi}&\textbf{\nom{}}&\acc{}&always\\
%\hdashline
Nias\il{Nias}&\textbf{\abs{}}&\erg{}&always\\
%\hdashline
Oromo (Boraana\il{Oromo (Boraana)})&\textbf{\nom{}}&\acc{}&possible\\
%\hdashline
Oromo (Harar\il{Oromo (Harar)})&\textbf{\nom{}}&\acc{}&possible\\
%\hdashline
Savosavo\il{Savosavo}&\textbf{\nom{}}/\acc{}&\acc{}&always\\
%\hdashline
Tennet\il{Tennet} (\emph{class-membership})&\textbf{\nom{}}&\acc{}&no\\
Tennet\il{Tennet} (\emph{identity})&\acc{}&\acc{}&yes\\
%\hdashline
Turkana\il{Turkana} (\emph{construction 1})&\textbf{\nom{}}&\acc{}&no\\
Turkana\il{Turkana} (\emph{construction 2})&\acc{}&\acc{}&always\\
%\hdashline
Walapai\il{Walapai}&\acc{}&\textbf{\nom{}}&no\\
%\hdashline
Wappo\il{Wappo}&\acc{}&\acc{}&only future\\
%\hdashline
Wolaytta\il{Wolaytta}&\textbf{\nom{}}&\acc{}/\textbf{\nom{}}&restricted\\
%\hdashline
Yavapai\il{Yavapai}&\acc{}&\textbf{\nom{}}&no\\
%\hdashline
Zayse\il{Zayse}&\textbf{\nom{}}&\acc{}&no\\
\hline \hline
\end{tabular}
}
\caption{Overview of the marking of nominal predication}\label{OverviewNomPred}%\\
\end{table}

                                                          
