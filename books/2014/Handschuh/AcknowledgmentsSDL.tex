\addchap{Acknowledgments}

%%%%%%%%%%%%%%%%%%%%%%%NEW

This book started out as a PhD dissertation at Universit\"at Leipzig by the same title, which I submitted in December 2010 and successfully defended in July 2011. 
The original dissertation has undergone minor revisions, but its contents have in essence remained the same.

Many people have contributed to the work in its current state, : to all of whom I would like to express my gratitude.
The original research started out as part of the  project `Marked-absolutive and marked-nominative case systems in synchronic and diachronic perspective' of the \textit{Forschergruppe} `Grammatik und Verarbeitung verbaler Argumente' in Leipzig. 
Financial support for my research has been provided by the Deutsche Forschungsgemeinschaft through a grant to this project.
Michael Cysouw\aimention{Cysouw, Michael}, who was the principal investigator of the project, has patiently accompanied the development of the dissertation from the very beginning. 
He has always provided me with much valued feedback on all aspect of my work from methodological discussion to stylistic comments.  
Without his encouragement, this work would probably not have been completed.
The referees for the dissertation were Martin Haspelmath\aimention{Haspelmath, Martin} and Helen de Hoop\aimention{{de Hoop}, Helen}.
Their comments during and after the writing process, respectively, have been much appreciated and greatly improved the final result. 
 
The research for this book work was carried out during the heyday of the Leipzig typology community.
I was lucky to have had the opportunity to conduct my research as a member of the Max Planck Institute for Evolutionary Anthropology's linguistics department.
Its excellent facilities and stimulating research environment have contributed much to the completion of this work. 
Through the \textit{Forschergruppe}, I also had the chance to collaborate with the larger linguistic community both at the University of Leipzig and the Max Planck Institute for Human Cognitive and Brain Science.

During this time, I presented parts of my work at various occasions in the MPI's `work in progress' series, at the meetings and workshops of the \textit{Forschergruppe} and at `Typologiekolloquium' at Universit\"at Leipzig.
Through these and through the regular interaction with the members of the Leipzig linguistic community and the numerous guests over the years, my work has been greatly inspired.
In particular, I would like to mention (in alphabetical order): Balthasar Bickel\aimention{Bickel, Balthasar}, Bernard Comrie\aimention{Comrie, Bernard}, Tom G\"uldemann\aimention{G{\"u}ldemann, Tom}, Andrej Malchukov\aimention{Malchukov, Andrej L.}, Gereon M\"uller, Jochen Trommer, S\o ren Wichmann, and Alena Witzlack-Makarevich\aimention{Witzlack-Makarevich, Alene}, with whom I also taught a course at \emph{Leipzig Springschool on Linguistic Diversity} in 2008. 
The preparations for this course have been very valuable for my research. 
In addition, I would like to thank the following people for providing information on the languages of their expertise:
Lea Brown\aimention{Brown, Lea} on Nias\il{Nias}, Gerrit Dimmendaal\aimention{Dimmendaal, Gerrit} on Turkana\il{Turkana}, and Claudia Wegener\aimention{Wegener, Claudia} on Savosavo\il{Savosavo}.
Another important factor in completing a work of this extent are your fellow sufferers, a.k.a the other PhD students.
For sharing the woes of PhD life, more often a laugh and the occasional drink, I want to thank my peers at the MPI: Joseph Atoyebi, Joseph Farquharson, Diana Forker, Linda Gerlach, Thomas Goldammer, Iren Hartmann, Hagen Jung, Zaira Khalilova, Nina Kottenhagen, Christfried Naumann, Andrey Nefedov, Sven Siegmund, Eugenie Stapert, Matthias Urban, and Jan Wohlgemuth.

I would further like to thank the audiences of my presentations at the following conferences, where parts of this work were presented:
\emph{NAWUKO} 2006 at Konstanz,  \emph{ALT} 2007 at Paris, \emph{Syntax of the World's languages} 2008 at Berlin and \emph{Case in and across languages} 2009 at Helsinki. 
I also would like to thank the organizers of those conferences.

Finally, there are the people who contributed to the publication process of this very book you are holding in your hands (or more likely seeing on your screen). 
I am very glad for the opportunity to publish a revised and updated version of my original thesis as one of the first books of Language Science Press.
I must thank Martin Haspelmath for accepting my book for the series Studies in Diversity Linguistics and thus making my work widely available.
Being one of the first book being published meant witnessing many, and starting some, discussions on formatting.  
Martin Haspelmath and Stefan M\"uller were very helpful in answering my questions on all kinds of issues from alphabetization to the correct usage of hyphenation in large compounds.
Together with Timm Lichte, Stefan M\"uller also reliably and quickly gave feedback on the more technical issues with the \LaTeX -style.  
The tedious work of proof-reading has been done by Eitan Grossman, Daniel W. Hieber and Aaron Huey Sonnenschein. 
All three of them have done an incredible job in a very short time. 
While I was preparing the book for publication, I could rely on the great atmosphere at the linguistics department at Universität Regensburg. 
Special thanks go to my colleagues there, Martine Bruil, Christian Rapold and Zarina Molochieva, for their support in many ways. 

\enlargethispage{\baselineskip}
Needless to say, all shortcomings, omissions and remaining fallacies of this work are my own responsibility.



 
